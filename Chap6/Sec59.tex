\section{質量拡散と熱拡散の係数}\label{sec:59}

拡散フラックス$\vec{i}$と熱フラックス$\vec{q}$は,流体に濃度や温度の勾配が存在することにより生じる.
しかし,$\vec{i}$が濃度勾配のみに,$\vec{q}$が温度勾配のみに依存すると考えてはならない.
実際には,各フラックスは両方の勾配に依存する.


濃度や温度の勾配が小さい場合には,$\vec{i}$や$\vec{q}$が$\Grad \mu$と$\Grad T$の1次関数であるとしてよい
(\S~49と同様の理由で,$\vec{i}$や$\vec{q}$は圧力勾配には依存しない).そこで
\[
    \vec{i} = -\alpha \Grad \mu - \beta \Grad T, \quad 
    \vec{q} = - \delta \Grad \mu - \gamma \Grad T + \mu\vec{i}
    \mytag{1}
\]
と書こう.

係数$\beta,\delta$の間には単純な関係が成り立つ.
これは輸送係数に関する\emph{Onsagerの相反定理}の帰結であり,その概略は以下の通りである(『統計物理学』\S~120参照).

ある閉じた系を考え,$x_a \; (a=1,2,3,\ldots)$を系の状態を記述する量としよう.
統計力学的平衡状態における$x_a$の値は,系全体のエントロピー$S$が最大になるという条件により決定される.
つまり,
\begin{equation}
    X_a \equiv - \pdv{S}{x_a}
\end{equation}
とするとき,全ての$a$に対して$X_a=0$でなければならない.


系は平衡状態に近い状態にあると仮定する.
つまり,全ての$x_a$は平衡状態での値からわずかに異なるだけで,$X_a$は小さいとする.
この系では,系を平衡状態に向かわせるような過程が起こる.
$x_a$の時間変化を$X_a$の級数で展開して1次まで取る.
\begin{equation}\label{eq59.2:xaの時間微分の展開}
    \dot{x_a} = - \sum_b \gamma_{ab} X_b
\end{equation}
相反定理は,\emph{輸送係数}$\gamma_{ab}$が添字$a,b$に関して対称であることを主張する
(証明は『統計物理学』を見よ).
\begin{equation}
    \gamma_{ab} = \gamma_{ba}
\end{equation}
また\eqref{eq59.2:xaの時間微分の展開}より,エントロピーの時間変化率は
\begin{equation}\label{eq59.4:Sの時間微分の展開}
    \dot{S} = \sum_a \pdv{S}{x_a} \dot{x_a} = - \sum_a X_a \dot{x_a}
\end{equation}
で与えられる
(もし$x_a$が座標の関数なら,$a$に関する和だけでなく,系の体積での積分も行わなければならない).



今考えている問題では,$\dot{x_a}$として$\vec{i}, \vec{q}-\mu\vec{i}$の成分を取ることにする.
すると,\eqref{eq58.7:混合流体でのエントロピー変化(積分形)}と\eqref{eq59.4:Sの時間微分の展開}を見比べることにより,
$X_a$は$\dfrac{1}{T} \Grad\mu, \dfrac{1}{T^2} \Grad T$の成分に対応することが分かる.
よって\ajMaru{1}を
\[
    \vec{i} = -\alpha T \cdot \frac{1}{T} \Grad \mu - \beta T^2 \cdot \frac{1}{T^2} \Grad T, \quad 
    \vec{q}-\mu\vec{i} = - \delta T \cdot \frac{1}{T} \Grad \mu - \gamma T^2 \cdot \frac{1}{T^2} \Grad T 
\]
と書くことにより,輸送係数の対称性から
\[
    \beta T^2 = \delta T 
    \qquad \yueni \delta = \beta T
\]
が成り立つことが分かり,\ajMaru{1}は
\begin{equation}
    \vec{i} = -\alpha \Grad \mu - \beta \Grad T, \quad 
    \vec{q} = - \beta T \Grad \mu - \gamma \Grad T + \mu\vec{i}
\end{equation}
となる.
第1式より$\Grad \mu = - \dfrac{1}{\alpha}(\vec{i} + \beta \Grad T)$であり,これを第2式に代入して
\[
    \vec{q} = \frac{\beta T}{\alpha}(\vec{i} + \beta \Grad T) - \gamma \Grad T + \mu\vec{i}.
\]
よって
\begin{equation}\label{eq59.6:iとqの展開}
    \vec{i} = -\alpha \Grad \mu - \beta \Grad T, \quad
    \vec{q} = \left( \mu + \frac{\beta T}{\alpha} \right) \vec{i} - \kappa \Grad T ,
\end{equation}
\begin{equation}
    \kappa = \gamma - \frac{\beta^2 T}{\alpha}
\end{equation}
となる.


もし拡散フラックス$\vec{i}$が0なら,\emph{純粋な熱伝導}となる.
これは$\alpha d\mu + \beta dT =0$のときに成り立つ.
これを積分することにより,拡散フラックスが0の時の濃度の温度依存性$f(c, T)=0$が得られる.
また,$\vec{i}=\vec{0}$なら$\vec{q}= -\kappa \Grad T$であるから,$\kappa$は熱伝導度であることが分かる.



さて,変数として$p,T,c$を選ぶと
\[
    \Grad \mu = \PDV{\mu}{c}{p,T} \Grad c + \PDV{\mu}{T}{c,p} \Grad T + \PDV{\mu}{p}{c,T} \Grad p
\]
が成り立つ.
Gibbsの自由エネルギーから導かれるMaxwell関係式 $\dPDV{\mu}{p}{c,T} = \dPDV{V}{c}{p,T}$($V$は比体積)も用いると
\begin{align*}
    \vec{i} &= - \alpha \left[ \PDV{\mu}{c}{p,T} \Grad c + \PDV{\mu}{T}{c,p} \Grad T 
    + \PDV{V}{c}{p,T} \Grad p \right] - \beta \Grad T \\
    &= - \alpha \PDV{\mu}{c}{p,T} \left[ \Grad c 
    + \frac{ \alpha\PDV{\mu}{T}{c,p} + \beta }{ \alpha \PDV{\mu}{c}{p,T} } \Grad T 
    + \frac{ \PDV{V}{c}{p,T} }{ \PDV{\mu}{c}{p,T} } \Grad p \right].
\end{align*}
ここで
\setcounter{equation}{8}
\begin{equation}\label{eq59.9:D,k_Tの定義}
    D = \frac{\alpha}{\rho} \PDV{\mu}{c}{p,T}, \quad
    \frac{\rho k_T D}{T} = \alpha\PDV{\mu}{T}{c,p} + \beta,
\end{equation}
\begin{equation}
    k_p = p \frac{ \PDV{V}{c}{p,T} }{ \PDV{\mu}{c}{p,T} }
\end{equation}
により係数$D, k_T, k_p$を定義すると
\begin{equation}\label{eq59.11:iの表現}
    \vec{i} = -\rho D \left( \Grad c + \frac{k_T}{T} \Grad T + \frac{k_p}{p} \Grad p \right)
\end{equation}
となる.
また\eqref{eq59.9:D,k_Tの定義}から
\[
    \beta = \frac{k_T}{T} \alpha \PDV{\mu}{c}{p,T} - \alpha \PDV{\mu}{T}{c,p}
\]
\[
    \yueni \frac{\beta T}{\alpha} = k_T \PDV{\mu}{c}{p,T} - T \PDV{\mu}{T}{c,p}
\]
であるから
\begin{equation}\label{eq59.12:qの表現}
    \vec{q} = \left[ k_T \PDV{\mu}{c}{p,T} - T \PDV{\mu}{T}{c,p} + \mu \right] \vec{i} - \kappa \Grad T
\end{equation}
を得る.



係数$D$は\emph{拡散係数}または\emph{質量輸送係数}と呼ばれ,(温度や圧力の勾配がなく)濃度勾配のみが存在するときの拡散フラックスを与える.
温度勾配による拡散フラックスは\emph{熱拡散係数} $k_T D$により与えられ,無次元数$k_T$は\emph{熱拡散比}と呼ばれる.



\eqref{eq59.11:iの表現}の最後の項は,流体中に(例えば外力場によって作られた)顕著な圧力勾配が存在する場合にのみ考慮される.
係数$k_pD$は\emph{圧拡散係数}と呼ぶことができる.


1成分の流体では,もちろん拡散フラックスは存在しない.
よって$c \to 0$または1の極限では,$k_T$と$k_p$は0になる.


エントロピーが増加しなければならないという条件は,\eqref{eq59.6:iとqの展開}の係数に制約を加える.
\eqref{eq58.7:混合流体でのエントロピー変化(積分形)}に代入することにより
\begin{align}
    \pdv{t} \int \rho s \,\dV &= \int \frac{\Phi}{T} dV
    - \int \frac{ \left( \cancel{ \frac{\beta T}{\alpha}\vec{i} } - \kappa \Grad T \right) \cdot\Grad T }{T^2} dV 
    - \int \frac{ \vec{i}\cdot \left( -\frac{1}{\alpha} \right) (\vec{i} + \cancel{\beta \Grad T} ) }{T} dV \nonumber \\
    &= \int \frac{\Phi}{T} dV + \int \frac{ \kappa (\Grad T)^2 }{T^2} dV + \int \frac{ \vec{i}^2 }{\alpha T} dV .
\end{align}
よって,$\kappa>0$に加えて$\alpha>0$でなければならない.
熱力学不等式$\dPDV{\mu}{c}{p,T}>0$(『統計物理学』\S~96を見よ)より,拡散係数$D$も正でなければならない.
しかし,$k_T,k_p$は正にも負にもなりうる.




以上の$\vec{i}$や$\vec{q}$の表式を\eqref{eq58.3:1成分に対する連続の式}や\eqref{eq58.6:混合流体でのエントロピー変化}に代入すれば,
混合流体の方程式系が完成するが,長くなるからやめておく.
ここでは,圧力勾配が大きくない場合を考える.
さらに,濃度や温度の変化が小さいために,(一般には$c$や$T$の関数である)
\eqref{eq59.11:iの表現}\eqref{eq59.12:qの表現}に現れる係数は定数とみなせると仮定する.
さらに,濃度や温度の勾配による流れを除けば,流体のマクロな運動は存在しないと仮定する.
この場合の速度は$\Grad c$や$\Grad T$に比例するから,\eqref{eq58.3:1成分に対する連続の式}や\eqref{eq58.6:混合流体でのエントロピー変化}で
$\v$が含まれる項は2次以上として無視することができる.
また\eqref{eq58.6:混合流体でのエントロピー変化}で$\vec{i}\cdot\Grad\mu$も2次のオーダーである.
よって
\[
    \begin{cases}
        \rho \dpdv{c}{t} &= -\Div \vec{i} \\[8pt]
        \rho T \dpdv{s}{t} &= - \Div(\vec{q} - \mu \vec{i})
    \end{cases}
    \mytag{2}
\]
となる.

第1式は
\[
    \rho\pdv{c}{t} = - \Div \left[ -\rho D \left( \Grad c + \frac{k_T}{T} \Grad T \right) \right]
\]
\begin{equation}\label{eq59.14:cの拡散方程式}
    \yueni \pdv{c}{t} = D \left( \Laplacian c + \frac{k_T}{T} \Laplacian T \right)
\end{equation}
となる($1/T$の微分は高次のオーダーとして現れるから無視することができる).
次に,\ajMaru{2}の第2式の$\dpdv{s}{t}$を書き換えよう.
Maxwell関係式$-\dPDV{s}{c}{p,T} = \dPDV{\mu}{T}{p,c}$を用いると
\[
    \pdv{s}{t} = \PDV{s}{T}{c,p} \pdv{T}{t} + \PDV{s}{c}{T,p} \pdv{c}{t}
    = \frac{c_p}{T} \pdv{T}{t} - \PDV{\mu}{T}{p,c} \pdv{c}{t} .
\]
よって第2式は
\[
    \rho T \left[ \frac{c_p}{T} \pdv{T}{t} - \PDV{\mu}{T}{p,c} \pdv{c}{t} \right]
    = -\Div \left[ \left\{ k_T \PDV{\mu}{c}{p,T} - T \PDV{\mu}{T}{c,p} \right\}
    \left\{ -\rho D \left( \Grad c + \frac{k_T}{T} \Grad T \right) \right\} - \kappa \Grad T \right] .
\]
やはり$T$の微分を無視することができるから
\[
    \rho c_p \pdv{T}{t} - \rho T \PDV{\mu}{T}{p,c} \pdv{c}{t}
    = \rho D \left\{ k_T \PDV{\mu}{c}{p,T} - T \PDV{\mu}{T}{c,p} \right\}
    \left( \Laplacian c + \frac{k_T}{T} \Laplacian T \right) + \kappa \Laplacian T 
\]
右辺の$\left( \Laplacian c + \dfrac{k_T}{T} \Laplacian T \right)$を\eqref{eq59.14:cの拡散方程式}を用いて書き換えれば
\[
    \rho c_p \pdv{T}{t} - \cancel{ \rho T \PDV{\mu}{T}{p,c} \pdv{c}{t} }
    = \rho \left\{ k_T \PDV{\mu}{c}{p,T} - \cancel{ T \PDV{\mu}{T}{c,p} } \right\} \pdv{c}{t}  + \kappa \Laplacian T .
\]
\begin{equation}\label{eq59.14:Tの拡散方程式}
    \yueni \pdv{T}{t} - \frac{k_T}{c_p} \PDV{\mu}{c}{p,T} \pdv{c}{t} = \chi \Laplacian T .
\end{equation}
線形方程式系\eqref{eq59.14:cの拡散方程式}\eqref{eq59.14:Tの拡散方程式}は,流体中の濃度と温度の分布を与える.



濃度が小さい場合は特に重要である.
$c \to 0$では$k_T \to 0$(しかし$D$は有限のまま)であるから
\begin{equation}\label{eq59.16:純粋な拡散方程式}
    \pdv{c}{t} = D \Laplacian c
\end{equation}
および$\dpdv{T}{t} = \chi \Laplacian T$となる.


\eqref{eq59.16:純粋な拡散方程式}の境界条件は,場合によって異なる.
\begin{itemize}
    \item 物体の表面が流体に溶けない場合.拡散フラックス$\vec{i} = -\rho D \Grad c$の法線成分は0でなければならないから$\dpdv{c}{n}=0$である.
    \item 物体が流体に溶けることができ,物体から拡散がある場合.表面付近では速やかに平衡状態になり,流体の濃度は飽和濃度$c_0$に等しくなる.つまり境界条件は$c=c_0$である.
    \item 固体表面が拡散物質を吸収する場合.$c=0$となる.これは例えば,固体表面で化学反応が起こる場合に見られる.
\end{itemize}



純粋な拡散方程式\eqref{eq59.16:純粋な拡散方程式}は純粋な熱伝導方程式と全く同じ形をしている.
よって\S~51-52で得られた式を,$T$を$c$に,$\chi$を$D$に置き換えるだけで当てはめることができる.
断熱の境界条件は不溶表面の条件に,一定温度の境界条件は表面での溶出の条件に対応する.


特に(51.5)との類似から,$t=0$で溶質が原点に集中していた場合の,$t>0$での濃度分布を書き下すことができる.
\begin{equation}\label{eq59.17:拡散方程式のGreen関数}
    c(r,t) = \frac{M}{8\rho(\pi Dt)^{3/2}} \exp\left( - \frac{r^2}{4Dt} \right)
\end{equation}
ここで$M$は溶質の全質量である.



問題などは省略する.


\BackToTheToc