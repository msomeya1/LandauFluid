\section{懸濁粒子の拡散}


流体中の分子運動により,懸濁粒子は不規則に運動する(\emph{Brown運動}).
$t=0$で粒子が原点にあるとしよう.
粒子のその後の運動は拡散とみなすことができ,濃度は,ある体積要素内に粒子が存在する確率を表す.
その確率を求めるために,拡散方程式の解\eqref{eq59.17:拡散方程式のGreen関数}を用いよう
(この方法を用いることができるのは,希薄溶液の拡散では,各粒子が互いにほとんど影響を及ぼさず,各粒子の運動が独立であるとみなせるからである).


時刻$t$に,原点からの距離が$r \sim r+dr$の領域に粒子が存在する確率を$w(r,t) \, dr$としよう.
\eqref{eq59.17:拡散方程式のGreen関数}で$M/\rho=1$とし,球殻の体積$4\pi r^2 \, dr$をかければ
\begin{equation}
    w(r,t) \, dr = \frac{1}{2\sqrt{\pi(Dt)^3}} \exp\left( -\frac{r^2}{4Dt} \right) r^2 \, dr .
\end{equation}
平均二乗距離
\begin{equation}
    \overline{r^2} \equiv \int_{0}^{\infty} r^2 w(r,t) \, dr
\end{equation}
は
\begin{align}
    \overline{r^2} &= \frac{1}{2\sqrt{\pi(Dt)^3}} \int_{0}^{\infty} r^4 \exp\left( -\frac{r^2}{4Dt} \right) \, dr \nonumber \\
    & \gyoukan{←積分公式 $\displaystyle \int_{0}^{\infty} x^{2n} e^{-ax^2} \, dx = \dfrac{(2n-1)!!}{2^{n+1}} \sqrt{\dfrac{\pi}{a^{2n+1}}} $ } \nonumber \\
    &= \frac{1}{2\sqrt{\pi(Dt)^3}} \cdot \frac{3}{8} \sqrt{\pi(4Dt)^5} = 6Dt. \label{eq60.3:Brown運動の平均二乗距離}
\end{align}
よって粒子の平均移動距離は$\sqrt{t}$に比例する.


懸濁粒子の拡散係数は\emph{移動度}と呼ばれるものから計算することができる.
粒子に一定の外力$\vec{f}$(例えば重力)が働いているとしよう.
定常状態では,各粒子に働く力は運動する粒子に働く抗力と釣り合わなければならない.
速度が小さい場合には,抗力は速度に比例し,$\v/b$と書ける($b$は定数).
これを外力$\vec{f}$と等しいと置いて
\begin{equation}
    \v = b \vec{f}
\end{equation}
を得る.
よって,外力の作用により生じる速度は外力に比例する.
$b$は\emph{移動度}と呼ばれ,原理的には流体力学の方程式から計算することができる.
例えば,半径$R$の球に働く抗力は((20.14)から)$6\pi \eta Rv$であるから,移動度は
\begin{equation}\label{eq60.5:球状粒子の移動度}
    b = \frac{1}{6\pi\eta R}
\end{equation}
となる.
粒子が球状でない場合には,抗力と速度の方向は異なり,$f_i = a_{ij} v_j$となる.
$a_1, a_2, a_3$を$a_{ij}$の主値とすると
\begin{equation}
    b = \frac{1}{3} \left( \frac{1}{a_1} + \frac{1}{a_2} + \frac{1}{a_3} \right).
\end{equation}


さて,移動度$b$と拡散係数$D$の関係を導こう.
そのために拡散フラックス$\vec{i}$を書き下す.
(温度が一定と仮定すれば)濃度勾配による拡散フラックスは$-\rho D \Grad c$である.
また,外力によって生じる速度$\v$の寄与は$\rho c\v = \rho cb \vec{f}$である.よって
\begin{align}
    \vec{i} &= -\rho D \Grad c + \rho cb \vec{f} \\
    &= - \frac{\rho D}{\PDV{\mu}{c}{T,p}} \Grad \mu + \rho cb \vec{f} \nonumber
\end{align}
となる.
$\mu$は懸濁粒子の化学ポテンシャルであり,その濃度依存性は(希薄溶液の場合)
\[
    \mu = k_B T \log c + \varPsi(p,T)
\]
であるから(『統計物理学』\S~87),
\[
    \vec{i} = - \frac{\rho Dc}{k_BT} \Grad \mu + \rho cb \vec{f}.
\]
熱力学的平衡状態では拡散がなく$\vec{i}=\vec{0}$である.
一方,外力が存在する場合の平衡条件は,懸濁粒子のポテンシャルエネルギーを$U$として
$\mu+U=\const$すなわち$\Grad\mu=-\Grad U = \vec{f}$である.したがって
\[
    \left( - \frac{\rho Dc}{k_BT} + \rho cb \right) \vec{f} = \vec{0}
\]
\begin{equation}
    \yueni D = k_BT b.
\end{equation}
これが拡散係数と移動度の間の\emph{Einsteinの関係式}である.
特に球状粒子の場合,\eqref{eq60.5:球状粒子の移動度}を代入し
\begin{equation}\label{eq60.9:Stokes-Einsteinの関係式}
    \yueni D = \frac{k_BT}{6\pi\eta R}
\end{equation}
を得る(\emph{Stokes-Einsteinの関係式}).
この式はAvogadro数の測定に使われた,歴史的に重要な式である.


懸濁粒子の並進Brown運動と拡散の関係を考えたのと同様に,回転Brown運動と拡散の関係を考えることができる.
並進拡散係数が抗力から計算されたように,回転拡散係数は回転する粒子に働く力のモーメントから計算することができる.




%%%%%%%%%% 問題1 %%%%%%%%%%

\begin{mondai}{}{}
1次元Brown運動において,粒子は壁$x=0$に到達すると吸着して止まるものとする.
$t=0$で$x=x_0$にある粒子が時刻$t$までに壁に到達する確率を求めよ.
\end{mondai}
\begin{kaitou}
粒子の存在確率$w(r,t)$は,
\begin{itemize}
    \item 初期条件:$t=0$で$w=\delta(x-x_0)$
    \item 境界条件:$x=0$で$w=0$
\end{itemize} 
のもとで拡散方程式を解くことで得られる.
式(52.4)で$T_0(x) \to \delta(x-x_0)$として
\[
    w(x, t) = \frac{1}{2\sqrt{\pi Dt}} \left\{ \exp\left[ -\frac{(x-x_0)^2}{4Dt} \right] - \exp\left[ -\frac{(x+x_0)^2}{4Dt} \right] \right\}
\]
となる.


単位時間に壁に吸着する確率は,$x=0$での拡散フラックス$D \eval{\dpdv{w}{x}}_{x=0}$で与えられる.
よって求める確率は
\begin{align*}
    W(t) &= D \int_{0}^{t} \eval{\pdv{w(x,t')}{x}}_{x=0} dt' \\
    &= D \int_{0}^{t} \frac{1}{2\sqrt{\pi Dt'}} \left\{
        - \frac{2(x-x_0)}{4Dt'} \exp\left[ -\frac{(x-x_0)^2}{4Dt'} \right] 
        + \frac{2(x+x_0)}{4Dt'} \exp\left[ -\frac{(x+x_0)^2}{4Dt'} \right] \right\}_{x=0} \, dt' \\
    &= D \int_{0}^{t} \frac{1}{2\sqrt{\pi Dt'}} \cdot \frac{x_0}{Dt'} e^{-{x_0}^2/4Dt'} \, dt' \\
    & \gyoukan{ $\dfrac{x_0}{2\sqrt{Dt'}} = y$ とおくと $-\dfrac{x_0}{4\sqrt{Dt'^3}} dt' = dy$ } \\
    &= \int_{\infty}^{x_0/2\sqrt{Dt}} \left( -\frac{2}{\sqrt{\pi}} e^{-y^2} \, dy \right) 
    = \frac{2}{\sqrt{\pi}} \int_{x_0/2\sqrt{Dt}}^{\infty} e^{-y^2} \, dy\\
    &= \mathrm{erfc} \left( \frac{x_0}{2\sqrt{Dt}} \right) 
    = 1 - \erf \left( \frac{x_0}{2\sqrt{Dt}} \right) .
\end{align*}


\end{kaitou}




%%%%%%%%%% 問題2 %%%%%%%%%%

\begin{mondai}{}{}
懸濁粒子が大きな角度を回転するのにかかる時間のオーダー$\tau$を求めよ.
\end{mondai}
\begin{kaitou}
粒子の大きさを$a$とすると,$\tau$は$a$程度の距離をBrown運動により移動するのにかかる時間である.
よって\eqref{eq60.3:Brown運動の平均二乗距離}から
$\tau \sim \dfrac{a^2}{D}$となる.
\eqref{eq60.9:Stokes-Einsteinの関係式}より$D \sim \dfrac{k_BT}{\eta a}$であるから
\[
    \tau \sim \frac{\eta a^3}{k_BT}.
\]




\end{kaitou}

\BackToTheToc