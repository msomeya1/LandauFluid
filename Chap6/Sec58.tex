\section{混合流体の方程式}

これまでの議論では,流体が完全に一様であると仮定してきた.
各点で組成が異なるような混合流体を考える場合には,流体力学の方程式を変更しなければならない.


ここでは,2つの成分を持つ混合流体のみを扱うことにする.
混合流体の組成は濃度$c$,すなわち体積要素内の流体の全質量に対する,ある成分の質量の比によって記述される.


時間が経つと,一般に濃度の分布は変化する.
これには2つの変化がある.
まず,流体のマクロな運動がある場合には,流体のある一部分はそれ自体として動き,内部の組成は変化しない.
その結果,純粋に力学的な混合が生じる.
運動する各部分の組成は変化しないが,空間の各点における濃度は時間とともに変化する.
実際には,熱伝導や内部摩擦が起こるのだが,これらを無視すれば,濃度の変化は熱力学的な可逆過程であり,エネルギーの散逸を伴わない.

次に,組成の変化は,流体のある部分から別の部分へ成分が分子的に輸送されることでも生じる.
流体の各微小部分の間で起こる,成分の直接的な交換による濃度の均一化の過程は\emph{拡散}と呼ばれる.
拡散は不可逆過程であり,熱伝導や粘性と同様,混合流体中でエネルギーが散逸する原因の1つである.


流体の(全)密度を$\rho$とすると,全質量に対する連続の式は,前と同様
\begin{equation}\label{eq58.1:混合流体の連続の式}
    \pdv{\rho}{t} + \Div(\rho\v) = 0
\end{equation}
である.
この式は,ある体積の流体の全質量は,この体積に流体が出入りすることによってのみ変化することを示している.

厳密には,混合流体では速度の概念そのものを定義し直さなければならないことを強調しておこう.
連続の式を\eqref{eq58.1:混合流体の連続の式}の形に書くことにより,
速度は以前と同様に,単位質量の流体の全運動量として定義される.

Navier-Stokes方程式も前と同じである.
以下では,混合流体の残りの方程式を導くことにしよう.


拡散がない場合には,ある流体要素の組成は動いている間も変化しない.
このことは
\[
    \frac{Dc}{Dt} = \pdv{c}{t} + \v \cdot \Grad c = 0
\]
を意味している.\eqref{eq58.1:混合流体の連続の式}を用いれば
\[
    \pdv{(\rho c)}{t} + \Div(\rho c\v) = 0
\]
となり,ある成分の質量$\rho c$に対する連続の式を得る.
積分形では
\[
    \pdv{t} \int \rho c \, \dV = - \int \rho c\v \cdot \dS 
\]
であり,任意の体積中のこの成分の変化率は,流体の運動により体積の表面を通って輸送される量に等しい.


拡散がある場合,流体の運動によるフラックス$\rho c \v$の他に,(静止流体中でも存在する)成分の輸送を担う別のフラックスが存在する.
この拡散フラックス密度を$\vec{i}$としよう.
つまり,単位時間に単位面積を通って拡散により運ばれる成分の質量である
\footnote{2成分のフラックス密度の和は,もちろん$\rho\v$でなければならない.
つまり,一方が$\rho c \v +\vec{i}$なら他方は$\rho(1-c)\v -\vec{i}$である.}.

よって,任意の体積中の成分量の変化率は
\[
    \pdv{}{t} \int \rho c \, \dV = - \int \rho c\v \cdot \dS - \int \vec{i} \cdot \dS 
\]
であり,微分形では
\begin{equation}\label{eq58.2:1成分に対する連続の式1}
    \pdv{(\rho c)}{t} + \Div(\rho c\v) = - \Div \vec{i}
\end{equation}
となる.
\eqref{eq58.1:混合流体の連続の式}を用いると,1つの成分に対する連続の式は
\begin{equation}\label{eq58.3:1成分に対する連続の式}
    \rho\frac{Dc}{dt} = \rho \left( \pdv{c}{t} + \v \cdot \Grad c \right) = - \Div \vec{i}
\end{equation}
とも書ける.


もう1つの方程式を導くために,流体の熱力学的量が濃度の関数でもあることに注意して,\S~6,\S~49と同様の議論を行おう.
この場合,混合流体の化学ポテンシャルを$\mu$とすると
\[
    d\varepsilon = T ds + \frac{p}{\rho^2} d\rho + \mu dc
\]
であり,$d\varepsilon$に新たに$\mu dc$を加えなければならない.
よって\S~6の途中式と見比べることにより,$\rho T \dfrac{Ds}{Dt}$となっている部分を
\[
    \rho T \frac{Ds}{Dt} + \rho \mu \frac{Dc}{Dt}
    = \rho T \frac{Ds}{Dt} - \mu \Div \vec{i}
\]
としなければならない(\eqref{eq58.3:1成分に対する連続の式}を用いた).
したがって(49.4)は($-\kappa \Grad T$の代わりに一般的な表式$\vec{q}$を用いて)
\setcounter{equation}{5}%
\begin{align}
    \rho T \left( \pdv{s}{t} + \v \Grad s \right) 
    &= \sigma_{ij}' \pdv{v_i}{x_j} - \Div \vec{q} + \mu \Div \vec{i} \nonumber\\
    &= \sigma_{ij}' \pdv{v_i}{x_j} - \Div (\vec{q} - \mu\vec{i}) - \vec{i} \cdot \Grad \mu
    \label{eq58.6:混合流体でのエントロピー変化}
\end{align}
となる.


こうして我々は,混合流体の完全な方程式系
(連続の式\eqref{eq58.1:混合流体の連続の式},Navier-Stokes方程式,
ある成分の連続の式\eqref{eq58.2:1成分に対する連続の式1},エントロピー変化の式\eqref{eq58.6:混合流体でのエントロピー変化})を得た.
未知数として$c$が加わったため,1成分の流体の方程式よりも多くなっている.


ここで,\eqref{eq58.2:1成分に対する連続の式1}\eqref{eq58.6:混合流体でのエントロピー変化}はそのままでは,
単に方程式の形を決めているに過ぎないことに注意せよ.
なぜなら,$\vec{i},\vec{q}$の具体的な形が決まっていないからである.
\S~\ref{sec:59}で,$\vec{i},\vec{q}$を濃度/温度の勾配で表すことを学ぶ.



流体の全エントロピーの変化は,(49.4)の代わりに\eqref{eq58.6:混合流体でのエントロピー変化}を用いることにより
\begin{equation}\label{eq58.7:混合流体でのエントロピー変化(積分形)}
    \pdv{t} \int \rho s \,\dV = \int \frac{\Phi}{T} dV
    - \int \frac{(\vec{q}-\mu\vec{i})\cdot\Grad T}{T^2} dV - \int \frac{\vec{i}\cdot\Grad\mu}{T} dV 
\end{equation}
となる($\Phi$は粘性の散逸関数である).











\BackToTheToc