\section{相似則}

粘性流体の運動を論じる際,物理量の次元解析により数々の重要な結果を得ることができる.
ここでは,ある形の物体まわりの流れなど,特定の運動を考える.
もし物体が球でなければ,流れの方向も指定する.
あるいは,与えられた断面内の流れなど,境界で区切られた流れを考えてもよい.


このような場合,同じ形状の物体は\emph{幾何学的に相似}であり,一方から他方に移るには,すべての長さの次元を同じ比だけ変えてやればよい.
したがって,物体の形状が与えられているなら,その大きさを完全に知るためには,長さの次元(球や円筒の半径など)を1つ指定すれば十分である.


問題を明確にするために,物体まわりの流れを論ずることにし,まずは定常流を考える.
すると主流の速度は一定である($u$とする).

流体そのものの性質を表すパラメータとしては,動粘性率$\nu=\eta/\rho$のみが流体力学の方程式の中に現れる(非圧縮性流体を仮定する).
方程式の未知数としては,速度$\v$と,密度に対する圧力の比$p/\rho$があるが,これらは境界条件を通じて物体の形や大きさ,速度に依存する.
物体の形は与えられているので,物体の幾何学的性質は長さの次元$l$により決まる.
このとき,どのような流れも,3つのパラメータ$\nu,u,l$で指定される.これらの次元は(SI単位系で)
\[
    \nu : \si{m^2/s} , \quad
    l : \si{m} , \quad
    u : \si{m/s}
\]
である.これらの量から無次元量を作ってみよう.
\[
    \nu^\alpha l^\beta u^\gamma : \mathrm{m}^{2\alpha+\beta+\gamma} \, \mathrm{s}^{-\alpha-\gamma}
\]
より$2\alpha+\beta+\gamma=0, \alpha+\gamma=0$であり,$\gamma=1$とすると$\alpha=-1, \beta=1$となる.
つまり,3つのパラメータから作られる無次元量は\emph{Reynolds数}
\begin{equation}\label{eq19.1:Reynolds数}
    \R \equiv \frac{ul}{\nu} = \frac{\rho ul}{\eta}
\end{equation}
のみであり,他の無次元量は$\R$だけの関数となる.
Reynolds数は,粘性項$\sim \dfrac{\nu u}{l^2}$に対する移流項$\sim \dfrac{u^2}{l}$の比とみなすことができる.



さて,長さは$l$を,速さは$u$を単位として測ることにしよう.
つまり,無次元の変数$\dfrac{\vec{r}}{l}, \dfrac{\vec{v}}{u}$を導入する.
無次元のパラメータはReynolds数のみであるから,非圧縮性(粘性)流体の運動方程式を解いて得られる速度は
\begin{equation}\label{eq19.2:無次元化した速度}
    \v = u\, \vec{f} \pqty{ \frac{\vec{r}}{l}, \R }
\end{equation}
と書くことができる.
\eqref{eq19.2:無次元化した速度}から,同じ種類の2つの流れ(例えば,異なる半径の球のまわりの,粘性の異なる流れ)において,
Reynolds数が双方で等しければ,速度$\dfrac{\vec{v}}{u}$は$\dfrac{\vec{r}}{l}$の同じ関数となる.
長さや速度の単位を変えることにより一方から他方の流れが得られる場合,2つの流れは\emph{相似}であるという.
つまり,Reynolds数が等しい同じ種類の流れは相似である.
これは\emph{相似則}と呼ばれる(O. Reynolds 1883).


\eqref{eq19.2:無次元化した速度}と同様の式は,圧力に関しても得ることができる.
そのためには,$\nu,u,l$から$\text{(圧力)}/\text{(密度)}=\text{(速度)}^2$の次元を作る必要があり,それは例えば$u^2$である.
よって,無次元化した圧力$\dfrac{p}{\rho u^2}$は,無次元化座標$\dfrac{\vec{r}}{l}$と無次元パラメータ$\R$の関数である.
\begin{equation}\label{eq19.3:無次元化した圧力}
    p = \rho u^2 \, f\pqty{ \frac{\vec{r}}{l}, \R }
\end{equation}


次に,流れを特徴づける量のうち,座標に依存しないもの,例えば物体に働く抗力$F$について,同様の議論を行おう.
$\nu,u,l,\rho$($F$は単位質量あたりではない,普通の力であるから,$\rho$が必要である)から作られる,力の次元をもつ量は,例えば$\rho u^2l^2$である.
よって
\begin{equation}\label{eq19.4:無次元化した力}
    F = \rho u^2 l^2 \, f(\R) .
\end{equation}



続いて,流れにおいて重力が重要な役割を果たす場合を考えよう.
この場合,流れは4つのパラメータ$\nu,u,l,g$(重力加速度)によって指定される.
ここから2つの独立な無次元量を作ることができる.
それは例えば,Reynolds数と\emph{Froude数}
\begin{equation}
    \mathrm{F} = \frac{u^2}{gl}
\end{equation}
である.
%
%
\begin{details}
文献によっては,$\dfrac{u}{\sqrt{gl}}$をFroude数として定義しているものもある.
Landauの定義は,重力項$g$に対する移流項$\sim \dfrac{u^2}{l}$の比とみなすことができる.
また$\dfrac{u}{\sqrt{gl}}$は,表面と伝わる(微小振幅の)長波の速度に対する主流の速度の比とみなすことができる.
\end{details}
\noindent
この場合,\eqref{eq19.2:無次元化した速度}〜\eqref{eq19.4:無次元化した力}の関数$f$は,2つのパラメータ$\R, \mathrm{F}$に依存し,
両者が同じ値をもつ流れは相似である.



最後に,非定常流について触れておこう.
非定常流は,$\nu,u,l$だけでなく,流れの代表的な時間スケール$\tau$によっても特徴付けられる.
例えばある形の物体が流体中で振動しているなら,$\tau$は振動の周期である.
$\nu,u,l,\tau$から,2つの独立な無次元量を作ることができる.
それは例えば,Reynolds数と\emph{Strouhal数}
\begin{equation}
    \mathrm{S} = \frac{u\tau}{l}
\end{equation}
である(逆数$\dfrac{l}{u\tau}$で定義する文献もある).
これは,慣性項$\sim \dfrac{u}{\tau}$に対する移流項$\sim \dfrac{u^2}{l}$の比とみなすことができる.
この場合,パラメータ$\R,\mathrm{S}$が同じ値をとるとき,流れは相似である.

なお,振動が(外力の作用によらず)自発的に起こった場合,$\mathrm{S}$は運動の種類によって決まる$\R$の関数となる:
\[
    \mathrm{S} = f(\R) .
\]



\BackToTheToc