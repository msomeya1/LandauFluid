\section{粘性流体の運動方程式}

流体の運動中に生じるエネルギーの散逸が,流体の運動にどのような影響を及ぼすか考えよう.
エネルギーの散逸過程は,熱力学的に不可逆な運動;内部摩擦(粘性)や熱伝導によって生じる.



粘性流体の方程式系を導きたい.
導き方から明らかなように,連続の式はどのような流体に対しても成立する.
一方で,運動方程式とエネルギー方程式
\footnote{エネルギー方程式については\S~49で議論する.}
は書き改めなければならない.


\subsection*{粘性応力テンソル}

Euler方程式は,運動量フラックス密度テンソルを$\varPi_{ij}$として
\[
    \pdv{t} (\rho v_i) = - \pdv{\varPi_{ij}}{x_j}
\]
と書ける.(7.2)の運動量フラックスは,流体粒子の運動と圧力による,運動量の可逆な輸送を表している.
粘性流体の運動方程式を得るためには,(7.2)に不可逆な運動量の輸送を表す項$-\sigma'_{ij}$を加えればよい.
粘性流体の運動量フラックス密度テンソルは
\begin{equation}
    \varPi_{ij} = p \delta_{ij} + \rho v_iv_j -\sigma'_{ij}
    = -\sigma_{ij} + \rho v_iv_j
\end{equation}
と表される.
テンソル
\begin{equation}
    \sigma_{ij} = -p \delta_{ij} + \sigma'_{ij}
\end{equation}
は\emph{応力テンソル},$\sigma'_{ij}$は\emph{粘性応力テンソル}と呼ばれる.
$\sigma_{ij}$は,運動する流体によって直接運ばれる運動量($\rho v_iv_j$)以外の運動量フラックスを表している.



$\sigma'_{ij}$の一般的な形を導こう.
内部摩擦は,流体粒子の速度が異なるために,流体の各部分の間に相対運動が起こる場合にのみ生じる.
つまり$\sigma'_{ij}$は速度の空間微分の関数でなければならない.
ここでは,速度勾配が小さく,$\sigma'_{ij}$が速度の1階微分の線型結合で表される流体(ニュートン流体)を考えることにしよう.
この条件を満たすテンソルの一般的な形は
\[
    \sigma'_{ij} = a \pdv{v_i}{x_j} + a' \pdv{v_j}{x_i} + b \pdv{v_k}{x_k} \delta_{ij}
    \mytag{1}
\]
である(定数項は,$-p \delta_{ij}$として既に応力テンソルに組み込まれている).



流体が一様な角速度で回転している場合,内部摩擦は生じず$\sigma'_{ij}=0$でなければならない.
このことから,係数についての制約条件を得ることができる.
角速度ベクトルを$\vec{\Omega}$とすると,速度は$\v = \vec{\Omega}\times\vec{r}$または
$v_i = \varepsilon_{ilm}\Omega_lx_m$となる($\varepsilon_{ilm}$はEddingtonのイプシロン).
これを\ajMaru{1}に代入し
\begin{align*}
    \sigma'_{ij} &= a \pdv{x_j}(\varepsilon_{ilm}\Omega_lx_m) + a' \pdv{x_i}(\varepsilon_{jlm}\Omega_lx_m) + b \pdv{x_k}(\varepsilon_{klm}\Omega_lx_m) \delta_{ij} \\
    &= a \varepsilon_{ilm}\Omega_l\delta_{jm} + a' \varepsilon_{jlm}\Omega_l\delta_{im} + b \varepsilon_{klm}\Omega_l\delta_{km}\delta_{ij} \\
    &= \pqty{ a \varepsilon_{ilj} + a' \varepsilon_{jli} + b \varepsilon_{klk}\delta_{ij} } \Omega_l .
\end{align*}
Eddingtonのイプシロンは添字を1組交換すると符号が反転するから$\varepsilon_{jli} = -\varepsilon_{ilj}$であり,
添字に重複があると0になるから$\varepsilon_{klk}=0$である.よって
\[
    \sigma'_{ij} = (a-a') \varepsilon_{ilj} \Omega_l = 0
    \qquad\yueni a = a' .
\]

以上より,$\sigma'_{ij}$の一般的な形は
\[
    \sigma'_{ij} = a \pqty{ \pdv{v_i}{x_j} + \pdv{v_j}{x_i} } + b \pdv{v_k}{x_k} \delta_{ij}
\]
となる.あるいは,$a,b$を他の定数で表した次の形が便利である.
\begin{equation}\label{eq15.3:粘性応力テンソルの具体形}
    \sigma'_{ij} = \eta \pqty{ \pdv{v_i}{x_j} + \pdv{v_j}{x_i} - \frac{2}{3}\pdv{v_k}{x_k} \delta_{ij} } + \zeta \pdv{v_k}{x_k} \delta_{ij}
\end{equation}
$(\,)$内を$i,j$について縮約すると0になる.
$\eta,\zeta$は速度に無関係の定数で,$\eta$は\emph{粘性率},$\zeta$は\emph{第2粘性率}と呼ばれる.
\S~\ref{sec:16},\S~49で見るように,これらは正である.
\begin{equation}
    \eta > 0, \quad \zeta > 0.
\end{equation}



\subsection*{粘性流体の運動方程式}
さて,粘性流体の運動方程式を得よう.
$\varPi_{ij}$に$-\sigma'_{ij}$を加えたことに対応して,Euler方程式に$\dpdv{\sigma'_{ij}}{x_j}$を加えればよいことがわかる.
よって
\begin{align}
    \rho \pqty{ \pdv{v_i}{t} + v_j \pdv{v_i}{x_j} } &= - \pdv{p}{x_i} + \pdv{\sigma'_{ij}}{x_j} \notag \\
    &= - \pdv{p}{x_i} + \pdv{x_j} \bqty{ \eta \pqty{ \pdv{v_i}{x_j} + \pdv{v_j}{x_i} - \frac{2}{3}\pdv{v_k}{x_k} \delta_{ij} } } + \pdv{x_i} \pqty{ \zeta \pdv{v_k}{x_k} }.
\end{align}
これは粘性流体の運動方程式の最も一般的な形である.
$\eta,\zeta$は一般に温度と圧力の関数であり,流体中で一様とは限らないから,微分の外に出すことはできない.
しかし多くの場合には,$\eta,\zeta$は定数とみなしてよい.その場合には
\begin{align*}
    \pdv{\sigma'_{ij}}{x_j} &= \eta \bqty{ \pdv{v_i}{x_j}{x_j} + \pdv{x_i} \pqty{\pdv{v_j}{x_j}} - \frac{2}{3}\pdv{x_i} \pqty{\pdv{v_k}{x_k}} } + \zeta \pdv{x_i} \pqty{\pdv{v_k}{x_k}} \\
    &= \eta \Laplacian v_i + \pqty{\zeta+\frac{1}{3}\eta} \pdv{x_i} (\Div\v) .
\end{align*}
よって粘性流体の運動方程式をベクトル形式で表すと
\begin{equation}
    \rho \bqty{ \pdv{\v}{t} + (\v\cdot\Grad)\v } = -\Grad p + \eta\Laplacian\v + \pqty{\zeta+\frac{1}{3}\eta} \Grad \Div\v .
\end{equation}
この式は\emph{Navier-Stokes方程式}と呼ばれている.


粘性流体を論じる際,ほとんどの場合非圧縮であると仮定してよい.
その場合$\Div\v=0$であるから
\begin{equation}\label{eq15.7:非圧縮性流体のNS方程式}
    \pdv{\v}{t} + (\v\cdot\Grad)\v  = - \frac{1}{\rho}\Grad p + \frac{\eta}{\rho} \Laplacian\v 
\end{equation}
を得る.
\eqref{eq15.3:粘性応力テンソルの具体形}で$\Div\v=0$とおくことにより,非圧縮性流体での応力テンソルは
\begin{equation}\label{eq15.8:非圧縮性流体での応力テンソル}
    \sigma_{ij} = -p \delta_{ij} + \eta \pqty{ \pdv{v_i}{x_j} + \pdv{v_j}{x_i} }
\end{equation}
と簡単な形に書ける.


非圧縮性流体では,粘性はただ1つの係数$\eta$で表される.
このため,粘性率といえばふつう$\eta$のことを指す.
また,
\begin{equation}
    \nu = \frac{\eta}{\rho}
\end{equation}
は\emph{動粘性率}と呼ばれる.
ある温度での気体の粘性率$\eta$は圧力によらない
\footnote{これは気体分子運動論の結果である.『物理的運動学』\S~8参照.}
が,動粘性率$\nu$は圧力に反比例する.




Euler方程式で行ったのと同様にして,Navier-Stokes方程式から圧力の項を除くことができる.
\eqref{eq15.7:非圧縮性流体のNS方程式}の両辺のrotをとれば,理想流体での(2.11)に対応して
\[
    \pdv{t}(\Rot\v) = \Rot(\v \times \Rot\v) + \nu\Laplacian(\Rot\v) .
\]
流体が非圧縮であるから,右辺第1項を展開して$\Div\v=0$を用いると
\[
    \Rot(\v \times \Rot\v)
    = (\Rot\v \cdot \Grad)\v - (\v\cdot\Grad)(\Rot\v) + \cancel{(\Div\Rot\v)} \v - \cancel{(\Div\v)} (\Rot\v) ,
\]
\begin{equation}
    \yueni \pdv{t}(\Rot\v) + (\v\cdot\Grad)(\Rot\v) - (\Rot\v \cdot \Grad)\v = \nu\Laplacian(\Rot\v) .
\end{equation}

\spade
\eqref{eq15.7:非圧縮性流体のNS方程式}の両辺のdivをとると
\[
    \Div[ (\v\cdot\Grad)\v ] = -\frac{1}{\rho} \Laplacian p
\]
\[
    \yueni \Laplacian p = -\rho \pdv{x_i} \pqty{ v_j \pdv{v_i}{x_j} }
    = -\rho \bqty{ \pdv{v_j}{x_i} \pdv{v_i}{x_j} + v_j \pdv{x_j} \cancel{(\Div\v)} }
\]
また
$ \dpdv{(v_iv_j)}{x_i}{x_j} = \dpdv{x_i} \pqty{ \dpdv{v_i}{x_j} v_j + v_i \cancel{\dpdv{v_j}{x_j}} } $
でもあるから,結局
\begin{equation}
    \Laplacian p = -\rho \pdv{v_i}{x_j} \pdv{v_j}{x_i} = -\rho \pdv{(v_iv_j)}{x_i}{x_j}
\end{equation}
この式は,速度分布が分かっているときに圧力を求めるためのPoisson方程式である.



非圧縮性粘性流体の2次元流で,流線関数$\psi(x,y)$が満たすべき方程式を書き下そう.
(10.10)の導き方から分かるように,(10.10)の左辺に$\nu\Laplacian(\Rot\v)=\nu\Laplacian(-\Laplacian\psi)$を加え
\begin{equation}
    \pdv{t}(\Laplacian\psi) - \pdv{\psi}{x} \pdv{y}(\Laplacian\psi) + \pdv{\psi}{y} \pdv{x}(\Laplacian\psi) -\nu\Laplacian(\Laplacian\psi) = 0 .
\end{equation}


\subsection*{粘性流体の運動方程式の境界条件}

さて,粘性流体の運動方程式の境界条件を考えよう.
粘性流体と固体表面の間には常に分子間引力が働いており,表面に接する流体の層は表面に密着して静止状態にある.
よって境界条件は,固定された物体の表面で流体の速度が0となることである.
\begin{equation}
    \v = \vec{0}
\end{equation}
理想流体では法線速度$v_n$が0となることが条件であったが,粘性流体では,法線速度と接線速度の両方が0となることが条件である.
\begin{details}
一般に,Euler方程式に接線速度が0という条件を加えてしまうと,解が存在し得ないことに注意しよう.
これはEuler方程式が1階の偏微分方程式であり,$v_n=0$以外の境界条件を課すことは許されないためである.
\end{details}
\noindent
また,物体表面が動く場合には,$\v$は表面の速度に等しくなければならない.



粘性流体中にある物体表面に働く力を書き下すことは容易である.
表面の面積要素$\dS$に働く力は,この要素を通り抜ける運動量フラックスに等しく
\footnote{これは各要素の静止座標系に乗って考えた場合に限る.}
\[
    \varPi_{ij}dS_j = (\rho v_iv_j - \sigma_{ij}) dS_j .
\]
$\vec{n}$を法線ベクトルとすると$dS_j=n_j dS$,また表面では$\v=\vec{0}$である.
よって表面の単位面積に働く力$\vec{P}$は
\begin{equation}\label{eq15.14:粘性流体中の物体表面に働く力}
    P_i = -\sigma_{ij} n_j = p n_i - \sigma'_{ij} n_j .
\end{equation}
第1項は通常の圧力,第2項は粘性により表面に働く摩擦力である.
なお\eqref{eq15.14:粘性流体中の物体表面に働く力}の$\vec{n}$は流体から見て外向きを正とするベクトルであるから,
固体から見て内向きが正であることに注意.



固体以外の境界条件に触れておこう.
互いに混じり合わない2つの流体が接している場合,境界面で成り立つ条件は,
流体の速度が等しいこと
\[
    \v^{(1)} = \v^{(2)}
\]
と,もう一方の流体に働く力の大きさが等しく向きが逆となること
\[
    \sigma_{1,ij}n_{1,j} + \sigma_{2,ij}n_{2,j} = 0
\]
である(添字1,2は2つの流体を表す).
$\vec{n}_1,\vec{n}_2$は逆向きであるから,$\vec{n}\equiv\vec{n}_1=-\vec{n}_2$とおくと
\begin{equation}
    \sigma_{1,ij}n_j = \sigma_{2,ij}n_j .
\end{equation}
また,自由表面で満たされるべき条件は
\begin{equation}\label{eq15.16:粘性流体の境界条件(自由表面)}
    \sigma_{ij}n_j = -p n_i + \sigma'_{ij} n_j = 0
\end{equation}
である.


\begin{details}
曲線座標(特に円筒座標,球座標)におけるNavier-Stokes方程式について本文を参照のこと.
ここでは導出を省略し,先へ進むことにする.
\end{details}




\BackToTheToc