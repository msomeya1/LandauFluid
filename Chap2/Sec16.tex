\section{非圧縮性流体でのエネルギー散逸}\label{sec:16}

粘性によりエネルギーは散逸し,最終的には熱に変換される.
非圧縮性流体でのエネルギー散逸は,簡単に計算することができる.



非圧縮性流体の全運動エネルギー
\[
    E_\mathrm{kin} = \frac{1}{2} \rho \int v^2 \dV
\]
の時間微分を計算しよう.
Navier-Stokes方程式より
\begin{align*}
    \pdv{t} \pqty{\frac{1}{2}\rho v^2} &= \rho v_i \pdv{v_i}{t} \\
    &= v_i \bqty{ -\rho v_j\pdv{v_i}{x_j} - \rho\pdv{x_i} \pqty{\frac{p}{\rho}} + \pdv{\sigma'_{ij}}{x_j} } \\
    &= -\rho v_j \pdv{x_j} \pqty{\frac{1}{2}v_iv_i} - \rho v_i \pdv{x_i} \pqty{\frac{p}{\rho}} + v_i \pdv{\sigma'_{ij}}{x_j}
\end{align*}
ベクトル解析の公式より
\begin{align*}
    \rho v_j \pdv{x_j} \pqty{\frac{1}{2}v_iv_i} + \rho v_i \pdv{x_i} \pqty{\frac{p}{\rho}}
    &= \rho\v \cdot \Grad \pqty{ \frac{v^2}{2} + \frac{p}{\rho} } \\
    &= \Div \bqty{ \rho\v \pqty{ \frac{v^2}{2} + \frac{p}{\rho} } } - \rho \pqty{ \frac{v^2}{2} + \frac{p}{\rho} } \cancel{\Div\v}
\end{align*}
である.
また,第$j$成分が$v_i\sigma'_{ij}$であるようなベクトルを$\v\cdot\mat{\sigma'}$と書くことにすると
\[
    v_i \pdv{\sigma'_{ij}}{x_j} = \pdv{x_j} (v_i \sigma'_{ij}) - \sigma'_{ij} \pdv{v_i}{x_j}
    = \Div(\v\cdot\mat{\sigma'}) - \sigma'_{ij} \pdv{v_i}{x_j} .
\]
以上より
\begin{equation}\label{eq16.1:非圧縮性粘性流体での運動エネルギー保存則(微分形)}
    \pdv{t} \pqty{\frac{1}{2}\rho v^2} 
    = -\Div\bqty{ \rho\v \pqty{ \frac{v^2}{2} + \frac{p}{\rho} } - \v\cdot\mat{\sigma'} } - \sigma'_{ij} \pdv{v_i}{x_j} .
\end{equation}


\eqref{eq16.1:非圧縮性粘性流体での運動エネルギー保存則(微分形)}をある体積$V$で積分してガウスの発散定理を用いると
\begin{equation}\label{eq16.2:非圧縮性粘性流体での運動エネルギー保存則(積分形)}
    \pdv{t} \int_V \frac{1}{2}\rho v^2 \dV
    = -\int_S \bqty{ \rho\v \pqty{ \frac{v^2}{2} + \frac{p}{\rho} } - \v\cdot\mat{\sigma'} } \cdot \dS - \int_V \sigma'_{ij} \pdv{v_i}{x_j} \dV .
\end{equation}
$\rho\v \pqty{ \dfrac{v^2}{2} + \dfrac{p}{\rho} }$は流体の運動によるエネルギーフラクス,
$-\v\cdot\mat{\sigma'}$は内部摩擦によるエネルギーフラックスである.
よって\eqref{eq16.2:非圧縮性粘性流体での運動エネルギー保存則(積分形)}右辺の面積分は,
$V$の表面を通るエネルギーフラックスによる$E_\mathrm{kin}$の変化率を表す.
また第2項は,粘性散逸による$E_\mathrm{kin}$の単位時間あたりの減少率を表す.



積分を全体積にわたって行おう.
流体の体積が無限大なら,無限遠での速度は0であるから,面積分は0になる.
また,流体の体積が有限であっても,表面での法線速度は0であるから,やはり面積分は0になる.
よって
\[
    \dot{E}_\mathrm{kin} = - \int \sigma'_{ij} \pdv{v_i}{x_j} \dV
\]
となる.$\sigma'_{ij}$が対称であることを用い,\eqref{eq15.8:非圧縮性流体での応力テンソル}を代入すると
\[
    \sigma'_{ij} \pdv{v_i}{x_j} = \frac{1}{2} \sigma'_{ij} \pqty{ \pdv{v_i}{x_j} + \pdv{v_j}{x_i} }
    = \frac{1}{2} \eta \pqty{ \pdv{v_i}{x_j} + \pdv{v_j}{x_i} }^2 .
\]
よって非圧縮性流体のエネルギー散逸は
\begin{equation}\label{eq16.3:非圧縮性流体のエネルギー散逸}
    \dot{E}_\mathrm{kin} = -\frac{1}{2} \eta \int \pqty{ \pdv{v_i}{x_j} + \pdv{v_j}{x_i} }^2 \dV
\end{equation}
となる.
エネルギーが散逸することで力学的エネルギーは減少するはずであるから,\eqref{eq16.3:非圧縮性流体のエネルギー散逸}は負でなければならない.
よって
\[
    \eta > 0 ,
\]
つまり粘性率は正である.

%%%%%%%%%% 問題1 %%%%%%%%%%

\begin{mondai}{}{}
ポテンシャル流に対して,\eqref{eq16.3:非圧縮性流体のエネルギー散逸}を面積分に書き換えよ.
\end{mondai}
\begin{kaitou}
$v_i = \dpdv{\phi}{x_i}$より
$\dpdv{v_i}{x_j} = \dpdv{\phi}{x_i}{x_j} = \dpdv{v_j}{x_i}$
であるから
\[
    \pqty{ \pdv{v_i}{x_j} + \pdv{v_j}{x_i} }^2 = 4 \pdv{v_i}{x_j} \pdv{v_i}{x_j}
    = 4 \bqty{ \pdv{x_j}\pqty{v_i\pdv{v_i}{x_j}} - v_i \pdv[2]{v_i}{{x_j}} }
\]
$\Laplacian\v = \Grad(\Laplacian\phi) = 0$より最右辺第2項は消える.
結局
\begin{align*}
    \dot{E}_\mathrm{kin} &= -2\eta \int \pdv{x_j}\pqty{v_i\pdv{v_i}{x_j}} \dV 
    = -2\eta \int v_i\pdv{v_i}{x_j} \, dS_j \\
    &= -\eta \int \pdv{x_j}(v_iv_i) \, dS_j = -\eta \int \Grad v^2 \cdot \dS .\\
\end{align*}
    
\end{kaitou}

\BackToTheToc