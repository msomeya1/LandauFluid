\section{層流タイプの伴流}

\begin{details}
Landauでは,まず伴流の定性的特徴について述べ,次に定量的議論を行なっている.
しかし,ここでは前半の定性的議論のみ取り上げる.
\end{details}


物体のまわりの定常粘性流では,物体の後方に十分離れたところで,流れがある特性を持つ.
これは,物体の形によらず議論することができる.
物体の前方から,一定速度$\vec{U}$で流れが入射するとし,$\vec{U}$の方向に$x$軸をとる(原点は物体内部にとる).
実際の流体の速度は$\vec{U}+\v$と表せるとする($\v$は無限遠で0になる).


物体の後方に十分離れ,かつ$x$軸に近い領域では,$\v$は0でないことが知られており,
\emph{層流タイプの伴流}
\footnote{乱流タイプの伴流と区別してこの語を用いる.なお,伴流のことを\emph{後流}と呼ぶこともある.}
と呼ぶ.
層流タイプの伴流は,物体に近い流線に沿って動いた流体粒子により構成される.
伴流は本質的に回転的である.
%
\begin{details}
\spade
その理由は,物体まわりの粘性流の回転的な部分は,物体表面によるものだからである:
理想流体のポテンシャル流では,速度の接線成分$v_t$は表面で0でなくてもよいが,粘性流体では表面で$v_t=0$でなければならない.
外側でポテンシャル流のパターンが維持されるなら,表面で$v_t$の不連続性,すなわち表面渦度が発生する.

粘性によってこの不連続性は平滑化され,回転状態は流体中に浸透し,そこから移流によって伴流領域へ移動する.
\end{details}
%
\noindent%
一方,物体の近くを通らない流線上では,粘性の影響はほとんどなく,
(物体前方で渦度が0であったから)これらの流線上では理想流体と同じく渦度を0とみなすことができる.
よって,伴流の部分を除けば,流れはポテンシャル流である.




さて,伴流の特性と,物体に働く力を求めよう.
物体を取り囲む任意の(無限に大きい)閉曲面を通って輸送される全運動量は$\displaystyle\int\varPi_{ij} dS_j$である.
無限遠での圧力を$p_0$として$p=p_0+p'$と表すと
\[
    \varPi_{ij} = (p_0+p')\delta_{ij} + \rho(U_i+v_i)(U_j+v_j).
\]
定テンソルを閉曲面上で面積分すると0になるから,$p_0\delta_{ij} + \rho U_iU_j$の積分は0である.
次に,考えている領域内の質量は一定であるから(非圧縮性を仮定する),全質量フラックス$\displaystyle \int \rho \v \cdot \dS$は0である.
つまり$\displaystyle U_i \int \rho v_j \; dS_j=0$である.
さらに,物体から離れたところでは$\v \ll \vec{U}$であるから,無限に大きい閉曲面上で積分するとき,
$\rho U_jv_i$に比べて$\rho v_iv_j$を無視することができる.

以上より,この閉曲面を通る全運動量は
\[
    \int (p' \delta_{ij} + \rho U_j v_i) dS_j
\]
となる.




次に,物体後方の無限平面$x=x_1$と,物体前方の無限平面$x=x_2$の間の流体を考えよう.
この2面以外の境界は無限遠($y,z \to \pm \infty$)にあり,そこでは$p', \v' \to 0$であるから,$x=x_1, x_2$での積分を考えればよい.
よってこの領域に入ってくる全運動量は,$x=x_2$に入ってくる運動量フラックスと,$x=x_1$から出ていく運動量フラックスの差である.
この差は,単位時間に流体から物体に与えられる運動量,つまり物体に働く力に他ならない.
よって力$\vec{F}$の成分は
\[
    F_x = \left( \iint_{x_2} - \iint_{x_1} \right) (p' + \rho U v_x) \; dydz,
\]
\[
    F_y = \left( \iint_{x_2} - \iint_{x_1} \right) \rho U v_y \; dydz,
\]
\[
    F_z = \left( \iint_{x_2} - \iint_{x_1} \right) \rho U v_z\; dydz
\]
となる.

まず$F_x$について考えてみよう.
物体の前方では伴流がないから,$x=x_2$での積分は0である.
伴流の外側では,流れはポテンシャル流とみなせるから,ベルヌーイの方程式
\[
    p_0 + p' + \frac{1}{2}\rho |\vec{U}+\v|^2 = \const = p_0 + \frac{1}{2}\rho U^2
\]
が成り立つ.
$\rho \vec{U}\cdot\v$に比べて$\rho v^2/2$を無視すると
\[
    p' = -\rho \vec{U}\cdot\v = -\rho U v_x.
\]
よってこの近似のもとでは,$F_x$の被積分関数は伴流の外側で0となる.
したがって,$x=x_1$の,伴流が存在する領域で積分を行えばよい.
また伴流の内部では,$p'$は$\rho v^2$のオーダーであり,$\rho U v_x$に比べて小さい.以上より
\begin{equation}\label{eq21.1:伴流でのFx}
    F_x = -\rho U \iint_{x_1 | \mathrm{wake}} v_x \; dydz
\end{equation}
となる(積分は,物体から十分離れた,伴流の断面で行う).
物体がないときに比べれば,伴流内の速度は遅くなっているから,$v_x<0,\;F_x>0$である.
% また,\eqref{eq21.1:伴流でのFx}に$\D t$をかけることにより,積分は体積の次元を持つ.
% よって\eqref{eq21.1:伴流でのFx}は,物体が存在することによる,伴流領域での流量の減少率を表している.




次に,物体を流れに垂直な方向へ動かす力$(F_y, F_z)$,つまり\emph{揚力}について考えよう.
伴流領域の外側では流れはポテンシャル流であるから,$v_y = \dpdv{\Phi}{y}, v_z = \dpdv{\Phi}{z}$と書くことができて
\[
    F_y = \left( \iint_{x_2} - \iint_{x_1} \right) \rho U \pdv{\Phi}{y} \; dydz
    = \int dz \rho U \cdot \eval{\Phi}_{x=x_2} - \iint_{x_1} \rho U v_y \; dydz .
\]
伴流のない前方では$\Phi\to0$であるから第1項は消える.
また$x=x_1$のうち伴流の外側であればやはり$\Phi\to0$となる.よって
\begin{equation}
    \begin{cases}
        \displaystyle F_y = - \rho U \iint_{x_1 | \mathrm{wake}} v_y \; dydz \\[10pt]
        \displaystyle F_z = - \rho U \iint_{x_1 | \mathrm{wake}} v_z \; dydz \\[10pt]
    \end{cases} .
\end{equation}
ただし積分は,物体から十分離れた伴流領域内で行う.
物体が対称軸を持ち,かつ流れが軸に平行なら,物体まわりの流れも軸対称である($x=x_1$で$v_y, v_z \simeq 0$).
もちろん揚力は0である.



最後に,伴流領域内部の流れについて考えよう.
Navier-Stokes方程式で各項の大きさを見積もると,物体からの距離が$r \gg \nu/U$となるところでは$\nu\Laplacian\v$を無視することができる.
しかし伴流内部では,これは成立しない;$\dpdv[2]{\v}{x}$に比べ$\dpdv[2]{\v}{y},\dpdv[2]{\v}{z}$が大きくなるからである.
移流項$(\v \cdot \Grad)\v$とこれらが同程度であるという条件から,伴流の幅を見積もることができる.
伴流の幅を$Y$,つまり$x$軸から$Y$程度離れると,速度が急に変化するとしよう.
\[
    (\v \cdot \Grad)\v \simeq (U+v)\pdv{v}{x} \sim \frac{Uv}{x}, \quad
    \nu\Laplacian\v \sim \frac{\nu v}{Y^2}
\]
が同程度であるから
\begin{equation}\label{eq21.3:伴流の幅の見積り}
    Y \sim \sqrt{\frac{\nu x}{U}}
\end{equation}
となる.つまり,層流タイプの伴流の幅は,物体からの距離の平方根に比例して大きくなる.
しかし$Y$は,$x$に比べれば小さい.というのは$\dfrac{Y}{x} \sim \sqrt{\dfrac{\nu}{Ux}} \ll 1$だからである.

伴流領域内で,$x$が増加するにつれて速度がどのように減少するかを,\eqref{eq21.1:伴流でのFx}から見積もることができる.
積分領域の大きさは$Y^2$であるから,\eqref{eq21.3:伴流の幅の見積り}も使えば
\[
    F_x \sim \rho U v Y^2 \sim \eta vx
\]
\begin{equation}
    \yueni v \sim \frac{F_x}{\eta x}.
\end{equation}


\BackToTheToc