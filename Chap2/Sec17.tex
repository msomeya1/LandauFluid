\section{管を通る流れ}

ここでは,非圧縮性粘性流体の簡単な定常流を扱おう.


\subsection*{Couette流}
流体が2つの板に挟まれていて,上の板は下の板に対して一定の相対速度$\vec{u}$で動いているとする.
$x$軸を$\vec{u}$の方向にとり,下の板を$xz$平面($y=0$),上の板を$y=h$とする.
このとき,流れは$x$方向にのみ存在し,全ての量は$y$のみに依存することは明らかである.
定常流であるから,\eqref{eq15.7:非圧縮性流体のNS方程式}より
\[
    \begin{cases}
        0 = \ddv[2]{v}{y} \\[7pt]
        0 = \ddv{p}{y}
    \end{cases}
\]
となる($\Div\v=0$は自動的に満たされている).
第2式より$p=\const$であり,第1式より$v=ay+b$と書ける.
境界条件は$y=0$で$v=0$,$y=h$で$v=u$であるから
\begin{equation}\label{eq17.1:平行平板間のCouette流}
    v(y) = u\frac{y}{h}
\end{equation}
となり,流速分布は線形である.平均流速は
\begin{equation}
    \overline{v} = \frac{1}{h} \int_0^h v(y)\; dy = \frac{u}{h^2} \frac{h^2}{2} = \frac{1}{2}u .
\end{equation}
粘性応力テンソル$\sigma_{ij}'$のうち0でないものは
\[
    \sigma_{xy}' = \eta \dv{v}{y} = \eta\frac{u}{h}
\]
である.よって\eqref{eq15.14:粘性流体中の物体表面に働く力}より,板に働く力の
法線成分($y$成分)は一定値$p$であり,
接線成分($x$成分)は
\begin{equation}
    y=0 \;\text{で}\; \sigma_{xy} = \sigma_{xy}' = \eta\frac{u}{h},
\end{equation}
\[
    y=h \;\text{で}\; -\eta\frac{u}{h}
\]
となる.

\begin{details}
\eqref{eq15.14:粘性流体中の物体表面に働く力}によれば,固体が流体から受ける力の$i$成分は,
$\vec{n}$を\underline{固体表面から外へ向く}法線ベクトルとして$\sigma_{ij}n_j$と表され,今の場合
\[
    \sigma_{ij} = -p \delta_{ij} + \sigma_{ij}'
    = \pmqty{
        -p & \eta u/h & 0 \\
        \eta u/h & -p & 0 \\
        0 & 0 & -p 
    }
\]
である.
$y=0$の板にとっては$\vec{n}=\pmqty{0 \\ 1 \\ 0}$であるから,力の$x$成分は$\eta u/h$,$y$成分は$-p$であり,
$y=h$の板にとっては$\vec{n}=\pmqty{0 \\ -1 \\ 0}$であるから,力の$x$成分は$-\eta u/h$,$y$成分は$p$である.
\end{details}


\subsection*{圧力勾配により駆動される流れ}

次に,2つの固定された板に挟まれた流体が,圧力勾配によって流れている場合を考えよう.

座標系は前と同様とし,流れは$x$方向とする.
Navier-Stokes方程式は
\[
    \begin{cases}
        0 = - \dpdv{p}{x} + \eta \dpdv[2]{v}{y} \\[7pt]
        0 = \dpdv{p}{y}
    \end{cases} .
\]
第2式より圧力は$y$によらず,$x$のみの関数である(すなわち圧力は上下方向に一様で,流れの方向にのみ変化する).
すると$\dpdv{p}{x}$は$x$のみの関数,$\dpdv[2]{v}{y}$は$y$のみの関数であるから,両者は定数でなければならない.
つまり
\[
    \dv{p}{x} = \const
\]
であり,圧力は$x$に比例する.
管両端の圧力差を$\D p$,管の長さを$l$として$\ddv{p}{x}=-\dfrac{\D p}{l}\;(<0)$と書くことにすると,第1式は
\[
    \dv[2]{v}{y} = -\frac{\D p}{\eta l}.
\]
$y=0,h$で$v=0$の条件から$v(y)$を求めると
\begin{equation}
    v(y) = \frac{\D p}{2\eta l} y(h-y) .
\end{equation}
となり,速度分布は$y=\dfrac{1}{2}h$で極値
\[
    v_* = \frac{\D p h^2}{8\eta l}
\]
をとる放物線型である.
また平均速度は
\begin{equation}\label{eq17.5:圧力勾配により駆動される流れの平均流速}
    \overline{v} = \frac{1}{h} \int_0^h v(y)\; dy
    = \frac{1}{h} \cdot \frac{\D p}{2\eta l} \cdot \frac{h^3}{6} = \frac{\D p h^2}{12\eta l}
\end{equation}
であり,$v_* = \dfrac{3}{2}\overline{v}$となることが分かる.
さらに,下の板に働く摩擦力は
\begin{equation}
    \eval{\sigma_{xy}'}_{y=0} = \eta \eval{\dv{v}{y}}_{y=0} 
    = \frac{\D p}{2l} (h-2y)_{y=0} = \frac{\D p h}{2l} 
\end{equation}
であり,上の板に働く摩擦力も同じである.


\subsection*{Hagen-Poiseuille流}

最後に,任意の形の断面をもつ管の中の定常流を考えよう.但し,断面の形は管に沿って変わらないとする.
管の軸方向に$x$軸をとると,流れは$x$軸に平行で,$y,z$のみの関数である.
このとき連続の式は自動的に満たされる.
また,Navier-Stokes方程式の$y,z$成分$\dpdv{p}{y}=\dpdv{p}{z}=0$より,圧力は管の断面内では一様で,$x$方向のみに変化する.
\eqref{eq15.7:非圧縮性流体のNS方程式}の$x$成分は
\begin{equation}\label{eq17.7:管の中の定常流の方程式}
    \pdv[2]{v}{y} + \pdv[2]{v}{z} = \frac{1}{\eta} \dv{p}{x} = -\frac{\D p}{\eta l}.
\end{equation}
よって流速分布$v(y,z)$は,$\Laplacian v=\const$という2次元Poisson方程式を,管断面の周上で$v=0$という条件で解くことで求められる.




半径$R$の円形断面の管について,\eqref{eq17.7:管の中の定常流の方程式}を解こう.
円の中心を原点とする極座標をとると,対称性から$v=v(r)$であり,極座標でのラプラシアンの表式から
\[
    \frac{1}{r} \dv{r} \pqty{r\dv{v}{r}} = -\frac{\D p}{\eta l}
\]
\[
    \dv{r} \pqty{r\dv{v}{r}} = -\frac{\D p}{\eta l}r
\]
\[
    r\dv{v}{r} = -\frac{\D p}{2\eta l}r^2 + a
\]
\[
    \dv{v}{r} = -\frac{\D p}{2\eta l}r + \frac{a}{r}
\]
\begin{equation}\label{eq17.8:管の中の流速(極座標での一般形)}
    v(r) = -\frac{\D p}{4\eta l}r^2 + a \log r + b.
\end{equation}
管の中心で速度が有限の値をもつためには,$a=0$でなければならない.
また,$r=R$で$v=0$という条件から$b$を決めると
\begin{equation}\label{eq17.9:円形断面の管の中の流速}
    v(r) = \frac{\D p}{4\eta l} (R^2-r^2)
\end{equation}
となり,この場合の流速分布も放物線的である.


管の断面を通って毎秒運ばれる流体の質量$Q$(\emph{流量})を求めよう.
半径$r \sim r+dr$のリングを通る流量は$dQ \simeq \rho \cdot 2\pi rdr \cdot v$で近似できるから
\[
    Q = 2\pi\rho \int_0^R rv  \; dr
\]
となる.\eqref{eq17.9:円形断面の管の中の流速}を代入し
\begin{equation}\label{eq17.10:Hagen-Poiseuilleの法則}
    Q = 2\pi\rho \cdot \frac{\D p}{4\eta l} \int_0^R r(R^2-r^2) \; dr
    = \frac{\pi\D p}{2\nu l} \pqty{\frac{R^4}{2}-\frac{R^4}{4}} = \frac{\pi\D p}{8\nu l} R^4
\end{equation}
を得る.
つまり,流量は半径の4乗に比例する(\emph{Hagen-Poiseuilleの法則}).

なお,$r=0$での速度を$v_* = \dfrac{\D p}{4\eta l}R^2$とおくと,
\[
    \overline{v} = \frac{Q}{\rho\pi R^2} = \frac{\D p}{8\eta l}R^2 = \frac{1}{2} v_*
\]
である.



\BackToTheToc


%%%%%%%%%% 問題1 %%%%%%%%%%

\begin{mondai}{}{}
内径$R_1$,外径$R_2$のドーナツ状の断面をもつ管の中の流れを求めよ.
\end{mondai}
\begin{kaitou}
\eqref{eq17.8:管の中の流速(極座標での一般形)}の定数$a,b$を,境界条件:$r=R_1,R_2$で$v=0$のもとで求める.
但し計算の便宜のため,$a,b$の定義は
\[
    v(r) = -\frac{\D p}{4\eta l}r^2 + a \log\pqty{\frac{r}{R_2}} + b
\]
とする.
\[
    \begin{cases}
        0 = -\dfrac{\D p}{4\eta l} {R_1}^2 + a \log\pqty{\dfrac{R_1}{R_2}} + b \\[7pt]
        0 = -\dfrac{\D p}{4\eta l} {R_2}^2 + b \\
    \end{cases}
\]
第2式より$b = \dfrac{\D p}{4\eta l}{R_2}^2$であり,これを第1式に代入して
\[
    a \log\pqty{\frac{R_1}{R_2}} = \frac{\D p}{4\eta l} ({R_1}^2-{R_2}^2)
    \qquad\yueni a = \frac{\D p}{4\eta l} \cdot \frac{ {R_2}^2-{R_1}^2 }{\log(R_2/R_1)}
\]
となる.よって流速分布は
\[
    v(r) = \frac{\D p}{4\eta l} \bqty{ {R_2}^2 - r^2 + \frac{ {R_2}^2-{R_1}^2 }{\log(R_2/R_1)} \log\pqty{\frac{r}{R_2}} } .
\]


不定積分
\[
    \int r \log\pqty{\frac{r}{R_2}} \; dr = \frac{r^2}{2} \log\pqty{\frac{r}{R_2}} - \frac{r^2}{4}  
\]
に注意すると
\begin{align*}
    Q &= 2\pi\rho \int_{R_1}^{R_2} rv \; dr \\
    &= 2\pi\rho \cdot \frac{\D p}{4\eta l} \bqty{ 
        {R_2}^2\frac{r^2}{2} - \frac{r^4}{4} + \frac{ {R_2}^2-{R_1}^2 }{\log(R_2/R_1)} \Bqty{ \frac{r^2}{2} \log\pqty{\frac{r}{R_2}} - \frac{r^2}{4} }
    }_{R_1}^{R_2} \\
    &= \frac{\pi\D p}{2\nu l} \cdot \frac{1}{4} \bqty{
        2{R_2}^2 ({R_2}^2-{R_1}^2) - ( {R_2}^4-{R_1}^4 ) + 0
        - \frac{ {R_2}^2-{R_1}^2 }{\cancel{\log(R_2/R_1)}} \Bqty{ -2{R_1}^2 \cancel{\log\pqty{\frac{R_2}{R_1}}} }
        - \frac{ {R_2}^2-{R_1}^2 }{\log(R_2/R_1)} ({R_2}^2-{R_1}^2)
    } \\
    &= \frac{\pi\D p}{8\nu l} \bqty{ {R_2}^4-{R_1}^4 - \frac{ ({R_2}^2-{R_1}^2)^2 }{\log(R_2/R_1)} }
\end{align*}


    
\end{kaitou}







%%%%%%%%%% 問題2 %%%%%%%%%%

\begin{mondai}{}{}
楕円形の断面をもつ管の中の流れを求めよ.
\end{mondai}
\begin{kaitou}
楕円の形を$\dfrac{y^2}{a^2}+\dfrac{z^2}{b^2}=1$とする.
円は楕円の特別な場合であるから,流速分布$v(y,z)$は\eqref{eq17.9:円形断面の管の中の流速}から類推して
\[
    v(y,z) = v_0 \pqty{1-\dfrac{y^2}{a^2}-\dfrac{z^2}{b^2}}
\]
と書けるはずである.
境界条件は満たされているから,未定の定数$v_0$を\eqref{eq17.7:管の中の定常流の方程式}から求めればよい.
\[
    v_0 \pqty{ -\frac{2}{a^2}-\frac{2}{b^2} } = -\frac{\D p}{\eta l}
    \qquad\yueni v_0 = \frac{\D p}{2\eta l} \cdot \frac{a^2b^2}{a^2+b^2}
\]
よって,
\[
    v(y,z) = \frac{\D p}{2\eta l} \cdot \frac{a^2b^2}{a^2+b^2} \pqty{1-\dfrac{y^2}{a^2}-\dfrac{z^2}{b^2}}
\]
を得る.


流量は
\[
    Q = \int_{y^2/a^2+z^2/b^2\ika1} \rho v \; dydz
\]
で与えられる.
これを計算するために,まず変数変換$\eta=y/a, \zeta=z/b$により$(\eta,\zeta)$座標系に移り,次に$(\eta,\zeta)$系での極座標系に移る.
\[
    v = v_0 (1-\eta^2-\zeta^2) = v_0(1-r^2), \quad
    \int_{y^2/a^2+z^2/b^2\ika1} dydz = \int_0^1 ab\cdot 2\pi r \; dr
\]
であるから
\[
    Q = \rho v_0 \cdot 2\pi ab \int_0^1 r(1-r^2)dr
    = \rho \frac{\D p}{2\eta l} \frac{a^2b^2}{a^2+b^2} \cdot 2\pi ab \cdot \frac{1}{4}
    = \frac{\pi\D p}{4\nu l} \cdot \frac{a^3b^3}{a^2+b^2} 
\]

    
\end{kaitou}







%%%%%%%%%% 問題3 %%%%%%%%%%

\begin{mondai}{}{}
1辺の長さが$a$の正三角形の断面をもつ管の中の流れを求めよ.
\end{mondai}
\begin{kaitou}
正三角形内の任意の点Pから三辺に下ろした垂線の長さを$h_1, h_2, h_3$とすると,解は
\[
    v = v_0 \, h_1h_2h_3
\]
と書ける(これは明らかに境界条件を満たす).
まずはこのことを証明しよう.

\begin{details}
Landau本文では幾何的に証明されているが,ここでは座標を導入する.
\end{details}

図のように座標を設定し,$\mathrm{P}(y,z)$とおく.
三辺の方程式は$z=0, y+\dfrac{z}{\sqrt{3}}=\dfrac{a}{2}, -y+\dfrac{z}{\sqrt{3}}=\dfrac{a}{2}$であるから,点と直線の距離の公式より
\[
    h_1 = z, \;
    h_2 = \frac{\sqrt{3}}{2} \pqty{ \frac{a}{2} - y - \frac{z}{\sqrt{3}} }, \;
    h_3 = \frac{\sqrt{3}}{2} \pqty{ \frac{a}{2} + y - \frac{z}{\sqrt{3}} }
\]
となり,
\[
    v = v_0 \, h_1h_2h_3
    = \frac{3}{4} v_0 z \bqty{ \pqty{\frac{a}{2} - \frac{z}{\sqrt{3}}}^2 -y^2 } .
\]
よって
\begin{align*}
    \pdv[2]{v}{y} + \pdv[2]{v}{z} &=
    \frac{3}{4} v_0 \pqty{\pdv[2]{y} + \pdv[2]{z}} \, z \bqty{ \pqty{\frac{a}{2} - \frac{z}{\sqrt{3}}}^2 -y^2 } \\
    &= \frac{3}{4} v_0 \bqty{ -2z + \pdv[2]{z} \pqty{ \frac{a^2}{4}z - \frac{a}{\sqrt{3}}z^2 + \frac{1}{3}z^3 } } \\
    &= \frac{3}{4} v_0 \pqty{ -2z - \frac{2a}{\sqrt{3}} + 2z } = - \frac{\sqrt{3}}{2}av_0
\end{align*}
は定数となり\eqref{eq17.7:管の中の定常流の方程式}を満たす.
これと$-\dfrac{\D p}{\eta l}$を比べ$v_0 = \dfrac{2}{\sqrt{3}a} \cdot \dfrac{\D p}{\eta l}$となるから
\[
    v = \frac{2}{\sqrt{3}a} \cdot \frac{\D p}{\eta l} h_1h_2h_3
    = \frac{\sqrt{3}\D p}{2\eta la} z \bqty{ \pqty{\frac{a}{2} - \frac{z}{\sqrt{3}}}^2 -y^2 } .
\]


流量は
\begin{align*}
    Q &= \int \rho v \; dydz \\
    &= \frac{\sqrt{3}\D p}{2\nu la} \int_0^{\frac{\sqrt{3}}{2}a} dz \int_{-(\frac{a}{2} - \frac{z}{\sqrt{3}})}^{\frac{a}{2} - \frac{z}{\sqrt{3}}} dy \cdot z \bqty{ \pqty{\frac{a}{2} - \frac{z}{\sqrt{3}}}^2 -y^2 } \\
    &= \frac{\sqrt{3}\D p}{2\nu la} \int_0^{\frac{\sqrt{3}}{2}a} dz \cdot z \cdot \frac{1}{6} \Bqty{2\pqty{\frac{a}{2} - \frac{z}{\sqrt{3}}}}^3 \\
    &= \frac{\sqrt{3}\D p}{2\nu la} \cdot \frac{1}{6} \pqty{\frac{2}{\sqrt{3}}}^3 \int_0^{\frac{\sqrt{3}}{2}a} dz \cdot z \pqty{ \frac{\sqrt{3}}{2}a-z }^3 \\
    &= \frac{\sqrt{3}\D p}{2\nu la} \cdot \frac{1}{6} \pqty{\frac{2}{\sqrt{3}}}^3 \cdot \frac{1}{20} \pqty{\frac{\sqrt{3}}{2}a}^5
    = \frac{\sqrt{3}\D p a^4}{320 \nu l} .
\end{align*}

\begin{details}
※流量の積分計算では(いわゆる)1/6公式
\[
    -\int_{\alpha}^{\beta} (x-\alpha)(x-\beta)\; dx = \frac{1}{6}(\beta-\alpha)^3
\]
および1/20公式(?)
\[
    -\int_{\alpha}^{\beta} (x-\alpha)(x-\beta)^3\; dx = \frac{1}{20}(\beta-\alpha)^5
\]
を用いた.    
\end{details}



    
\end{kaitou}







%%%%%%%%%% 問題4 %%%%%%%%%%

\begin{mondai}{}{}
共通の軸をもつ2つの円筒があり,外側の円筒は半径$R_2$で静止し,内側の円筒は半径$R_1$で速度$u$で軸の方向に動いている.
2つの円筒間の流体の運動を求めよ.
\end{mondai}
\begin{kaitou}
円筒の軸を$z$軸とする円筒座標を用い,流れは$z$方向を向いているとする.
流れは$r$のみに依存し$v_z=v(r)$と書ける.
また,移流項が0になること,圧力が$r$のみに依存することは容易に分かる($z$方向の圧力勾配はないという仮定であろう).
Navier-Stokes方程式の$z$成分から
\[
    \Laplacian v = \frac{1}{r} \dv{r} \pqty{ r \dv{v}{r} } = 0
    \qquad\yueni v = a \log r + b .
\]
境界条件:$r=R_1$で$v=u$,$r=R_2$で$v=0$から定数$a,b$を求めよう.
\[
    \begin{cases}
        u = a \log R_1 + b \\
        0 = a \log R_2 + b \\
    \end{cases}
    \qquad\yueni u = a \log (R_1/R_2)
\]
\[
    a = \frac{u}{\log(R_1/R_2)} \; (<0), \quad
    b = - a\log R_2 = -u \frac{\log R_2}{\log(R_1/R_2)}
\]
よって
\[
    v = u \frac{\log(r/R_2)}{\log(R_1/R_2)}
    = u \frac{\log(R_2/r)}{\log(R_2/R_1)}
\]
を得る.

円筒の単位長さに働く摩擦力を求めよう.
ゼロでない粘性応力テンソルは
\[
    \sigma_{zr}' = \eta \pdv{v_z}{r} = -\frac{\eta u}{r\log(R_2/R_1)}
\]
のみであるから,内側の円筒(単位長さあたりの表面積は$2\pi R_1$)に働く摩擦力の$z$成分は
\[
    2\pi R_1 \eval{\sigma_{zr}'}_{r=R_1} = -\frac{2\pi\eta u}{\log(R_2/R_1)} .
\]
同様に外側の円筒については$\dfrac{2\pi\eta u}{\log(R_2/R_1)}$である.


    
\end{kaitou}







%%%%%%%%%% 問題5 %%%%%%%%%%

\begin{mondai}{}{}
水平面に対して角度$\alpha$で傾いている板の上に,厚さ$h$の流体層があり,上面は自由表面である.
重力により生じる流れを求めよ.
\end{mondai}
\begin{kaitou}
板の表面を$xy$平面とし,$x$軸は流れの方向,$z$軸は板から自由表面へ向かう方向にとる.
流れは$z$のみに依存し$v_x=v(z)$となる.
重力場中のNavier-Stokes方程式より
\[
    \begin{cases}
        0 = \eta \ddv[2]{v}{z} + \rho g \sin\alpha \\[7pt]
        0 = -\ddv{p}{z} - \rho g \cos\alpha \\
    \end{cases}
    \qquad\yueni
    \begin{cases}
        v = - \dfrac{g\sin\alpha}{2\nu} z^2 + az + b \\[7pt]
        p = -\rho gz\cos\alpha + c \\
    \end{cases} .
\]
境界条件:
$z=h$(自由表面)で$\sigma_{xz}=\eta\ddv{v}{z}=0,\; \sigma_{zz}=-p=-p_0$(大気圧),
$z=0$で$v=0$から定数$a,b,c$を決める.
\[
    \eval{\ddv{v}{z}}_{z=h} = - \frac{g\sin\alpha}{\nu} h + a = 0
    \qquad\yueni a = \frac{gh \sin\alpha}{\nu}
\]
\[
    \eval{p}_{z=h} = -\rho gh\cos\alpha + c = p_0
    \qquad\yueni c = p_0 + \rho gh\cos\alpha
\]
\[
    \eval{v}_{z=0} = b = 0
\]
以上より
\[
    \begin{cases}
        v = \dfrac{g\sin\alpha}{2\nu} (-z^2+2hz) = \dfrac{g\sin\alpha}{2\nu} \bqty{ h^2 - (z-h)^2 } \\[7pt]
        p = p_0 - \rho g \cos\alpha (z-h) \\
    \end{cases}    
\]

$y$方向の単位長さを通過する流量は
\[
    Q = \rho \int_0^h v \; dz
    = \dfrac{\rho g\sin\alpha}{2\nu} \pqty{ -\frac{h^3}{3} + h^3 }
    = \frac{\rho gh^3\sin\alpha}{3\nu}
\]




 
\end{kaitou}







%%%%%%%%%% 問題6 %%%%%%%%%%

\begin{mondai}{}{}
等温の粘性理想気体が円形断面(半径$R$)の管の中を流れているとき,圧力がどのように減少するか求めよ
(理想気体の粘性率$\eta$は圧力によらないことに注意せよ).
\end{mondai}
\begin{kaitou}
\eqref{eq17.10:Hagen-Poiseuilleの法則}より
\[
    - \dv{p}{x} = \frac{8\eta Q}{\pi\rho R^4}.
    \mytag{1}
\]
流量$Q$は,気体が圧縮性・非圧縮性のどちらであっても一定である.

\begin{itemize}
    \item 管の短い区間では,圧力勾配が非常に大きくない限り,気体を非圧縮とみなすことができる.
        つまり,$\rho$は定数であり,\ajMaru{1}が圧力変化を与える.
    \item 管の長い区間を考えるときには,$\rho$は変化しうるから,もはや$p$は$x$の1次関数ではなくなる.
        分子量を$m$,ボルツマン定数を$k_B$とすると$\rho=\dfrac{mp}{k_BT}$であるから,\ajMaru{1}に代入して
        \[
            - \dv{p}{x} = \frac{8\eta Q k_BT}{\pi mR^4} \cdot \frac{1}{p}
            \qquad\yueni -pdp = \frac{8\eta Q k_BT}{\pi mR^4} dx
        \]
        両辺を積分し,両端の圧力を$p_1,p_2$,管の長さを$l$とすれば
        \[
            {p_1}^2 - {p_2}^2 = \frac{16\eta Q k_BT}{\pi mR^4} l
        \]
\end{itemize}








    
\end{kaitou}


\BackToTheToc