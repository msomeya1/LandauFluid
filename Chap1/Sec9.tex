\section{ポテンシャル流}\label{sec:9}
循環の保存則から,重要な結果を導くことができる.

まずは定常流を考える.
ある流線上のある点で,渦度が0であるとする.その点を取り囲む無限小の閉曲線について循環を計算すると
\[
    \varGamma = \int_S \Rot\v\cdot\dS = 0
\]
であるから,この点のまわりの循環は0である.
時間の経過とともに,閉曲線は流体とともに動くが,無限小のままであり,同じ流線を囲むことは明らかである(定常流では流線と流体粒子の軌跡が一致することに注意).
一方で循環の値も0で一定のままである.
よって,流線上の全ての点で循環は0となり
\[
    \text{流線上のある1点で} \; \vec{\omega}=\vec{0} \; \text{ならば,その流線上の任意の点で} \; \vec{\omega}=\vec{0}
    \mytag{1}
\]
という結論が導かれる
\footnote{\ajMaru{1}は乱流では意味を持たない.また,流体粒子が衝撃波面を通過した後も,一般に渦度は0でなくなる(衝撃波面を通過すると,エントロピーが増加するため).}
.

定常流でない場合にも,流線の代わりに流跡線に対して\ajMaru{1}が成り立つ.


以上から,次の議論が可能であるように思える;
無限遠から流れてくる一様な定常流中に物体を置いたとき,無限遠では$\v=\const$つまり$\vec{\omega}=\vec{0}$であるから,
\ajMaru{1}より全空間にわたって$\vec{\omega}=\vec{0}$となる.
このような流れは,\emph{渦なしの流れ}(\emph{ポテンシャル流})と呼ばれる.
つまり
\[
    \text{無限遠で一様な,物体まわりの定常流はポテンシャル流である.}
    \mytag{2}
\]

また,次のことも言えそうである;
ある瞬間に流体全体で渦なしなら,流体内の任意の閉曲線に関する循環は0である
\footnote{
厳密には,流体が単連結である場合に限る.
多重連結の場合には,1点に縮めようとするときに流体の境界を越えてしまうような閉曲線についての循環は0にはならない(ストークスの定理が適用できない).
多重連結領域内の閉曲線についての循環が0となるのは,領域に切れ目を入れて,閉曲線の存在する領域を単連結にすることができる場合である.}
から,Kelvinの循環定理よりそれ以降も渦なしである.つまり
\[
    \text{ある瞬間に流れがポテンシャル流ならばその後もポテンシャル流である.}
    \mytag{3}
\]
(特に,最初に静止状態にあった流れはポテンシャル流でなければならない.)


しかし実際には,\ajMaru{2}\ajMaru{3}は正しくない.
というのは,物体の表面を通る流線に対しては,流線を取り囲む閉曲線をとることができないから,この流線上のいたるところで$\vec{\omega}=\vec{0}$とは限らないからである.
この場合,物体表面で「はがれ」が起こりうる.
すなわち,流線はある距離の間は物体表面に沿っているが,ある点で物体から離れ,流体の中に入っていく.
この結果,この点から後方へ「接線速度の不連続面」が生じる.
この面上では速度は接線方向に平行で,値は面の両側で不連続となる(一方の流体が他方の上を「滑る」).
数学的に言えば,この面は渦度が0でない面に対応している
\footnote{面に沿った,長さ$\delta l$,幅$\delta d$の長方形に対してStokesの定理を用いると
$\Rot\v \cdot \delta S \simeq (\v_2-\v_1)\delta l \neq 0$
}
.


このような不連続があるとき,理想流体の運動方程式の解は一意ではない.すなわち,不連続面の下側で速度が任意の値をとるから無限の解が存在してしまう.
しかし,このような不連続な解を論じることはあまり意味がない.なぜなら,接線方向の速度不連続は不安定であり,\emph{乱流}となるからである
.

実際には,そもそも完全な理想流体なるものが存在しないために,一意的な解が存在する.
実在流体は粘性を持ち,たとえそれがどんなに小さくても,物体に隣接する領域に\emph{境界層}が生じる.
そして,境界層理論(第4章)に基づいて,無数の中からただ1つの解が選ばれる
\footnote{一般に,物体まわりの流れでは,はがれを伴う解を捨てなければならない.はがれが生じるなら,乱流となる.}.

このように,物体まわりのポテンシャル流は実際には存在しそうもないが,それでも物体まわりのポテンシャル流の研究が重要となる場合がある.
1つは流線形の物体のまわりに生じる流れで,物体表面の薄い流体層と背後の伴流部分を除けば,流れはポテンシャル流となる(\S~46).

もう1つは,流体中で物体が微小に振動する場合である.
物体の代表的な長さ$l$に比べて振動の振幅$a$が非常に小さい($a \ll l$)場合には,Euler方程式の各項のオーダーを比較することで,物体まわりの流れがポテンシャル流となることが示せる;
振動する物体の速度を$u$,振動数を$\omega$とする(物体の変位は$ae^{i\omega t}$,速度は$ue^{i\omega t}$となる.ただし$u \sim a\omega$).
流体の速度や時間変化の仕方は物体の振動のそれに近いだろうから
\[
    \pdv{\v}{t} \sim \omega u \sim \frac{u^2}{a} .
\]
次に,流体の速度は,物体の大きさ$l$程度の距離にわたって大きく変化するだろうから
\[
    (\v\cdot\Grad)\v \sim \frac{v^2}{l} .
\]
$a \ll l$であるから,移流項を$\dpdv{\v}{t}$に比べて無視することができ,流体の運動方程式は
\[
    \pdv{\v}{t} = - \Grad h
\]
となる.rotをとって
\[
    \pdv{t} \Rot \v = 0
    \qquad\yueni \Rot\v = \const
\]
振動では速度の時間平均は0であるから,$\Rot\v = \vec{0}$となる.
よって,微小振動する物体まわりの流れは第1近似でポテンシャル流である.


\subsection*{ポテンシャル流の一般的性質}
さて,ポテンシャル流の一般的な性質を調べよう.
まず,ポテンシャル流となるのは(事実上)等エントロピー(順圧)の場合に限る.
もし流れが等エントロピー的でなければ,循環の保存則は成り立たないから,(もしある瞬間に偶然ポテンシャル流になったとしても)渦度0の状態は持続しないからである.

次に,ポテンシャル流中にとった閉曲線についての循環を考えると,Stokesの定理より
\begin{equation}\label{eq9.1:ポテンシャル流は循環0}
    \oint_C \v\cdot\dl = \int_S \Rot\v \cdot \dS = 0
\end{equation}
となる.もし,ある流線が閉じていたとすると,この閉曲線の沿った循環は明らかに0ではない
\footnote{流線の方向は$\v$の方向であるから,常に$\v\cdot\dl\neq0$であり,$\varGamma\neq0$である.}.
よって,
\[
    \text{ポテンシャル流中には閉じた流線は存在できない}
    \mytag{4}
\]
ということになる
\footnote{もちろん非ポテンシャル流(渦ありの流れ)では循環は0ではないから,閉じた流線は存在しうる.
しかし,渦ありだからと言って必ず閉じた流線が存在する,というわけでもない.}.

繰り返しの注意になるが,流体が多重連結領域の場合には,\eqref{eq9.1:ポテンシャル流は循環0}や\ajMaru{4}は成り立たない.


\subsection*{速度ポテンシャル}
rotが0である任意のベクトル場と同様,ポテンシャル流の速度は\emph{速度ポテンシャル}$\phi$を用いて
\begin{equation}
    \v = \Grad\phi
\end{equation}
と書くことができる(他の分野と異なり,マイナスを付けないのが慣例である).
\eqref{eq2.10:rotとhを用いたEuler方程式}の形のEuler方程式に代入し
\[
    \pdv{t}(\Grad\phi) + \Grad \pqty{\frac{1}{2}v^2} - \v \times \cancel{\Rot( \Grad\phi)} = - \Grad h,
\]
\[
    \Grad \pqty{ \pdv{\phi}{t} + \frac{1}{2}v^2 + h } = 0.
\]
よって
\begin{equation}\label{eq9.3:圧力方程式}
    \pdv{\phi}{t} + \frac{1}{2}v^2 + h = f(t) \quad\text{(時間の任意関数)}.
\end{equation}
これはポテンシャル流における方程式の第一積分で,\emph{圧力方程式}と呼ばれる.
これを
\[
    \pdv{t} \pqty{ \phi - \int f\,dt } + \frac{1}{2}v^2 + h = 0
\]
と書き換えればわかるように,$\displaystyle\phi'=\phi-\int f\,dt$とおいても$\Grad\phi'= \Grad\phi = \v$となる.
つまり,$f(t)$は一般性を失うことなく0とすることができる.

特に定常流の場合には,$\dpdv{\phi}{t} = 0, \; f(t) = \const$とすることができるから
\begin{equation}\label{eq9.4:定常流での圧力方程式}
    \frac{1}{2}v^2 + h = \const
\end{equation}
となり,Bernoulliの定理と同じ形になる.
しかし,\eqref{eq5.3:Bernoulliの定理}右辺の$\const$は流線によって異なる値をとるのに対し,\eqref{eq9.4:定常流での圧力方程式}右辺の$\const$は
流体全体にわたって一定であることに注意する必要がある(ここに,ポテンシャル流の優位性が現れている).




\BackToTheToc