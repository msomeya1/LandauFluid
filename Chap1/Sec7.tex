\section{運動量フラックス}
\S~\ref{sec:6}と同様に,$\rho\v$の時間変化を発散の形で書こう.
連続の式とEuler方程式を用いると
\begin{align*}
	\pdv{t}(\rho v_i) &= v_i \pdv{\rho}{t} + \rho \pdv{v_i}{t} \\
	&= v_i \bqty{ -\pdv{x_j}(\rho v_j) } - \rho v_j \pdv{v_i}{x_j} - \pdv{p}{x_i} \\
	&= - \pdv{x_j}(\rho v_iv_j) - \pdv{p}{x_i} \\
	&= - \pdv{x_j}( p\delta_{ij} + \rho v_iv_j ).
\end{align*}
よって,\emph{運動量フラックス密度テンソル}を
\setcounter{equation}{1}
\begin{equation}
	\varPi_{ij} \equiv p\delta_{ij} + \rho v_iv_j
\end{equation}
で定義すれば
\footnote{
これは2階テンソルであり,
$
\displaystyle	
\mat{\varPi} =
\begin{pmatrix}
	p + \rho {v_x}^2 & \rho v_xv_y & \rho v_xv_z \\
	\rho v_xv_y & p + \rho {v_y}^2 & \rho v_yv_z \\
	\rho v_xv_z & \rho v_yv_z & p + \rho {v_z}^2 \\
\end{pmatrix}
$
と書けば明らかなように対称である.
},
\setcounter{equation}{0}
\begin{equation}\label{eq7.1:理想流体の運動量流保存則(微分形)}
	\pdv{t}(\rho v_i) = -\pdv{\varPi_{ij}}{x_j}
\end{equation}
となる.

$\varPi_{ij}$の意味を明らかにするため\eqref{eq7.1:理想流体の運動量流保存則(微分形)}を体積積分する.
\setcounter{equation}{2}
\begin{equation}
	\pdv{t} \int_V \rho v_i \dV = - \int_V \pdv{\varPi_{ij}}{x_j} dV = - \int_S \varPi_{ij} dS_j
\end{equation}
\begin{details}
2つ目の等号はGaussの発散定理を用いているだけである.
明示的に書けば,例えば$i=1$の成分は
\[
	\int_V \pdv{\varPi_{1j}}{x_j} dV = \int_V \pqty{ \pdv{\varPi_{11}}{x} + \pdv{\varPi_{12}}{y} + \pdv{\varPi_{13}}{z} } \dV
	= \int_S ( \varPi_{11} dS_x + \varPi_{12} dS_y + \varPi_{13} dS_z ) = \int_S \varPi_{1j} dS_j
\]
\end{details}
左辺は,考えている体積に含まれる運動量の$i$成分の時間変化率を表している.
よって右辺は,単位時間に表面から流れ出す運動量である.
$\vec{n}$を外向き法線ベクトルとすると$\varPi_{ij} dS_j = \varPi_{ij} n_j dS$であるから,単位表面積を通る運動量フラックスは$\varPi_{ij} n_j$であることがわかる.
ベクトルで書けば
\begin{equation}\label{eq7.4:運動量フラックスベクトル}
	p \vec{n} + \rho\v (\v\cdot\vec{n})
\end{equation}
となる.

このように$\varPi_{ij}$は,$x_j$軸に垂直な単位面積を単位時間に流れる運動量の$j$成分を表す.
また\eqref{eq7.4:運動量フラックスベクトル}は,$\vec{n}$方向の運動量フラックスを与える.
$\vec{n}$を$\v$の方向にとることにより,運動と同じ(縦)方向に輸送される運動量は$p+\rho v^2$であることがわかる.
また$\vec{n}$を$\v$と垂直な方向にとることにより,垂直(横)方向に輸送される運動量は$p$であることがわかる.





\BackToTheToc