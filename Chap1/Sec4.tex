\section{対流が起こらない条件}\label{sec:4}
流体は,熱力学的には平衡でなくても,力学的には平衡(マクロな運動がない)でありうる(例えば\eqref{eq3.1:静水圧平衡}は,温度が一様でなくても成立する力学的平衡の条件である).
このような平衡は,ある特定の条件下でのみ安定である.
不安定の場合,流体中の温度を一様にするような流れ(\emph{対流})が生じる.
よって,安定の条件は対流がないことであり,以下のように定式化される.
\begin{details}
回りくどい言い方だが,対偶論法を使っていると思えばよい.
つまり,「不安定$\implies$対流がある」の対偶を取ると「対流がない$\implies$安定」となるから,
安定の条件は対流がないことだ,というわけである.
\end{details}


高さ$z$にある,比体積$V(p,s)$の流体部分を考える($p,s$は高さ$z$における平衡状態での圧力とエントロピー).
この流体が断熱的に$\zeta$だけ上昇し,比体積が$V(p',s)$になったとする($p'$は高さ$z+\zeta$での圧力).
平衡が安定であるためには,元の位置へ戻そうとす力が働かなければならない.
\begin{details}
これは必要条件だが十分条件とは限らない,というが,その意味が今ひとつわからない.
中立安定のように,平衡点へ収束はしないがその周りに留まり続ける例があるため,「復元力が働く$\implies$安定」とは限らない,ということだろうか?
\end{details}
\noindent
つまり,変位した流体が,そこに元からあった流体よりも重い(比体積で比べると小さい)から
\[
    V(p',s') > V(p',s)
\]
でなければならない($s'$は高さ$z+\zeta$における平衡状態でのエントロピー).
$\D s \equiv s'-s \simeq \ddv{s}{z}\zeta$とおくと
\[
    V(p',s') - V(p',s) = V(p', s+\D s) - V(p',s) \simeq \PDV{V}{s}{p} \D s = \PDV{V}{s}{p} \dv{s}{z}\zeta
\]
であるから
\begin{equation}\label{eq4.1:対流がないときのエントロピー勾配の条件1}
    \PDV{V}{s}{p} \dv{s}{z} > 0.
\end{equation}
熱力学の公式
\[
    \PDV{V}{s}{p} = \frac{T}{c_p} \PDV{V}{T}{p}
\]
を用いると
\begin{details}
Gibbsの自由エネルギー,エンタルピーの全微分から導かれるMaxwell関係式
\[
    \PDV{s}{p}{T} = -\PDV{V}{T}{p}, \quad
    \PDV{T}{p}{s} = \PDV{V}{s}{p}
\]
を用いると,定圧比熱は
\[
    c_p = T\PDV{s}{T}{p} = -T \frac{ \dPDV{s}{p}{T} }{ \dPDV{T}{p}{s} } = T \frac{ \dPDV{V}{T}{p} }{ \dPDV{V}{s}{p} } .
\]
\end{details}
\begin{equation} 
    \PDV{V}{T}{p} \dv{s}{z} > 0.
\end{equation}
多くの物体は加熱により膨張するから$\dPDV{V}{T}{p}>0$であり
\begin{equation}
    \dv{s}{z} > 0.
\end{equation}
つまり,エントロピーは高さとともに増加しなければならない.

ここから,温度勾配の満たすべき条件を導こう.$s$を$T,p$の関数とみて
\begin{align*}
    \dv{s}{z} &= \PDV{s}{T}{p} \dv{T}{z} + \PDV{s}{p}{T} \dv{p}{z} \\
    &= \frac{c_p}{T} \dv{T}{z} -\PDV{V}{T}{p} \dv{p}{z} \\
    & \gyoukan{ \Eqref{gray}{eq3.4:1次元の静水圧平衡}より $dp/dz=-\rho g=-g/V$} \\
    &= \frac{c_p}{T} \dv{T}{z} + \frac{g}{V}\PDV{V}{T}{p} >0.
\end{align*}
\spade
$\beta\equiv\dfrac{1}{V}\dPDV{V}{T}{p}$を(体積)熱膨張率として
\begin{equation}\label{eq4.4:対流ないときの温度勾配1}
    -\dv{T}{z} < \frac{gT}{c_p V} \PDV{V}{T}{p} = \frac{g\beta T}{c_p}.
\end{equation}
理想気体の場合$\beta T = \dfrac{1}{V}\cdot\dfrac{nR}{p}T = 1$であるから
\begin{equation}\label{eq4.5:対流ないときの温度勾配2}
    -\dv{T}{z} < \frac{g}{c_p} \equiv\Gamma_d \,\text{(乾燥断熱減率)}.
\end{equation}
\eqref{eq4.4:対流ないときの温度勾配1},\eqref{eq4.5:対流ないときの温度勾配2}の右辺は(普通の物質では)正である.
よって,高さとともに温度が上がるなら安定である.
また,高さとともに温度が下がる場合でも,温度減率がある程度小さく$-\ddv{T}{z} < \dfrac{g\beta T}{c_p}$ならば安定だが,
温度減率が大きく$-\ddv{T}{z} > \dfrac{g\beta T}{c_p}$ならば不安定であり,対流が発生する.
\SI{20}{\tccelsius}の水の場合$\dfrac{g\beta T}{c_p} \simeq \SI{1}{\tccelsius}/\SI{6.7}{km}$,
大気の場合$\Gamma_d \simeq \SI{1}{\tccelsius}/\SI{100}{m}$である.



\BackToTheToc