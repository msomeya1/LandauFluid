\section{非圧縮性流体中の内部波}

重力波の中には,非圧縮性流体の内部を伝播するものがある.
そのような波動は,重力場によって流体が不均質となったことで生じる.
この場合,必然的に圧力は(したがってエントロピーも)高さによって異なる値をとる.
よって流体粒子が上下方向に変位することで力学的平衡は敗れ,振動が生じる.
なぜなら,運動が断熱的であるため,流体粒子は元の位置でのエントロピーのまま新しい位置に移動し,新しい位置での平衡状態でのエントロピーとは異なるからである.


以下では,重力場によって密度が大きく変化する距離に比べて波長が短く
\footnote{\spade
密度と圧力の勾配の関係は$\Grad p = \dPDV{p}{\rho}{s} \Grad\rho = c^2 \Grad\rho$で与えられる($c$は音速).
よって静水圧平衡の式$\Grad p = \rho\vec{g}$は$\Grad\rho=\dfrac{\rho}{c^2}\vec{g}$となる.
したがって,重力場によって密度が大きく変化する距離を$l$とすると,$\dfrac{\rho}{l} \sim \dfrac{\rho g}{c^2}$より$l \sim \dfrac{c^2}{g}$であり,
空気では$l \sim \SI{10}{km}$,水では$l \sim \SI{200}{km}$となる.以下では,波長がこれより短い波動を扱うことになる.
},
圧力による密度変化は無視できるとする.
すなわち,密度変化は熱膨張によってのみ生じると仮定する(Boussinesq近似).




このような運動の方程式系を書き下そう.
以下,添字0は力学的平衡での値,ダッシュ($'$)はそこからのずれを表すものとする.
エントロピー$s=s_0+s'$の保存則は,微小量の1次までとる近似で
\begin{equation}\label{eq13.1:内部波でのエントロピー保存則}
    \pdv{s'}{t} + \v \cdot \Grad s_0 = 0
\end{equation}
となる($\v$も微小量とする).
$s_0$は,平衡状態における他の量と同様,鉛直座標$z$の(与えられた)関数である.



次に,Euler方程式で2次の微小量$(\v\cdot\Grad)\v$を無視し,平衡状態での圧力分布が$\Grad p_0 = \rho_0 \vec{g}$で与えられることを用いると,同程度の近似で
\begin{align*}
    \pdv{\v}{t} &= - \frac{1}{\rho} \Grad p + \vec{g} \\
    &= - \frac{1}{\rho_0+\rho'} \Grad(p_0+p') + \vec{g} \\
    &\simeq - \frac{1}{\rho_0} \pqty{1-\frac{\rho'}{\rho_0}} \Grad(p_0+p') + \vec{g} \\
    &\simeq - \cancel{ \frac{1}{\rho_0}\Grad p_0 } - \frac{1}{\rho_0}\Grad p' + \frac{\rho'}{{\rho_0}^2} \Grad p_0 + \cancel{\vec{g}} \\
    &= - \frac{1}{\rho_0}\Grad p' + \frac{\rho'}{{\rho_0}^2} \Grad p_0 .
\end{align*}
第1項について,波長程度の距離では$\rho_0$が変化しないと仮定しているから,$\rho_0$をgradの中に入れることができる.
また,$\Grad p_0 = \rho_0 \vec{g}$を用いて第2項を$\dfrac{\rho'}{\rho_0} \vec{g}$と書き,
さらに密度変化がエントロピー変化のみであることから$\rho'=\dPDV{\rho_0}{s_0}{p}s'$を代入すると$\dfrac{\vec{g}}{\rho_0} \dPDV{\rho_0}{s_0}{p} s'$となる.
したがってEuler方程式は
\begin{equation}\label{eq13.2:内部波でのEuler方程式}
    \pdv{\v}{t} = \frac{\vec{g}}{\rho_0} \PDV{\rho_0}{s_0}{p} s' - \Grad\pqty{\frac{p'}{\rho_0}} .
\end{equation}
$\rho_0$が波長程度の距離では変化しないことから,連続の式は
\begin{equation}\label{eq13.3:内部波での連続の式}
    \Div\v=0
\end{equation}
となる.


さて,方程式系\eqref{eq13.1:内部波でのエントロピー保存則}〜\eqref{eq13.3:内部波での連続の式}の解を,平面波
\[
    \v \propto e^{i( \vec{k}\cdot\vec{r} -\omega t )}
\]
の形に求めよう($s',p'$も同様).
\eqref{eq13.3:内部波での連続の式}より
\begin{equation}\label{eq13.4:内部波は縦波}
    \vec{k} \cdot \v = 0 ,
\end{equation}
つまり速度が\emph{波数ベクトル}$\vec{k}$に垂直な横波である.
一方,\eqref{eq13.1:内部波でのエントロピー保存則}\eqref{eq13.2:内部波でのEuler方程式}へ代入すると
\[
    -i\omega s' + \v \cdot \Grad s_0 = 0 ,
    \mytag{1}
\]
\[
    -i\omega\v = \frac{1}{\rho_0} \PDV{\rho_0}{s_0}{p} s' \vec{g} - \frac{i\vec{k}}{\rho_0} p' .
    \mytag{2}
\]
\ajMaru{2}と$\vec{k}$の内積をとって\eqref{eq13.4:内部波は縦波}を用いると
\[
    0 = \frac{1}{\rho_0} \PDV{\rho_0}{s_0}{p} s' \vec{g}\cdot\vec{k} - \frac{ip'}{\rho_0} k^2 .
\]
\begin{align*}
    \yueni \frac{ip'}{\rho_0} &= \frac{1}{\rho_0} \PDV{\rho_0}{s_0}{p} s' \frac{ \vec{g}\cdot\vec{k} }{k^2} \\
    & \gyoukan{$z$軸と$\vec{k}$のなす角を$\theta$とすると$\vec{g}\cdot\vec{k}=-gk\cos\theta$} \\
    &= - \frac{1}{\rho_0} \PDV{\rho_0}{s_0}{p} s' \frac{g \cos\theta}{k} .
\end{align*}
再び\ajMaru{2}に戻して
\begin{align*}
    -i\omega\v &= \frac{1}{\rho_0} \PDV{\rho_0}{s_0}{p} s' \vec{g} + \frac{1}{\rho_0} \PDV{\rho_0}{s_0}{p} s' g\cos\theta \frac{\vec{k}}{k} \\
    &= \frac{1}{\rho_0} \PDV{\rho_0}{s_0}{p} s'\pqty{ \vec{g} +  g\cos\theta \frac{\vec{k}}{k} }
\end{align*}
両辺と$\Grad s_0 = \ddv{s_0}{z} \Unit{z}$の内積をとり,左辺に\ajMaru{1}を用いると
\[
    -i\omega(i\omega s') = \frac{1}{\rho_0} \PDV{\rho_0}{s_0}{p} s' \dv{s_0}{z} {(-g+g\cos^2\theta)} ,
\]
\[
    \omega^2 = - \frac{g}{\rho_0} \PDV{\rho_0}{s_0}{p} \dv{s_0}{z} \sin^2\theta .
\]
よって
\begin{equation}\label{eq13.5:内部波の分散関係}
    \omega^2 = {\omega_0}^2 \sin^2\theta ,
\end{equation}
\begin{equation}\label{eq13.6:内部波の固有振動数}
    {\omega_0}^2 = - \frac{g}{\rho} \PDV{\rho}{s}{p} \dv{s}{z} .
\end{equation}
ただし,平衡状態での量を表す添字0は省いた.

\eqref{eq4.1:対流がないときのエントロピー勾配の条件1}より
$\dPDV{V}{s}{p} \ddv{s}{z} = - \dfrac{1}{\rho^2} \dPDV{\rho}{s}{p} \ddv{s}{z} > 0$であるから,
\eqref{eq13.6:内部波の固有振動数}の右辺は正である.
すなわち,\S~\ref{sec:4}の安定条件が満たされていれば,実数$\omega_0$は存在する.



\eqref{eq13.5:内部波の分散関係}より,振動数は波数ベクトルの大きさにはよらず,方向のみに依存する.
また$\theta=0$ならば$\omega=0$であるから,鉛直方向に伝わる波動は存在しないことがわかる.



流体が力学的にも熱力学的にも平衡ならば,温度は一定であるから
\[
    \dv{s}{z} = \PDV{s}{p}{T} \dv{p}{z} = -\rho g \PDV{s}{p}{T}
\]
である.$c_p$を単位質量あたりの比熱とすると
\[
    \PDV{s}{p}{T} = \frac{1}{\rho^2} \PDV{\rho}{T}{p}, \quad
    \PDV{\rho}{s}{p} = \frac{T}{c_p} \PDV{\rho}{T}{p}
\]
\begin{details}
後者は\S~\ref{sec:4}の補足で説明した式$\dPDV{V}{s}{p} = \dfrac{T}{c_p} \dPDV{V}{T}{p}$より明らか.
前者は,その補足で言及しているMaxwell関係式$\dPDV{s}{p}{T} = - \dPDV{V}{T}{p}$より明らか.
\end{details}
\noindent
が成り立つから
\[
    {\omega_0}^2 = - \frac{g}{\rho} \frac{T}{c_p} \PDV{\rho}{T}{p} \cdot \bqty{ -\rho g \cdot \frac{1}{\rho^2} \PDV{\rho}{T}{p} }
    = \frac{Tg^2}{c_p \rho^2} \bqty{\PDV{\rho}{T}{p}}^2
\]
\begin{equation}
    \yueni \omega_0 = \sqrt{\frac{T}{c_p}} \frac{g}{\rho} \vqty{\PDV{\rho}{T}{p}}
\end{equation}
となる.
特に理想気体では$-\dfrac{1}{\rho} \dPDV{\rho}{T}{p} = \dfrac{1}{T}$であるから
\begin{equation}
    \omega_0 =   \frac{g}{\sqrt{c_pT}} 
\end{equation}
となる.

\spade
周波数が波数ベクトルの方向に依存するため,群速度$\vec{U} = \dpdv{\omega}{\vec{k}}$は$\vec{k}$に平行ではなくなる.
$\vec{n} = \dfrac{\vec{k}}{k}$,$\vec{\nu}$を$z$軸上向きの単位ベクトル($=\Unit{z}$)とすると,
$\cos\theta=\dfrac{\vec{k}\cdot\vec{\nu}}{k} = \vec{n}\cdot\vec{\nu}$であるから
\[
    \omega(\vec{k}) = \omega_0 \sqrt{ 1- \pqty{\frac{\vec{k}\cdot\vec{\nu}}{k}}^2 } .
\]
\begin{details}
ここで
\begin{align*}
    \pdv{k_i} \pqty{\frac{\vec{k}\cdot\vec{\nu}}{k}}^2 
    &= \frac{1}{k^2} \pdv{k_i} (\vec{k}\cdot\vec{\nu})^2 + (\vec{k}\cdot\vec{\nu})^2 \pdv{k_i} \pqty{\frac{1}{k^2}} \\
    &= \frac{1}{k^2} \pdv{k_i} (k_j\nu_j k_l\nu_l) + (\vec{k}\cdot\vec{\nu})^2 \pqty{ -\frac{2k_i}{k^4} } \\
    &= \frac{1}{k^2} (\delta_{ij}\nu_j k_l\nu_l + k_j\nu_j \delta_{il}\nu_l) - \frac{2(\vec{k}\cdot\vec{\nu})^2}{k^4} k_i  \\
    &= \frac{k_l\nu_l+k_j\nu_j}{k^2} \nu_i - \frac{2(\vec{k}\cdot\vec{\nu})^2}{k^4} k_i .
\end{align*}
よって$\vec{n} = \dfrac{\vec{k}}{k}$に注意すると
\begin{align*}
    \pdv{\vec{k}} \pqty{\frac{\vec{k}\cdot\vec{\nu}}{k}}^2
    &= \frac{2(\vec{k}\cdot\vec{\nu})}{k^2} \vec{\nu} - \frac{2(\vec{k}\cdot\vec{\nu})^2}{k^4} \vec{k} \\
    &= \frac{2}{k} \bqty{ (\vec{n}\cdot\vec{\nu})\vec{\nu} - (\vec{n}\cdot\vec{\nu})^2\vec{n} } .
\end{align*}
\end{details}
%
%
\begin{align}
    \vec{U} &= \pdv{\omega}{\vec{k}} = \frac{-\omega_0}{2\sqrt{\qquad}} \pdv{\vec{k}} \pqty{\frac{\vec{k}\cdot\vec{\nu}}{k}}^2 \notag \\
    &= \frac{-{\omega_0}^2}{2\omega} \cdot \frac{2}{k} (\vec{n}\cdot\vec{\nu}) \bqty{ \vec{\nu} - (\vec{n}\cdot\vec{\nu})\vec{n} } \notag \\
    &= \frac{-{\omega_0}^2}{\omega k} (\vec{n}\cdot\vec{\nu}) \bqty{ \vec{\nu} - (\vec{n}\cdot\vec{\nu})\vec{n} } 
\end{align}
これと$\vec{n}$との内積$\propto (\vec{n}\cdot\vec{\nu}) - (\vec{n}\cdot\vec{\nu}) = 0$より,$\vec{U}$は$\vec{k}$と垂直であることがわかる.

また,
\[
    \bqty{ \vec{\nu} - (\vec{n}\cdot\vec{\nu})\vec{n} }^2 = \nu^2 -2(\vec{n}\cdot\vec{\nu})^2 +2(\vec{n}\cdot\vec{\nu})^2
    = 1-\cos^2\theta = \sin^2\theta
\]
に注意すると,$\vec{U}$の大きさは
\[
    U = \frac{{\omega_0}^2}{\omega k} |\vec{n}\cdot\vec{\nu}| \cdot |\vec{\nu} - (\vec{n}\cdot\vec{\nu})\vec{n}|
    = \frac{\omega_0}{k\sin\theta} \cos\theta \cdot \sin\theta
    = \frac{\omega_0}{k} \cos\theta ,
\]
$\vec{U}$の鉛直成分は
\[
    \vec{U}\cdot\vec{\nu} = -\frac{{\omega_0}^2}{\omega k} (\vec{n}\cdot\vec{\nu}) \bqty{ 1-(\vec{n}\cdot\vec{\nu})^2 }
    = -\frac{\omega_0}{k\sin\theta} \cdot \cos\theta \sin^2\theta
    = -\frac{\omega_0}{k} \cdot \cos\theta \sin\theta
\]
となる.



\BackToTheToc