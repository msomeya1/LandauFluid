\section{エネルギーフラックス}\label{sec:6}
空間に固定された体積要素に含まれるエネルギーの収支は,以下の式で表される.
\begin{equation}\label{eq6.1:理想流体のエネルギー流保存則(微分形)}
    \pdv{t}\pqty{ \frac{1}{2}\rho v^2 + \rho\varepsilon} + \Div \bqty{\rho\v\pqty{\frac{1}{2}v^2 + h}} = 0
\end{equation}
あるいは,$h=\varepsilon+\dfrac{p}{\rho}$より
\[
    \pdv{t} \pqty{ \frac{1}{2}\rho v^2 + \rho\varepsilon } + \Div\bqty{ \rho\v \pqty{ \frac{1}{2}v^2 + \varepsilon + \frac{p}{\rho} } } = 0.
    \mytag{1}
\]
以下\ajMaru{1}を証明しよう.

\begin{details}
連続の式
\[
	\pdv{\rho}{t} + \Div(\rho\v) = 0
\]
とEuler方程式
\[
	\rho \pqty{ \pdv{\v}{t} + \Grad\frac{v^2}{2} - \v\times\Rot\v } = -\Grad p
	\mytag{2}
\]
に注意する.
\end{details}

\begin{align*}
    \text{(\ajMaru{1}の左辺)} 
	&= \frac{v^2}{2} \pdv{\rho}{t} + \rho\v \cdot \pdv{\v}{t} + \varepsilon \pdv{\rho}{t} + \rho \pdv{\varepsilon}{t} + \frac{v^2}{2} \Div(\rho\v) + \rho\v \cdot \Grad \frac{v^2}{2} \\
	&\phantom{=} + \varepsilon \Div(\rho\v) + \rho\v \cdot \Grad \varepsilon + p \Div \v + \v \cdot \Grad p \\
	&= \frac{v^2}{2} \cancel{ \pqty{ \pdv{\rho}{t} + \Div(\rho\v) } } 
    + \ucomment{ \v \cdot \pqty{ \rho\pdv{\v}{t} + \rho\Grad \frac{v^2}{2} + \Grad p } }{$(\,)=\rho\v\times\Rot\v$,これと$\v$の内積$=0$} \\
	&\phantom{=} + \varepsilon \cancel{ \pqty{ \pdv{\rho}{t} + \Div(\rho\v) } } 
	+ \rho \pdv{\varepsilon}{t} + \rho\v \cdot \Grad\varepsilon + p \Div{\v} 
\end{align*}
残った項を消すために,熱力学第1法則を用いる.
\[
	d \varepsilon = Tds -p d \pqty{ \frac{1}{\rho} } = Tds + \frac{p}{\rho^2} d\rho 
\]
より
\[
	\begin{cases}
		\dpdv{\varepsilon}{t} = T \dpdv{s}{t} + \dfrac{p}{\rho^2} \dpdv{\rho}{t} \\
		\Grad\varepsilon = T \Grad s + \dfrac{p}{\rho^2} \Grad \rho 
	\end{cases}
\]
であるから,
\begin{align*}
	\text{(\ajMaru{1}の左辺)} &= \rho \pqty{ T \pdv{s}{t} + \frac{p}{\rho^2} \pdv{\rho}{t} }
		+ \rho \v \cdot \pqty{ T \Grad s + \frac{p}{\rho^2} \Grad \rho } + p \Div{\v} \\
	&= \frac{p}{\rho} \cancel{ \pqty{ \pdv{\rho}{t} + \v \cdot \Grad \rho + \rho \Div \v } }
		+ \rho T \pqty{ \pdv{s}{t} + \v\cdot\Grad s } \\
	&= \rho T \DDt{s} = 0 .
\end{align*}


\begin{details}
後の都合上,結果を一般の形に書き直しておく.	
粘性流体の運動方程式は,\ajMaru{2}の右辺に粘性応力テンソル$\sigma'_{ij}$の微分を加えた
\[
	\rho \pqty{ \pdv{\v}{t} + \Grad\frac{v^2}{2} - \v\times\Rot\v } = -\Grad p + \pdv{\sigma'_{ij}}{x_j}
\]
である.
よって下線部は0ではなく$v_i \dpdv{\sigma'_{ij}}{x_j}$となり,
\[
	\pdv{t}\pqty{ \frac{1}{2}\rho v^2 + \rho\varepsilon } + \Div \bqty{\rho\v\pqty{\frac{1}{2}v^2 + h}} 
	= \rho T \DDt{s} + v_i \dpdv{\sigma'_{ij}}{x_j}
	\mytag{3}
\]
を得る.\ajMaru{3}は\S~49で参照する.

\end{details}

さて,\eqref{eq6.1:理想流体のエネルギー流保存則(微分形)}を体積積分してGaussの発散定理を用いると
\begin{equation}\label{eq6.2:理想流体のエネルギー流保存則(積分形)}
    \pdv{t} \int_V\pqty{ \frac{1}{2}\rho v^2 + \rho\varepsilon} \dV
    = -\int_V \Div\bqty{\rho\v\pqty{\frac{1}{2}v^2 + h}} \dV
    = -\int_S \rho\v \pqty{\frac{1}{2}v^2 + h} \cdot \dS.
\end{equation}
左辺は体積中の全エネルギーの時間変化率を表しているから,右辺は,単位時間にこの体積から流れ出すエネルギーである.
ベクトル
\begin{equation}\label{eq6.3:エネルギーフラックス密度}
    \rho\v \pqty{\frac{1}{2}v^2 + h}
\end{equation}
は\emph{エネルギーフラックス密度}と呼ぶことができる.
その大きさは,$\v$に垂直な単位面積を単位時間に通過するエネルギーに等しい.

\eqref{eq6.2:理想流体のエネルギー流保存則(積分形)}の物理的意味を考えるために,次のように変形する;
\[
    \pdv{t} \int_V\pqty{ \frac{1}{2}\rho v^2 + \rho\varepsilon} \dV
    = -\int_S \rho\v \pqty{\frac{1}{2}v^2 + \varepsilon} \cdot \dS -\int_S p\v \cdot \dS.
\]
右辺第1項は,流体の運ぶ全エネルギー(運動エネルギーと内部エネルギー)を表す.
第2項は,体積内の流体が周囲からされた仕事を表す.




\BackToTheToc