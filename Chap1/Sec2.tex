\section{Euler方程式}\label{sec:2}
\subsection*{Euler方程式}
流体中の体積に働く力は,(エネルギーの散逸過程,つまり粘性や熱伝導が無視できる\emph{理想流体}の場合には)境界面上の圧力の積分
\[
    - \int p \,\dS = - \int \Grad p \dV
\]
に等しい.
\begin{details}
\nazenara
発散定理
$\displaystyle \int\vec{A}\cdot\dS = \int\Div\vec{A} \dV$
で$\vec{A}=p\vec{c}$($\vec{c}$は定ベクトル)とおけば
\[
    \int p\vec{c}\cdot\dS = \int \vec{c} \cdot \Grad p \dV
    \qquad\yueni \vec{c}\cdot\pqty{ \int p\,\dS - \int\Grad p \dV } = 0 .
\]
任意の$\vec{c}$についてこの式が成り立つためには,$()$内は$\vec{0}$でなければならない.
\end{details}
\noindent
つまり,単位体積の流体は周囲の流体から$-\Grad p$の力(圧力傾度力)を受ける.
この単位体積の運動方程式は
\begin{equation}\label{eq2.1:Euler方程式(Lagrange的)}
    \rho\DDt{\v} = -\Grad p .
\end{equation}

ここで微分$\dDDt{\v}$は,固定点での流速変化ではなく,流体粒子の速度変化を表している($D/Dt$をLagrange微分,物質微分などという).これを固定点での微分に書き直そう.
時間$\D t$の間の流体粒子の速度変化$\D\v$は,
\begin{itemize}
    \item 固定点における変化
    \item $\D t$の間に位置が$\D \vec{r}$ずれたことによる変化
\end{itemize}
の和であるから,
\[
    \D\v = \pdv{\v}{t} \D t + \pqty{ \pdv{\v}{x} \D x + \pdv{\v}{y} \D y + \pdv{\v}{z} \D z } = \pdv{\v}{t} \D t + (\D\vec{r}\cdot\Grad)\v.
\]
よって,
\begin{equation}\label{eq2.2:速度のLagrange微分}
    \DDt{\v} = \pdv{\v}{t} + (\v\cdot\Grad)\v.
\end{equation}
一般に
\[
    \DDt{} = \pdv{t} + \v\cdot\Grad
\]
となる.右辺第1項は固定点での時間変化,第2項は「移流項」である.

\eqref{eq2.2:速度のLagrange微分}を\eqref{eq2.1:Euler方程式(Lagrange的)}へ代入し
\begin{equation}\label{eq2.3:Euler方程式}
    \pdv{\v}{t} + (\v\cdot\Grad)\v = -\frac{1}{\rho} \Grad p .
\end{equation}
これが求める運動方程式で,\emph{Euler方程式}と呼ばれる.
もし流体が一様重力場中にあるなら,単位体積では力$\rho\vec{g}$が働くから,\eqref{eq2.3:Euler方程式}は
\begin{equation}\label{eq2.4:一様重力場中のEuler方程式}
    \pdv{\v}{t} + (\v\cdot\Grad)\v = -\frac{1}{\rho} \Grad p + \vec{g}
\end{equation}
となる.
ここで$\vec{g}$は重力加速度ベクトルで,一般には単位質量あたりの外力$\vec{K}$である.


\subsection*{エネルギー方程式(断熱の式)}
理想流体では流体内の熱輸送がなく,断熱的である.つまり,流体粒子の(単位質量あたり
\footnote{日本語版では「単位体積あたり」となっているが,誤植だろう.日本語版には,他にも同様のミスがある.}
の)エントロピー$s$は
\begin{equation}
    \DDt{s} = 0
\end{equation}
または
\begin{equation}\label{eq2.6:展開した理想流体の断熱の式}
    \pdv{s}{t} + \v\cdot\Grad s = 0
\end{equation}
を満たす.\eqref{eq1.1:連続の式}と\eqref{eq2.6:展開した理想流体の断熱の式}より
\begin{align}
    \pdv{(\rho s)}{t} + \Div(\rho s\v) &= s \pdv{\rho}{t} + \rho \pdv{s}{t} + \rho\v\cdot\Grad{s} + s \Div(\rho\v) \notag \\
    &= s \cancel{\pqty{\pdv{\rho}{t} + \Div(\rho\v)}} + \rho \cancel{\pqty{\pdv{s}{t} + \v\cdot\Grad s}} = 0 
\end{align}
となり,エントロピーに対する連続の式が得られる.
$\rho s\v$は「エントロピーフラックス密度」とでも呼ぶべき量である.

もし,ある時刻に流体の全体積にわたってエントロピーが一様なら,その後もエントロピーは一定・一様である.
この場合の断熱の式は
\begin{equation}
    s = \const
\end{equation}
という簡単な形になる.以下では,このような運動(\emph{等エントロピー運動})を扱う.

等エントロピー運動では,\eqref{eq2.3:Euler方程式}を違った形に書くことができる.
$h = \varepsilon + pV = \varepsilon + \dfrac{p}{\rho}$
を,単位質量あたりの\emph{エンタルピー}とすると($\varepsilon$は単位質量あたりの内部エネルギー,$V=1/\rho$は比体積)
\[
    dh = Tds + Vdp = \frac{1}{\rho} dp .
\]
よって$\dfrac{1}{\rho}\Grad p = \Grad h$となり,\eqref{eq2.3:Euler方程式}は
\begin{equation}\label{eq2.9:hを用いたEuler方程式}
    \pdv{\v}{t} + (\v\cdot\Grad)\v = -\Grad h .
\end{equation}
ここからさらに,$h$を消去することができる.
\[
    \frac{1}{2} \Grad v^2 = \v \times \Rot \v + (\v\cdot\Grad)\v
\]
であるから
\begin{details}
\nazenara
$\varepsilon_{ijk}$をEddingtonのイプシロン(Levi-Civitaの記号)とする.
すなわち,$(ijk)$が(123)の偶置換ならば$+1$,奇置換ならば$-1$,それ以外なら0となる量で,$\vec{A}\times\vec{B}$の$i$成分は$\varepsilon_{ijk}A_jB_k$を表される.
恒等式$\varepsilon_{ijk}\varepsilon_{lmk} = \delta_{il}\delta_{jm}-\delta_{im}\delta_{jl}$に注意すると
\begin{align*}
    \pqty{\v \times \Rot \v}_i &= \varepsilon_{ijk}v_j \pqty{\varepsilon_{klm}\pdv{v_m}{x_l}} = (\delta_{il}\delta_{jm}-\delta_{im}\delta_{jl}) v_j \pdv{v_m}{x_l} \\
    &= v_j \pdv{v_j}{x_i} - v_j \pdv{v_i}{x_j} = \frac{1}{2}\pdv{x_i}(v_jv_j) - v_j \pdv{v_i}{x_j} \\
    &= \pqty{ \frac{1}{2} \Grad v^2 - (\v\cdot\Grad)\v }_i
\end{align*}
となる.
\end{details}
%
\noindent
\eqref{eq2.9:hを用いたEuler方程式}は
\begin{equation}\label{eq2.10:rotとhを用いたEuler方程式}
    \pdv{\v}{t} = \v \times \Rot \v - \Grad\pqty{\frac{1}{2}v^2+h} .
\end{equation}
両辺のrotを取ると,$\Rot\Grad f=0$であるから
\begin{equation}\label{eq2.11:理想流体の渦度方程式}
    \pdv{t}(\Rot\v) = \Rot(\v \times \Rot \v) .
\end{equation}
これは速度だけを含んでいる.
$\vec{\omega}\equiv\Rot\v$で\emph{渦度}を定義すると
\[
    \pdv{\vec{\omega}}{t} = \Rot(\v \times \vec{\omega}) .
\]
これは\emph{渦度方程式}と呼ばれる.


\subsection*{運動方程式の境界条件}
理想流体の境界条件は,流体が固体の表面を通り抜けないことである.
固体表面に垂直な速度成分を$v_n$とすると,
\begin{itemize}
    \item 固体表面が静止しているとき
    \begin{equation} v_n = 0 . \end{equation}    
    \item 固体表面が動くときは,$v_n$は固体表面の速度の垂直成分に等しい.
\end{itemize}
混じり合わない2つの流体の境界では,圧力が等しく($p^{(1)}=p^{(2)}$),
境界面に垂直な流速が境界面の速度の垂直成分に等しい(${v_n}^{(1)}={v_n}^{(2)}={v_n}^{(\textrm{boundary})}$).

% \S~1で述べたように,流体を記述する変数は$\v,\rho,p$の5つであり,方程式も5本必要である.理想流体では連続の式,Euler方程式,断熱の式である.

\begin{mondai}{}{}
ある時刻$t=t_0$での流体粒子の座標を$a$とする(これはLagrange座標と呼ばれる).
$a,t$を用いて,理想流体の1次元運動の方程式を書け.
\end{mondai}

\begin{kaitou}
このようなLagrange的記述のもとでは,任意の流体粒子の座標$x$は$a,t$の関数とみなされる;$x=x(a,t).$
$t=t_0$での密度分布を$\rho_0(a)$とすると,連続の式は
\[
    \rho dx = \rho_0 da
    \qquad\yueni \rho \PDV{x}{a}{t} = \rho_0 .
\]
流体粒子の速度は$v = \dPDV{x}{t}{a}$であり,その偏微分$\dPDV{v}{t}{a}$は加速度を表す.
粒子に働く力を考え,Euler方程式は
\[
    \rho_0 \PDV{v}{t}{a} = - \PDV{p}{a}{t} .
\]
断熱の式は
\[
    \PDV{s}{t}{a} = 0 .
\]
\end{kaitou}




\BackToTheToc