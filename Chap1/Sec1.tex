\section{連続の式}
\subsection*{流体力学の基礎的概念}
流体力学は流体(液体と気体)のマクロな現象を扱う.流体は連続媒質とみなされる.
よく流体の微小な体積要素を「流体粒子」というが,この中にも十分多くの分子が含まれている.

式で書けば,考えている流体の大きさを$L$,流体粒子の大きさを$l$,分子の平均自由行程を$\lambda$として
\[
    \lambda \ll l \ll L
\]
となる($\mathrm{Kn} \equiv \lambda/L \ll 1$はKnudsen数と呼ばれる).
「流体粒子の変位」と言ったときも,各分子の変位ではなく,微小な体積要素そのものの変位を意味している(体積要素内の分子たちの重心の変位,と言ってもよいかもしれない).


\subsection*{流体を記述する変数}
流体の運動は,速度$\v(x,y,z,t)$を指定すれば決まる.
また,熱力学的な状態は,2つの変数を指定してしまえば,残りの変数は状態方程式から導くことができる.流体力学では,圧力$p(x,y,z,t)$と密度$\rho(x,y,z,t)$がよく用いられる.
以上の5変数が指定されれば,流体が完全に記述される
\footnote{そのためには5本の支配方程式が必要であり,それは連続の式,運動方程式(3成分),エネルギー方程式である.}.

これらの変数は座標$(x,y,z)$と時間$t$の関数であるが,$(x,y,z)$は空間上に固定された点の座標を表す(Euler的記述)のであって,特定の粒子に乗った座標(Lagrange的記述)ではないことに注意.

\begin{details}
流体力学で最大限に用いる平衡熱力学は,文字通り平衡状態を対象としている.
流体は常に「流れて」いるので,平衡状態とは思えないが,熱力学を適用していいのだろうか?


この疑問にマクロな視点から答えるとすれば,それは「流体全体としては平衡状態でなくても,各部分(流体粒子)は\emph{局所的平衡}にあると考えられるので,熱力学を適用してよい」となるだろうか.
流体力学でのあらゆる変数(速度,密度,圧力,温度など)は,各流体粒子に対して定義されたマクロな変数である.
すなわち,流体粒子程度の距離ではこれらの変数はほとんど変化しないと考えて,各流体粒子に平衡熱力学を適用する.



ミクロな視点で考えるなら,流体力学の方程式はいずれもBoltzmann方程式のモーメント(Boltzmann方程式に質量や運動量,エネルギーをかけて積分したもの)として得られることを思い出す必要がある.
分布関数$f$は一般には複雑な形をとるが,局所的平衡状態における関数$f_0$で近似して計算すれば,これは散逸過程を無視しているので理想流体に対応する.
また,$f_0$からのずれを1次まで考慮して計算すれば,これは線形の粘性流体に対応する(この場合の熱伝導率や粘性も分布関数から計算される).
これらの議論は『物理的運動学』\S~5--8に詳しい.


いずれにせよ,
「理想流体の章では散逸過程のない局所的平衡状態を仮定し,粘性流体の章では散逸過程を1次近似している.
それ以上の高次項(非線形効果)についてはこの巻の範囲外(運動学の範囲)である.」
と考えておけば,見通しよく読み進められると思う.


% もはや不要のリンクだが一応.
% https://www.kurims.kyoto-u.ac.jp/~kenkyubu/kokai-koza/R1-yamada.pdf
% https://physics.stackexchange.com/questions/67966/fluids-in-thermodynamic-equlibrium

\end{details}


\subsection*{連続の式}
まず,物質の質量保存則を定式化しよう.
空間内のある体積$V_0$を考える.この中の流体の質量は
\[
    \int_{V_0} \rho \dV .
\]
次に,$V_0$表面の面積要素(ベクトル)を$\dS$
%\footnote{Landauで用いられている$d\vec{f}$ではなく,より馴染みのある$\dS$を使うことにする.}
とする($\dS$の向きは外向き法線ベクトルと一致し,大きさは要素の面積と等しい).
$\dS$を通って,単位時間に流れ出る流体の質量は$\rho\v\cdot \dS$であるから,流れ出る全質量は
\[
    \int_{S_0} \rho\v\cdot \dS .
\]
これは,単位時間の質量の減少率に等しいから
\begin{equation}
    \pdv{t} \int_{V_0} \rho \dV = - \int_{S_0} \rho\v\cdot \dS .
\end{equation}
Gaussの発散定理を用いると
\[
    \int_{V_0} \bqty{ \pdv{\rho}{t} + \Div(\rho\v) } \dV = 0 .
\]
これが任意の$V_0$について成り立つから
\begin{equation}\label{eq1.1:連続の式}
    \pdv{\rho}{t} + \Div(\rho\v) = 0 .
\end{equation}
これが\emph{連続の式}である.
$\Div(\rho\v) = \v\cdot \Grad \rho + \rho \Div\v$であるから,
% \begin{details}
% \[
%     \pdv{(\rho v_i)}{x_i} = \pdv{\rho}{x_i}v_i + \rho \pdv{v_i}{x_i}
% \]
% \end{details}
\begin{equation}
    \pdv{\rho}{t} + \v\cdot \Grad \rho + \rho \Div\v = 0,
\end{equation}
または
\[
    \DDt{\rho} + \rho \Div\v = 0
\]
とも書ける($D/Dt$の定義は\S~\ref{sec:2}参照)
\footnote{流体中の微小体積$\delta V$の相対的な時間変化は
$\dfrac{1}{\delta V}\dDDt{\delta V} = \Div\v$であるから,連続の式は
\[
    \DDt{} (\rho\delta V) = 0
\]
とも書ける.これがLagrange的記述での質量保存則を表していることは明らかである.}.

ベクトル
\begin{equation}
    \vb*{j} = \rho\v
\end{equation}
は\emph{質量フラックス密度}と呼ばれ,大きさは,$\v$に垂直な単位面積を単位時間に通過する流体の質量に等しい.



\BackToTheToc