\section{音波の反射と屈折}

音波が異なる2つの流体の境界に入射すると,反射や屈折を起こす.
1つ目の媒質中の運動は,入射波と\emph{反射波}の組み合わせであるが,2つ目の媒質には\emph{屈折波}しか存在しない.
これら3つの関係は,境界条件により決定される.


2つの媒質の平面境界($yz$平面とする)における,単色の縦波の反射と屈折について考えよう.
3つの波は同じ振動数$\omega$と同じ波数ベクトルの成分$k_y,k_z$を持つが,境界に垂直な波数$k_x$は異なる値を持つ.
その理由は以下の通りである.
均質な無限媒質の場合,一定の$\vec{k},\omega$を持つ単色波は運動方程式を満たす.
境界面の存在は,単に境界条件を加えるに過ぎない.
この場合の境界は$x=0$であり,$t,y,z$には依存しない.
よって,$t,y,z$に対する解の依存性は全ての時間や座標に対して等しい.
よって$\omega,k_y,k_z$は入射波と同じ値をとる.


この結果からただちに,反射波と屈折波の伝播方向を与える関係式を導くことができる.
入射波を含む平面を$xy$平面とすると,入射波では$k_z=0$であり,反射波と屈折波でも同様のことが成り立つ必要がある.
ゆえに,3つの波の伝播方向は同一平面上にある.


さて,波の伝播方向が$x$軸となす角を$\theta$としよう.
$k_y = \dfrac{\omega}{c}\sin\theta$であり,これが入射波と反射波で等しいことから
\begin{equation}
    \theta_1 = \theta_1'
\end{equation}
となる.
つまり入射角$\theta_1$と反射角$\theta_1'$は等しい.
次に入射波と屈折波で$k_y$が等しいことから
\begin{equation}\label{eq66.2:スネルの法則}
    \frac{\omega}{c_1}\sin\theta_1 = \frac{\omega}{c_2}\sin\theta_2
    \qquad\yueni \frac{\sin\theta_1}{\sin\theta_2} = \frac{c_1}{c_2}
\end{equation}
を得る($c_1,c_2は2つの媒質の音速$).
これは\emph{Snellの法則}と呼ばれる.


3つの波の強さの間に成り立つ定量的な関係を得るために,入射波,反射波,屈折波の速度ポテンシャルを次のように書く.
\begin{align*}
    \phi_1 &= A_1 \exp\left[ i\omega \left( \frac{x}{c_1}\cos\theta_1 + \frac{y}{c_1}\sin\theta_1 -t \right) \right] \\
    \phi_1' &= A_1' \exp\left[ i\omega \left( -\frac{x}{c_1}\cos\theta_1 + \frac{y}{c_1}\sin\theta_1 -t \right) \right] \\
    \phi_2 &= A_2 \exp\left[ i\omega \left( \frac{x}{c_2}\cos\theta_2 + \frac{y}{c_2}\sin\theta_2 -t \right) \right]     
\end{align*}
境界では圧力$p'=-\rho\dpdv{\phi}{t}$と法線速度$v_x=\dpdv{\phi}{x}$が等しいから
\[
    \rho_1(A_1+A_1') = \rho_2A_2,
    \mytag{1}    
\]
\[
    \frac{\cos\theta_1}{c_1}(A_1-A_1') = \frac{\cos\theta_2}{c_2} A_2.
    \mytag{2}
\]
音波の\emph{反射係数}$R$は,反射波と入射波のエネルギーフラックス密度の(時間)平均の比として定義される.
平面波のエネルギーフラックスは$c\rho v^2$であるから,
$R = \dfrac{c_1\rho_1 \overline{{v_1'}^2}}{c_1\rho_1 \overline{{v_1}^2}} = \dfrac{|A_1'|^2}{|A_1|^2}$となる.
\begin{details}
$\text{\ajMaru{1}} \divisionsymbol \text{\ajMaru{2}}$より
\[
    \frac{\rho_1c_1}{\cos\theta_1} \cdot \frac{A_1+A_1'}{A_1-A_1} = \frac{\rho_2c_2}{\cos\theta_2}
\]
\[
    \frac{A_1+A_1'}{A_1-A_1'} = \frac{1+\frac{A_1'}{A_1}}{1-\frac{A_1'}{A_1}}
    = \frac{\rho_2c_2\cos\theta_1}{\rho_1c_1\cos\theta_2}
    \overset{\Eqref{gray}{eq66.2:スネルの法則}}{=}
    \frac{\rho_2\tan\theta_2}{\rho_1\tan\theta_1}
\]
\[
    1+\frac{A_1'}{A_1} = \frac{\rho_2\tan\theta_2}{\rho_1\tan\theta_1} \left( 1-\frac{A_1'}{A_1} \right)
\]
\[
    \left( 1 + \frac{\rho_2\tan\theta_2}{\rho_1\tan\theta_1} \right)\frac{A_1'}{A_1}
    = \frac{\rho_2\tan\theta_2}{\rho_1\tan\theta_1} -1
\]
\[
    \frac{A_1'}{A_1} = \frac{\rho_2\tan\theta_2-\rho_1\tan\theta_1}{\rho_2\tan\theta_2+\rho_1\tan\theta_1}  
\]
\end{details}
よって
\begin{equation}
    R = \left( \frac{\rho_2\tan\theta_2-\rho_1\tan\theta_1}{\rho_2\tan\theta_2+\rho_1\tan\theta_1} \right)^2
\end{equation}
となる.
\eqref{eq66.2:スネルの法則}を用いて,$\theta_2$を$\theta_1$で表そう.
\[
    \frac{\tan\theta_2}{\tan\theta_1} = \frac{\sin\theta_2}{\sin\theta_1} \frac{\cos\theta_1}{\cos\theta_2}
    = \frac{c_2}{c_1}\frac{\cos\theta_1}{\sqrt{1-\sin^2\theta_2}}
    = \frac{c_2}{c_1}\frac{\cos\theta_1}{\sqrt{1-\left( \frac{c_2}{c_1}\sin\theta_1 \right)^2}}
    = \frac{c_2\cos\theta_1}{\sqrt{{c_1}^2-{c_2}^2\sin^2\theta_1}}
\]
であるから
\begin{equation}
    R = \left( \frac{\rho_2c_2\cos\theta_1 - \rho_1\sqrt{{c_1}^2-{c_2}^2\sin^2\theta_1}}{\rho_2c_2\cos\theta_1 + \rho_1\sqrt{{c_1}^2-{c_2}^2\sin^2\theta_1}} \right)^2
\end{equation}
となる.


特に垂直入射($\theta_1=0$)のときは
\begin{equation}
    R = \left( \frac{\rho_2c_2 - \rho_1c_1}{\rho_2c_2 + \rho_1c_1} \right)^2
\end{equation}
となる.



最後に,$R=0$となるのはどのような場合か調べよう.
\[
    {\rho_2}^2{c_2}^2\cos^2\theta_1 = {\rho_1}^2 ({c_1}^2 - {c_2}^2 \sin^2\theta_1)
    = {\rho_1}^2 ({c_1}^2 - {c_2}^2) + {\rho_1}^2{c_2}^2\cos^2\theta_1
\]
\[
    ({\rho_2}^2-{\rho_1}^2){c_2}^2\cos^2\theta_1 = {\rho_1}^2 ({c_1}^2 - {c_2}^2)
\]
\begin{equation}\label{eq66.6:音波が全屈折する条件}
    \tan^2\theta_1 = \frac{1}{\cos^2\theta_1} -1
    = \frac{({\rho_2}^2-{\rho_1}^2){c_2}^2}{{\rho_1}^2 ({c_1}^2 - {c_2}^2)} -1
    = \frac{{\rho_2}^2{c_2}^2 - {\rho_1}^2{c_1}^2}{{\rho_1}^2 ({c_1}^2 - {c_2}^2)}
\end{equation}
のときである.
つまり,\eqref{eq66.6:音波が全屈折する条件}が成り立つとき,波は全て屈折する.
これが起こるのは,$c_1>c_2$かつ$\rho_2c_2>\rho_1c_1$が成り立つとき,または$c_1<c_2$かつ$\rho_2c_2<\rho_1c_1$が成り立つときである.





問題は省略する.


\BackToTheToc