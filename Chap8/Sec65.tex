\section{音波のエネルギーと運動量}

音波のエネルギーの表式を導こう.
一般に,単位体積当たりの流体のエネルギーは$\rho\varepsilon+\dfrac{1}{2}\rho v^2$である.
これを2次のオーダーまで取ることにすると
\[
    \rho_0\varepsilon_0 + \rho' \eval{\pdv{(\rho\varepsilon)}{\rho}}_{\rho=\rho_0} + \frac{1}{2} {\rho'}^2 \eval{\pdv[2]{(\rho\varepsilon)}{\rho}}_{\rho=\rho_0}
    +\frac{1}{2}\rho_0 v^2
\]
となる(音波は断熱的であるから,偏微分はエントロピー一定のもとで行うことにする.なお$\dfrac{1}{2}\rho' v^2$は3次のオーダーである).
\[
    d(\rho\varepsilon) = \varepsilon d\rho + \rho d\varepsilon
    = \varepsilon d\rho + \rho\left(Tds+\frac{p}{\rho^2}d\rho\right)
    = \left( \varepsilon+\frac{p}{\rho} \right) d\rho + \rho T ds
    = h d\rho + \rho T ds
\]
より
\[
    \eval{\pdv{(\rho\varepsilon)}{\rho}}_{\rho=\rho_0} = h_0, \quad
    \eval{\pdv[2]{(\rho\varepsilon)}{\rho}}_{\rho=\rho_0} = \eval{\pdv{h}{\rho}}_{\rho=\rho_0}
    = \eval{\pdv{h}{p}\pdv{p}{\rho}}_{\rho=\rho_0} = \frac{c^2}{\rho_0}
\]
であるから,単位体積当たりのエネルギーは
\[
    \rho_0\varepsilon_0 + h_0 \rho' + \frac{c^2{\rho'}^2}{2\rho_0} + \frac{1}{2} \rho_0 v^2
\]
となる.
第1項$\rho_0\varepsilon_0$は,流体が静止しているときのエネルギーであり,音波に関係しない.
第2項$ h_0 \rho'$は,流体の質量変化に伴うエネルギーの変化である.
この項は,全体積で積分すると消える($\because \int\rho' dV=0$).
よって音波による流体のエネルギー変化量は積分
\[
    \int \left( \frac{1}{2} \rho_0 v^2 + \frac{c^2{\rho'}^2}{2\rho_0} \right) dV
\]
で与えられる.
被積分関数
\begin{equation}
    E = \frac{1}{2} \rho_0 v^2 + \frac{c^2{\rho'}^2}{2\rho_0}
\end{equation}
は音のエネルギー密度とみなすことができる.

進行平面波の場合,上式は簡単になる.
$\rho' = \dfrac{v \rho_0}{c}$を代入して
\begin{equation}
    E = \rho_0 v^2
\end{equation}
となる.
一般にはこの式は成り立たないが,全エネルギーの時間平均については同様の式を得ることができる.
よく知られた力学の一般理論により,微小振動を行っている系の全平均ポテンシャルエネルギーは,全平均運動エネルギーに等しい.
よって
\begin{equation}
    \int \overline{E} \dV = \int \rho_0 \overline{v^2} \dV
\end{equation}
が成り立つ.


次に,音波が伝播している流体のある体積を考え,この体積を囲む閉曲面を通るエネルギーフラックスを求めよう.
流体のエネルギーフラックス密度は,式(6.3)より$\rho\v\left( \dfrac{1}{2}v^2+h \right)$である.
今の場合,$v^2$の項は3次のオーダーであるから無視することができる.
よって$\rho(h_0+h')\v$となる.
エンタルピーの微小変化は$h'=\dPDV{h}{p}{s}p'=\dfrac{p'}{\rho}$であるから$\rho h_0\v + p'\v$となり,表面を通る全エネルギーフラックスは
\[
    \int \left( \rho h_0\v + p'\v \right)\cdot\dS
\]
となる.
第1項は,考えている体積中の流体の質量が変化したことによるエネルギーフラックスである.
我々はエネルギー密度を求める際,対応する項$h_0 \rho'$を省いたから,ここでも省くことにする.
よって全エネルギーフラックスは単に
\[
    \int p' \v \cdot \dS
\]
となる.
音波のエネルギーフラックス密度は,ベクトル
\begin{equation}
    \vec{q} = p' \v
\end{equation}
により表される.
エネルギー密度とフラックス密度は,エネルギー保存則
\begin{equation}
    \pdv{E}{t} + \Div \vec{q} = 0
\end{equation}
により結ばれている.
\begin{details}
\begin{align*}
    \text{(左辺)} 
    &= \pdv{t} \left( \frac{1}{2} \rho_0 v^2 + \frac{c^2{\rho'}^2}{2\rho_0} \right) + \Div(p'\v) \\
    &= \rho_0 \v \cdot \pdv{\v}{t} + \frac{c^2}{\rho_0}\rho'\pdv{\rho'}{t} + \v \cdot\Grad p' + p' \Div\v \\
    & \gyoukan{($p'=c^2\rho'$に注意する)} \\
    &= \v \cdot \left( \rho_0 \pdv{\v}{t} + \Grad p' \right) + \frac{p'}{\rho_0} \left( \pdv{\rho'}{t} + \rho_0 \Div\v \right) = 0
\end{align*}
\end{details}


進行平面波では,圧力変化は速度と$p'=c\rho_0v$の関係にある($v$の符号は適当に決める).
波の伝播方向の単位ベクトル$\vec{n}$を導入すれば
\begin{equation}
    \vec{q} = \rho_0 c v^2 \vec{n} = cE\vec{n}
\end{equation}
となる.
よって,期待されるように,平面音波のエネルギーフラックス密度は,エネルギー密度に音速をかけたものに等しい.




\BackToTheToc