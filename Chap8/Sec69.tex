\section{固有振動}

これまで,我々は無限媒質中の振動のみを議論してきた.
そして,そのような媒質中では任意の振動数を持った波が伝播できることを学んだ.

しかし,有限の大きさを持つ容器内の流体を考える場合,状況はかなり異なる.
運動方程式(波動方程式)自体は変わらないが,固体壁や自由表面で満たされるべき境界条件を加えなければならない
\footnote{以下では\emph{自由振動},つまり変動する外力なしに生じる振動について考える.
外力の結果生じる振動は\emph{強制振動}と呼ばれる.}.


有限の流体に対する運動方程式は,全ての周波数に対して境界条件を満たす解を持つわけではなく,限られた振動数でのみ振動が起こる.
このような振動数は,考えている容器内の流体の\emph{固有振動数}と呼ばれる.


固有振動数の実際の値は,容器の大きさや形状に依存する.
そしてどのような場合にも,固有振動数は無限個存在する.
これらを求めるためには,適切な境界条件のもとで運動方程式を調べる必要がある.


最小の固有振動数$\omega_1$のオーダーは,次元解析から直ちに得ることができる.
問題に現れる,長さの次元を持つパラメータは物体(容器)の大きさ$l$のみである.
よって最小の固有振動数に対応する波長$\lambda_1$は$l$のオーダーであり,
$\omega_1$は音速をこの波長で割ることにより得られる.すなわち
\begin{equation}
    \lambda_1 \sim l, \quad 
    \omega_1 \sim \frac{c}{l}.
\end{equation}


固有振動の性質を調べよう.
時間について周期的な速度ポテンシャル$\phi = \phi_0(x,y,z) e^{-i\omega t}$を仮定すると,
$\phi_0$に関するHelmholtz方程式
\begin{equation}
    \Laplacian \phi_0 + \frac{\omega^2}{c^2} \phi_0 = 0
\end{equation}
が得られる.


境界条件のない無限媒質では,この方程式は実数と虚数両方の解を持つ.
特に$e^{i \vec{k}\cdot\vec{r}}$に比例する解をもち,この場合の速度ポテンシャルは
\[
    \phi \propto \exp[ i(\vec{k}\cdot\vec{r}-\omega t) ]
\]
となり,ある速度で伝播する進行波を表す.


しかし,有限の体積を持つ媒質では,複素数の解は一般に存在しない.
これは以下のようにして示すことができる.
$\phi_0$の満たすべき方程式は実であり,境界条件も実である.
よって$\phi_0$が解ならその複素共役$\phi_0^*$も解である.
一方,与えられた境界条件を満たす解は,定数係数を除いて一意に定まるから,$\phi_0^* = A\phi_0$でなければならない.
両辺の複素共役を取ると$\phi_0 = A^* \phi_0^*$となり,$AA^*=1$すなわち$|A|=1$を得る.
よって$\phi_0$は,実数$f,\alpha$を用いて$\phi_0 = f(x,y,z)e^{-i\alpha}$と書ける.
$\phi = f e^{-i(\omega t + \alpha)}$の実部を取れば
\begin{equation}\label{eq69.3:定在波のポテンシャルの一般形}
    \phi = f(x,y,z) \cos(\omega t + \alpha)
\end{equation}
を得る.
よって速度ポテンシャルは,座標のある関数と,時間の周期関数の積である.


この解は,進行波とは完全に異なる性質を持つ.
進行波では,波長の整数倍離れた点を除けば,異なる点での位相$\vec{k}\cdot\vec{r}-\omega t + \alpha$は各時刻で異なる値をとる.
一方,\eqref{eq69.3:定在波のポテンシャルの一般形}で表されるような波では,全ての点が任意の時刻において同じ位相$\omega t + \alpha$で振動している.
このような波は明らかに``伝播''しておらず,\emph{定在波}と呼ばれる.
よって固有振動は定在波である.


全ての量が1つの座標(例えば$x$)と時間の関数であるような,定常平面波を考えよう.
Helmholtz方程式$\dpdv[2]{\phi_0}{x} + \dfrac{\omega^2}{c^2}\phi_0=0$の解を$\phi_0 = a \cos\left( \dfrac{\omega x}{c} + \beta \right)$の形に書くと
$\phi = a \cos(\omega t + \alpha) \cos\left( \dfrac{\omega x}{c} + \beta \right)$となる.
$x,t$の原点を適当に選ぶことにより,$\alpha,\beta$は0にすることができる.よって
\begin{equation}
    \phi = a \cos \omega t \cos \frac{\omega x}{c}
\end{equation}
となる.
速度と圧力は
\[
    v = \pdv{\phi}{x} = -\frac{\omega a}{c} \cos \omega t \sin \frac{\omega x}{c}, \quad
    p' = -\rho \pdv{\phi}{t} = \rho\omega a \sin \omega t \sin \frac{\omega x}{c}.
\]
距離$\dfrac{\pi c}{\omega} = \dfrac{1}{2}\lambda$ずつ離れた点
$x = 0,  \dfrac{\pi c}{\omega}, \dfrac{2\pi c}{\omega}, \ldots$では,速度は常に0である.
これらの点は速度の\emph{節}であるという.
これらの中間の点$x = \dfrac{\pi c}{2\omega}, \dfrac{3\pi c}{2\omega}, \ldots$では,
速度の振幅が最大となり,\emph{腹}と呼ばれる.
圧力の節と腹は,速度とは逆の位置にある.
すなわち,圧力の節は速度の腹であり,圧力の腹は速度の節である.


固有振動の興味深い例は,小さな開口部をもつ容器(共鳴器)内の気体の振動である.
$l$を容器の長さとすると,閉じた容器の最小の固有振動数は$c/l$のオーダーである.
しかし,小さな開口部があると,かなり小さな振動数をもつ,新しい固有振動が現れる.
その原因は以下の通りである:
容器の中と外で気体に圧力差があると,差をなくすために気体は容器を出入りする.
開口部が小さいため,気体の出入りはゆっくりと起こり,振動の周期は長く,対応する振動数は短くなる.






問題は省略する.


\BackToTheToc