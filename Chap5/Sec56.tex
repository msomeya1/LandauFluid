\section{自由対流}

\S~3で見たように,重力場中の流体が力学的平衡状態にあるなら,温度は高度のみに依存する:$T=T(z)$.
温度分布がこの条件を満たさず,他の座標の関数でもある場合には,流体は力学的平衡になることはできない.
さらに,もし$T=T(z)$であっても,温度が上空ほど低く,温度勾配がある値(断熱減率)を超える場合は,やはり力学的平衡にはならない(\S~4).

力学的平衡にならない場合には,流体を混合して温度を一様にしようとする流れが生じる.
重力場中でのこのような運動は\emph{自由対流}と呼ばれる.


この対流を記述する方程式を導こう.
流体は非圧縮であると仮定する.
このことは,流体中で圧力がほんのわずかしか変化せず,圧力による密度変化が無視できることを意味する.
例えば,気圧が高度によって変化する大気の場合,この仮定は高い大気柱を考えないことに相当する
(高い大気柱では,密度が高度によって大きく変化してしまう).
もちろん,流体が非一様に加熱されることで生じる密度変化は無視できない.
これが対流を駆動するからである.


温度を$T=T_0+T'$と書く.
$T_0$はある一定の平均温度,$T'$はそこからのずれであり,$T' \ll T_0$と仮定する.
同様に密度も$\rho=\rho_0+\rho'$と書く($\rho_0$は定数).
$T'$が微小量であるから,$\rho'$も微小量であり
\begin{equation}\label{eq56.1:Boussinesq近似での密度変化}
    \rho' = \PDV{\rho_0}{T}{p} T' = -\rho_0 \beta T'.
\end{equation}
ここで$\beta = -\dfrac{1}{\rho_0}\dPDV{\rho_0}{T}{p}>0$は流体の熱膨張率である.


次に圧力も$p=p_0+p'$を書けるが,$p_0$は定数でなく,温度と密度が$T_0,\rho_0$である静水圧平衡での値である.
\begin{equation}
    p_0 = \rho_0 \vec{g} \cdot \vec{r} + \const
    = -\rho_0 gz + \const
\end{equation}
ここで$z$は鉛直上向きに取られている.



\spade
高さ$h$の気体柱では,静水圧の変化は$\rho_0 gh$であり,これによる密度変化は$\rho_0 gh/c^2$のオーダーである
($c$は音速.(64.4)参照).
上述の条件によれば,この変化は密度自身に比べて無視できるだけでなく,
熱的変化\eqref{eq56.1:Boussinesq近似での密度変化}に比べても無視できなければならない.
よって
\begin{equation}
    \frac{gh}{c^2} \ll \beta \Theta
\end{equation}
でなければならない.
ここで$\Theta$は温度差の特徴的な値である.


さて,方程式を得るために,Navier-Stokes方程式
\[
    \pdv{\v}{t} + (\v\cdot\Grad) \v = - \frac{1}{\rho} \Grad p + \nu \Laplacian\v + \vec{g}
\]
を書きかえることから始めよう.
$p=p_0+p'$,$\rho=\rho_0+\rho'$を代入して微小量の1次まで取れば,圧力項は
\begin{align*}
    - \frac{1}{\rho} \Grad p
    &= - \frac{1}{\rho_0 (1+\rho'/\rho_0)} \Grad (p_0 + p') \\
    &\simeq - \frac{1}{\rho_0} \left( 1 - \frac{\rho'}{\rho_0} \right) \Grad (p_0 + p') \\
    &\simeq - \frac{1}{\rho} \Grad p_0 - \frac{1}{\rho_0} \Grad p' + \frac{\rho'}{{\rho_0}^2} \Grad p_0 \\
    &= \vec{g} - \frac{1}{\rho_0} \Grad p' - \beta T' \vec{g}
\end{align*}
となるから,Navier-Stokes方程式は
\begin{equation}\label{eq56.4:自由対流の方程式1}
    \pdv{\v}{t} + (\v\cdot\Grad) \v = - \frac{1}{\rho} \Grad p' + \nu \Laplacian\v -\beta T' \vec{g}
\end{equation}
となる(以下$\rho_0$の添字を省く).
自由対流では,熱伝導方程式\eqref{eq50.2:非圧縮性流体での熱輸送の式}で粘性項が小さいから落とすことができ
\begin{equation}\label{eq56.5:自由対流の方程式2}
    \pdv{T'}{t} + \v \cdot \Grad T' = \chi \Laplacian T'
\end{equation}
となる.
式\eqref{eq56.4:自由対流の方程式1}と\eqref{eq56.5:自由対流の方程式2}及び連続の式$\Div\v=0$が,
自由対流を記述する方程式系をなす(A. Oberbeck 1879, J. Boussinesq 1903).




定常流では,対流の方程式は次のようになる.
\begin{align}
    (\v\cdot\Grad) \v &= -\Grad \left( \frac{p'}{\rho} \right) - \beta T' \vec{g} + \nu \Laplacian\v \\
    \v \cdot \Grad T' &= \chi \Laplacian T' \\
    \Div \v &= 0
\end{align}
未知数$\v, p'/\rho, T'$を含むこの方程式系は,3つのパラメータ$\nu, \chi, \beta g$を含む.
さらに,解は特徴的な長さ$h$と温度差$\Theta$も含むだろう
(特徴的な速度は存在しない.というのは,流れは外力によるものではなく,非一様に加熱されたことで生じるからである).
これらの量から2つの無次元量を作ることができる.
普通使われる量は
Prandtl数$\mathrm{P}=\dfrac{\nu}{\chi}$と
\emph{Rayleigh数}
\begin{equation}
    \mathscr{R} = \frac{\beta g \Theta h^3}{\nu\chi}
\end{equation}
である.
\emph{Grashof数}
\[
    \mathrm{G} = \frac{\beta g \Theta h^3}{\nu^2} = \frac{\mathscr{R}}{\mathrm{P}}
\]
もよく使われる.




Prandtl数は流体の物性のみで決まったが,Rayleigh数は対流そのものの性質を決める.



自由対流での相似則は
\begin{equation}
    \v = \frac{\nu}{h} \vec{f} \left( \frac{\vec{r}}{h}, \mathscr{R}, \mathrm{P} \right),
    T' = \Theta f \left( \frac{\vec{r}}{h}, \mathscr{R}, \mathrm{P} \right)
\end{equation}
となる.
Rayleigh数とPrandtl数が等しい2つの流れは相似である.
また,重力下における対流熱輸送はやはりNusselt数で記述されるが,
この場合$\mathrm{N}$は$\mathscr{R}, \mathrm{P}$のみの関数である.



対流は層流/乱流のどちらのタイプもありうる.
乱流の開始はRayleigh数により決まり,$\mathscr{R}$が十分大きいとき対流は乱流になる.





問題は省略する.



\BackToTheToc