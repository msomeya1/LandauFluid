\section{熱輸送の相似則}

流体中の熱輸送の過程は,流体が運動するために,固体中の問題よりも複雑である.
非一様に加熱された流体の運動は\emph{対流}と呼ばれる.



ここでは,流体の物性値が(温度によらず)一定と仮定できるほど,流体中の温度差が小さいと仮定する.
一方で,粘性散逸によって生じる温度差を無視できるほど,温度差が大きいとしよう.
このとき\eqref{eq50.2:非圧縮性流体での熱輸送の式}で粘性項を除くことができるから
\begin{equation}
    \pdv{T}{t} + \v \cdot \Grad T = \chi \Laplacian T 
\end{equation}
となる.
これと連続の式,Navier-Stokes方程式により,今考えている対流が完全に決定する.


以下では定常な対流のみを扱う(そのような対流が維持されるためには,流体に接している熱源が必要である).
時間微分項を落として
\begin{align}
    \v \cdot \Grad T &= \chi \Laplacian T \label{eq53.2:対流の支配方程式1} \\
    (\v \cdot \Grad) \v &= - \Grad \left( \frac{p}{\rho} \right) + \nu \Laplacian \v, \;
    \Div \v = 0.
\end{align}
この方程式系は3つの未知数$\v, T, \dfrac{p}{\rho}$および2つのパラメータ$\chi, \nu$を含んでいる.
よって方程式の解は,境界条件を通して,物体の長さスケール$l$,主流の速度$U$,流体と物体の温度差$T_1-T_0$に依存する.

これらのパラメータから無次元量を作る際,温度の次元をどのように決めるかという問題が生じる.
これを解決するために,温度が\eqref{eq53.2:対流の支配方程式1}のみに現れることに注目しよう.
\eqref{eq53.2:対流の支配方程式1}は$T$について線形であるから,温度に任意定数をかけても方程式を満たす.
したがって温度自身に(他の次元に依存しない)次元を与えることができる.
ここでは温度の単位をK(ケルビン)とする.




よって,この対流は5つのパラメータ
$\nu=\chi [\si{m^2s^{-1}}]$,$U[\si{m s^{-1}}]$,$l[\si{m}]$,$T_1-T_0[\si{K}]$で特徴付けられる.
これらの量から,2つの独立な無次元パラメータを作ることができる.
1つはReynolds数$\R = lU/\nu$であり,もう1つは\emph{Prandtl数}
\begin{equation}
    \mathrm{P} = \frac{\nu}{\chi}
\end{equation}
である.Prandtl数は熱拡散に対する粘性拡散の比である.

他の無次元量は$\R, \mathrm{P}$の組み合わせとして表現される.
例えば\emph{P\'{e}clet数}
\[
    \mathrm{Pe} = \R \cdot \mathrm{P} = \frac{lU}{\chi}
\]
である.P\'{e}clet数は熱拡散に対する移流拡散の比である.


Prandtl数は物性値であり,流れの性質にはよらない.
気体の場合は1のオーダーであるが,液体の場合は様々である.






\S~19と同様に,このような定常対流では,温度と速度の分布は
\begin{equation}
    \frac{T-T_0}{T_1-T_0} = f \left( \frac{\vec{r}}{l}, \R, \mathrm{P} \right), \quad
    \frac{\v}{U} = \vec{f} \left( \frac{\vec{r}}{l}, \R \right)
\end{equation}
の形になると結論づけることができる.
温度分布を与える無次元関数$f$は2つのパラメータ$\R, \mathrm{P}$に依存する.
しかし,温度を決める方程式には$\chi$が現れないから,$\vec{f}$は$\R$のみに依存する.
そして,Reynolds数とPrandtl数が等しい2つの対流は相似である.



さて,物体(固体)と流体の間の熱輸送は,\emph{熱伝達係数}
\begin{equation}
    \alpha = \frac{q}{T_1-T_0}
\end{equation}
により特徴付けられる.
ここで$q$は表面を通る熱フラックス密度であり,$T_1-T_0$は固体と流体の間の特徴的な温度差である.
流体の温度差が分かれば,境界で$q = -\kappa \dpdv{T}{n}$を計算することにより$\alpha$を求めることができる.


熱伝達係数は無次元ではない.
熱輸送を特徴付ける無次元量は\emph{Nusselt数}
\begin{equation}
    \mathrm{N} = \frac{\alpha l}{\kappa}
\end{equation}
である.
Nusselt数は,静止した流体での熱伝導に対する,対流による熱輸送の比であり,対流がなければ$\mathrm{N}=1$である.
相似則の議論から,どのような対流に対しても,Nusselt数はReynolds数とPrandtl数のみの関数として表される.
\begin{equation}
    \mathrm{N} = f(\R, \mathrm{P})
\end{equation}

%%%%%%%%%% 問題1 %%%%%%%%%%

\begin{mondai}{}{}
円形断面の管において,壁の温度が軸方向に線形に増加するものとする.
管内のPoiseuille流中の温度分布を求めよ.
\end{mondai}
\begin{kaitou}
管の軸を$z$軸とする円筒座標系を用いる.
$Az$を管の温度として,解を$T=Az + f(r)$の形に求めよう.
速度分布は,(17.9)より$v_z = 2\overline{v} (1-r^2/R^2)$となる($\overline{v}$は平均速度).
\eqref{eq53.2:対流の支配方程式1}に代入し
\[
    Av_z = \chi \Laplacian T = \chi \frac{1}{r} \dv{r} \left( r \dv{f}{r} \right)
\]
\begin{align*}
    \dv{r} \left( r \dv{f}{r} \right) &= \frac{2\overline{v}A}{\chi} \left( r - \frac{r^3}{R^2} \right) \\
    r \dv{f}{r} &= \frac{2\overline{v}A}{\chi} \left( \frac{r^2}{2} - \frac{r^4}{4R^2} \right) + C_1 \\
    \dv{f}{r} &= \frac{2\overline{v}A}{\chi} \left( \frac{r}{2} - \frac{r^3}{4R^2} \right) + \frac{C_1}{r} \\
    f &= \frac{2\overline{v}A}{\chi} \left( \frac{r^2}{4} - \frac{r^4}{16R^2} \right) + C_1 \log r + C_2 .
\end{align*}
解は$r=0$で有限でなければならないから$C_1=0$となる.
また$r=R$で$T=0$とすれば
\[
    C_2 = -\frac{2\overline{v}A}{\chi} \left( \frac{R^2}{4} - \frac{R^2}{16} \right) = -\frac{3\overline{v}AR^2}{8\chi}
\]
\[
    \yueni f(r) = -\frac{\overline{v}AR^2}{2\chi} \left[ \frac{3}{4} - \left( \frac{r}{R} \right)^2 + \frac{1}{4} \left( \frac{r}{R} \right)^4 \right]
\]
となる.
熱フラックス密度は
\[
    q = \kappa \eval{\pdv{T}{r}}_{r=R}
    = \kappa \cdot \frac{2\overline{v}A}{\chi} \left( \frac{R}{2} - \frac{R}{4} \right) 
    = \frac{1}{2} \rho c_p \overline{v} AR
\]
であり,温度伝導度$\chi$に依存しない.








\end{kaitou}


\BackToTheToc