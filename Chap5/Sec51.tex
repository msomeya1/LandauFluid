\section{無限媒質中の熱伝導}\label{sec:51}
静止した無限媒質中の熱伝導を考える.
$t=0$における全空間の温度分布$T_0(\vec{r})$が与えられているとき,$t>0$での温度分布を求める問題を考えよう.


$T$を座標に関してFourier変換しよう.
\begin{equation}
    T_{\vec{k}}(t) = \int T(\vec{r}, t) e^{-i\vec{k}\cdot\vec{r}} \, d^3\vec{r}, \quad
    T(\vec{r}, t) = \frac{1}{(2\pi)^3} \int T_{\vec{k}}(t) e^{i\vec{k}\cdot\vec{r}} \, d^3\vec{k}
\end{equation}
後者を\eqref{eq50.4:静止流体での熱伝導方程式}に代入し
\[
    \frac{1}{(2\pi)^3} \int \left( \dv{T_{\vec{k}}}{t} + \chi k^2 T_{\vec{k}} \right) e^{i\vec{k}\cdot\vec{r}} \, d^3\vec{k} = 0
\]
\[
    \dv{T_{\vec{k}}}{t} + \chi k^2 T_{\vec{k}} = 0
    \qquad \yueni T_{\vec{k}} = T_{0\vec{k}} \exp(-\chi k^2 t) .
\]
$t=0$で$T=T_0$に戻らなければならないから,$T_{0\vec{k}}$は$T_0$のFourier変換である.
\[
    T_{0\vec{k}} = \int T_0(\vec{r}') e^{-i\vec{k}\cdot\vec{r}'} \, d^3\vec{r}'
\]
したがって
\[
    T(\vec{r}, t) = \frac{1}{(2\pi)^3} \int d^3\vec{k} \int d^3\vec{r}' \, T_0(\vec{r}') e^{-\chi k^2 t} e^{i\vec{k}\cdot(\vec{r}-\vec{r}')} .
\]
積分の順序を入れ替え,まず$\vec{k}$についての積分を行おう.
\begin{align*}
    \int_{-\infty}^{\infty} dk_x \, e^{-\chi {k_x}^2 t} e^{ik_x(x-x')}
    &= \int_{-\infty}^{\infty} dk_x \, e^{-\chi {k_x}^2 t} \left\{ \cos k_x(x-x') + i \, \cancel{ \sin k_x(x-x') } \right\} \\
    & \gyoukan{←積分公式 $\displaystyle \int_{-\infty}^{\infty} e^{-\alpha x^2} \cos \beta x \, dx %
    = \sqrt{\dfrac{\pi}{\alpha}} e^{-\beta^2/4\alpha} \quad (\alpha>0) $ } \\
    &= \sqrt{\frac{\pi}{\chi t}} \exp\left[ - \frac{(x-x')^2}{4\chi t} \right]
\end{align*}
であるから
\begin{align}
    T(\vec{r}, t) 
    &= \frac{1}{(2\pi)^3} \int d^3\vec{r}' \, T_0(\vec{r}') \sqrt{\frac{\pi}{\chi t}}^3 %
    \exp[ - \frac{ (x-x')^2 + (y-y')^2 + (z-z')^2 }{4\chi t} ] \notag \\
    &= \frac{1}{8(\pi\chi t)^{3/2}} \int d^3\vec{r}' \, T_0(\vec{r}') \exp\left[ - \frac{(\vec{r}-\vec{r}')^2}{4\chi t} \right] .
    \label{eq51.2:3D熱伝導方程式の解}
\end{align}
この式は問題に対する完全な解である.
つまり,与えられた初期の温度分布から,任意の時刻における温度分布を求めることができる.

特に,初期の温度分布が$x$のみに依存する場合,$y',z'$についての積分を行えば
\begin{details}
$\displaystyle \int_{-\infty}^{\infty} dy' \, e^{-(y-y')^2/4\chi t} = 2\sqrt{\pi\chi t}$より
\end{details}
\begin{equation}
    T(x, t) = \frac{1}{2\sqrt{\pi\chi t}} \int_{-\infty}^{\infty} dx' \, T_0(x') \exp\left[ - \frac{(x-x')^2}{4\chi t} \right]
\end{equation}
となる.



時刻$t=0$で,原点では$T_0=\infty$であるが,それ以外の場所では$T_0=0$であるような状況を考えよう
(但し,全熱量$\propto \displaystyle\int d^3\vec{r} \, T_0(\vec{r})$は有限とする).
そのような初期分布はデルタ関数で表される.
\begin{equation}
    T_0(\vec{r}) = \const \times \delta(\vec{r})
\end{equation}
式\eqref{eq51.2:3D熱伝導方程式の解}は次のようになる.
\begin{equation}\label{eq51.5:3D点熱源の場合の解}
    T(\vec{r}, t) = \const \times \frac{1}{8(\pi\chi t)^{3/2}} \exp\left( - \frac{r^2}{4\chi t} \right) 
\end{equation}
時間が経過すると,原点$r=0$での温度は$t^{-3/2}$で減少する.
それに伴って周りの温度は上昇し,温度が0よりもかなり高い領域が広がっていく(図39).
この広がり方は主に\eqref{eq51.5:3D点熱源の場合の解}の指数因子によって決まる.
この領域の大きさのオーダーを$l$とすると
\begin{equation}\label{eq51.6:熱拡散のルートt則}
    \frac{l^2}{4\chi t} \sim \const
    \qquad\yueni l \sim \sqrt{\chi t}.
\end{equation}
つまり,時間の平方根で増加する.


\eqref{eq51.5:3D点熱源の場合の解}は3次元の場合の式であるが,1次元の場合にも同様の式が成り立つ.
$t=0$で,有限の熱が平面$x=0$に集中しているとき,その後の温度分布は
\begin{equation}\label{eq51.7:1D面熱源の場合の解}
    T(x, t) = \const \times \frac{1}{2\sqrt{\pi\chi t}} \exp\left( - \frac{x^2}{4\chi t} \right) 
\end{equation}
で与えられる.




\eqref{eq51.6:熱拡散のルートt則}を,やや異なった見方で解釈することができる.
物体の大きさのオーダーを$l$としよう.
すると,物体が非一様に加熱される場合,物体の温度がどこでもほぼ等しくなるのにかかる時間のオーダー$\tau$は
\begin{equation}
    \tau \sim \frac{l^2}{\chi}
\end{equation}
となる.
$\tau$は熱伝導の\emph{緩和時間}と呼ばれ,物体の長さの次元の2乗に比例し,温度伝導度に反比例する.




以上で得られた式によって記述される熱伝導の過程は,任意の摂動が瞬時に全空間に伝わるという特徴を持っている.
\eqref{eq51.5:3D点熱源の場合の解}から明らかなように,点源から発せられた熱は次の瞬間には媒質に伝わり,無限遠でのみ$T=0$である.
この特徴は,$\chi$が温度に依存しても(全空間で$\chi=0$でない限り)成り立つ.
しかし,$T=0$で$\chi=0$となるような温度依存性を持つ媒質の場合,熱の伝播は遅れ,摂動の効果は有限の領域にしか現れない
(考えている領域の外では$T=0$とする).
この結果,および,以下の問題の解は,
Ya. B. Zel'dovich and A. S. Kompaneets (1950)
による.





%%%%%%%%%% 問題1 %%%%%%%%%%

\begin{mondai}{}{}
熱伝導度と比熱は温度の冪で変化するが,密度は一定の媒質がある.
ある瞬間に任意の熱源から熱が伝播した領域の境界付近で,温度が0に近づく様子を調べよ(考えている領域の外では温度は0とする).
\end{mondai}
\begin{kaitou}
熱伝導度$\kappa$と比熱$c_p$が温度の冪で書けるなら,温度伝導度$\chi=\dfrac{\kappa}{\rho c_p}$やエンタルピー
$h = \displaystyle\int c_p \, dT$も温度の冪で変化する.
よって,$H=\rho h$を単位体積当たりのエンタルピーとして,$\chi = a H^n$と書ける.
熱伝導方程式$\rho c_p \dpdv{T}{t} = \Div(\kappa \Grad T)$は
\[
    \pdv{H}{t} = a \Div (H^n \Grad H)
    \mytag{1}
\]
となる.


ある短い時間の間には,領域の境界の一部を平面とみなし,一定速度$v$で動くと仮定することができる.
よって\ajMaru{1}の解は$H=H(x-vt)$とおけて($x$は境界面に垂直な方向の座標)
\[
    -v \dv{H}{x} = a \dv{x} \left( H^n \dv{H}{x} \right)
    \mytag{2}
\]
となる.1回積分して
\[
    -v H = a H^n \dv{H}{x}.
\]
$\ddv{x}{H} \sim H^{n-1}$より$x \sim H^n$となり
\[
    H \sim |x|
    \mytag{3}
\]
となる($|x|$は境界からの距離).

もし$n>0$なら,加熱された領域は$T=0$の領域との境界を持つ.
もし$n \ika 0$なら,$T=0$となる有限な領域は存在しない.つまり熱は各瞬間に全空間に分布する.


\end{kaitou}




%%%%%%%%%% 問題2 %%%%%%%%%%

\begin{mondai}{}{}
前問と同様の媒質において,$t=0$で単位面積あたり$Q$の熱が平面$x=0$に集中している(この面以外では$T=0$とする).
$t>0$での温度分布を求めよ.
\end{mondai}
\begin{kaitou}
1次元の場合,方程式\ajMaru{1}は
\[
    \pdv{H}{t} = a \pdv{x} \left( H^n \pdv{H}{x} \right)
    \mytag{4}
\]
となる.
この問題に現れるパラメータは
$Q[\si{J m^{-2}}]$,
$a=\dfrac{\chi}{H^n} \left[ \dfrac{\si{m^2 s^{-1}}}{(\si{Jm^{-3}})^n} \right]$
である.相似な解を求めるため,これらと変数$x[\si{m}]$,$t[\si{s}]$から無次元の変数$\xi$を作ろう.
$\xi = Q^\alpha a^\beta x^\gamma t^\delta$とすると
\[
    (\si{J m^{-2}})^\alpha \left( \dfrac{\si{m^2 s^{-1}}}{(\si{Jm^{-3}})^n} \right)^\beta \si{m}^\gamma \si{s}^\delta = 1
    \qquad \yueni
    \begin{cases}
        \alpha - n \beta = 0 \\
        -2\alpha + (2+3n)\beta + \gamma = 0 \\
        -\beta + \delta = 0 .
    \end{cases}
\]
$\gamma=1$と選ぶと$\alpha=-\dfrac{n}{2+n}$,$\beta=\delta=-\dfrac{1}{2+n}$となり
\[
    \xi = \frac{x}{(Q^n at)^{\frac{1}{2+n}}}
    \mytag{5}
\]
を得る.

次に,$H [\si{Jm^{-3}}]$の次元をもつ量を作ろう.
\[
    (\si{J m^{-2}})^\alpha \left( \dfrac{\si{m^2 s^{-1}}}{(\si{Jm^{-3}})^n} \right)^\beta \si{m}^\gamma \si{s}^\delta = \si{Jm^{-3}}
    \qquad \yueni
    \begin{cases}
        \alpha - n \beta = 1 \\
        -2\alpha + (2+3n)\beta + \gamma = -3 \\
        -\beta + \delta = 0.
    \end{cases}
\]
$\gamma=0$とすると$\alpha=\dfrac{2}{2+n}$,$\beta=\delta=-\dfrac{1}{2+n}$となる.
よって$H$の次元をもつ量として
$ H_0 \equiv \left( \dfrac{Q^2}{at} \right) ^{\frac{1}{2+n}} $が取れるから,求める解を次のようにおくことができる.
\[
    H(x, t) = H_0 f(\xi)
    \mytag{6}
\]
$f$は$\xi$のみの関数で,無次元である.以下,$'$(プライム)は$\xi$による微分とする.
\begin{align*}
    \pdv{H}{t} &= \dv{H_0}{t} f + H_0 f' \pdv{\xi}{t} \\
    &= -\frac{1}{2+n} \frac{H_0}{t} f + H_0 f' \left( -\frac{1}{2+n} \frac{\xi}{t} \right) \\
    &= -\frac{1}{2+n} \cdot \frac{H_0}{t} (f + \xi f').
\end{align*}
\[
    \pdv{H}{x} = H_0 f' \pdv{\xi}{x} = H_0 f' \frac{\xi}{x}, \quad
    H^n \pdv{H}{x} = {H_0}^{n+1} f^n f' \frac{\xi}{x} 
\]
\[
    \pdv{x} \left( H^n \pdv{H}{x} \right) = {H_0}^{n+1} ( f^n f' )' \left( \frac{\xi}{x}  \right)^2
\]
よって\ajMaru{4}は
\[
    -\frac{1}{2+n} \cdot \frac{H_0}{t} (f + \xi f') = a {H_0}^{n+1} ( f^n f' )' \left( \frac{\xi}{x}  \right)^2
\]
\begin{details}
無次元化を行ったから,次元を持つ量は全て消える.
\end{details}
\[
    -\frac{1}{2+n} (f + \xi f') = ( f^n f' )'
\]
\begin{equation}
    \yueni (2+n) \dv{\xi} \left( f^n \dv{f}{\xi} \right) + \xi \dv{f}{\xi} + f = 0.
    \tag{\star}
\end{equation}

(\star)を解くために$g=f^n$とおく.$f=g^{\frac{1}{n}}$であるから
\[
    f' = \frac{1}{n} g^{\frac{1}{n}-1} g', \quad
    f^n f' =  \frac{1}{n} g^{\frac{1}{n}} g'
\]
\[
    (f^n f')' =  \frac{1}{n} \left\{ \frac{1}{n} g^{\frac{1}{n}-1} (g')^2 + g^{\frac{1}{n}} g'' \right\} .
\]
よって(\star)は
\[
    \frac{2+n}{n} \left\{ \frac{1}{n} g^{\frac{1}{n}-1} (g')^2 + g^{\frac{1}{n}} g'' \right\}
    + \xi \frac{1}{n} g^{\frac{1}{n}-1} g' + g^{\frac{1}{n}} = 0.
\]
両辺を$\dfrac{1}{n}g^{\frac{1}{n}-1}$で割り
\[
    \frac{2+n}{n} \left\{ (g')^2 + n g g'' \right\} + \xi g' + ng = 0
\]
\[
    g' \left( \frac{2+n}{n} g' + \xi \right) + ng \left( \frac{2+n}{n} g'' + 1 \right) = 0.
\]
両辺に$\left( \dfrac{2+n}{n} g' + \xi \right)^{n-1}$をかけて
\[
    \left\{ g\left( \frac{2+n}{n} g' + \xi \right)^n \right\}' = 0
    \qquad \yueni
    g\left( \frac{2+n}{n} g' + \xi \right)^n = \const
\]
対称性より$\xi=0$で$f'=0$(よって$g'=0$)であるから,右辺の定数は0となり
\[
    \frac{2+n}{n} g' + \xi = 0
    \qquad\yueni g = \frac{n}{2+n} \cdot \frac{{\xi_0}^2 - \xi^2}{2}.
\]
ここで$\xi_0$は積分定数である.
以上より
\[
    f(\xi) = \left[ \frac{n}{2+n} \cdot \frac{{\xi_0}^2 - \xi^2}{2} \right]^{\frac{1}{n}}
    \mytag{7}
\]
を得る.


\subsubsection*{$n>0$のとき}
\ajMaru{7}は$-\xi_0 \ika \xi \ika \xi_0$に対応する,2つの平面$x=\pm x_0$に挟まれた領域内の温度分布を与える(その外側では$H=0$である).
\ajMaru{5}から,加熱された領域は$x_0 \sim t^{\frac{1}{2+n}}$で広がる.

定数$\xi_0$は,全ての熱量が一定という条件から決まる.
\[
    \int_{-x_0}^{x_0} H \, dx = \int_{-\xi_0}^{\xi_0} H_0 f(\xi) (Q^n at)^{\frac{1}{2+n}} \, d\xi
    = Q \int_{-\xi_0}^{\xi_0} f(\xi) \, d\xi
\]
が$Q$に等しいから,条件は
\[
    \int_{-\xi_0}^{\xi_0} \left[ \frac{n}{2+n} \cdot \frac{{\xi_0}^2 - \xi^2}{2} \right]^{\frac{1}{n}} \, d\xi = 1
\]
と書ける.$\xi = \xi_0 y$と置換し
\[
    \int_{-1}^{1} \left[ \frac{n {\xi_0}^2}{2(2+n)} (1-y^2) \right]^{\frac{1}{n}} \xi_0\, dy = 1 .
\]
岩波数学公式I(微分積分・平面曲線)第V篇・第1章\S~46(i)によれば
\[
    \int_{0}^{1} x^\alpha (1-x^\lambda)^\beta \, dx = \frac{1}{\lambda} B \left( \frac{\alpha+1}{\lambda}, \beta+1 \right) \quad
    (\alpha, \beta > -1, \lambda>0)
\]
であるから,$\alpha=0, \beta=\dfrac{1}{n}, \lambda=2$として
\[
    \int_{0}^{1} (1-x^2)^{\frac{1}{n}} \, dx = \frac{1}{2} B \left( \frac{1}{2}, \frac{1}{n}+1 \right)
\]
\[
    \int_{-1}^{1} (1-x^2)^{\frac{1}{n}} \, dx = B \left( \frac{1}{2}, \frac{1}{n}+1 \right)
    = \frac{ \Gamma\left(\frac{1}{2}\right) \Gamma\left(\frac{1}{n}+1\right) }{\Gamma\left(\frac{1}{n}+\frac{1}{2}+1\right)}
    = \frac{ \sqrt{\pi} \cdot \frac{1}{n} \Gamma\left(\frac{1}{n}\right) }{\left(\frac{1}{n}+\frac{1}{2}\right)\Gamma\left(\frac{1}{n}+\frac{1}{2}\right)}
    = \frac{ 2\sqrt{\pi} \Gamma\left(\frac{1}{n}\right) }{(2+n)\Gamma\left(\frac{1}{n}+\frac{1}{2}\right)}
\]
となる.規格化の積分は
\[
    \left[ \frac{n {\xi_0}^2}{2(2+n)} \right]^{\frac{1}{n}} \xi_0 \cdot \frac{ 2\sqrt{\pi} \Gamma\left(\frac{1}{n}\right) }{(2+n)\Gamma\left(\frac{1}{n}+\frac{1}{2}\right)} = 1 .
\]
\[
    {\xi_0}^{\frac{2}{n}+1} = \frac{2^{\frac{1}{n}-1} (2+n)^{\frac{1}{n}+1}}{n^{\frac{1}{n}} \sqrt{\pi} } \cdot \frac{ \Gamma\left(\frac{1}{n}+\frac{1}{2}\right) }{ \Gamma\left(\frac{1}{n}\right) }
\]
両辺$n$乗して
\[
    {\xi_0}^{2+n} = \frac{2^{1-n} (2+n)^{1+n}}{n \pi^{\frac{n}{2}} } \cdot \left( \frac{ \Gamma\left(\frac{1}{n}+\frac{1}{2}\right) }{ \Gamma\left(\frac{1}{n}\right) } \right)^n .
    \mytag{9}
\]
こうして定数$\xi_0$が決まった.




\subsubsection*{$n=-\nu<0$のとき}
$-{\xi_0}^2$を改めて${\xi_0}^2$とおくと
\[
    f(\xi) = \left[ \frac{\nu}{2(2-\nu)} ({\xi_0}^2 + \xi^2) \right]^{-\frac{1}{\nu}}
    \mytag{10}
\]
となる.この場合,熱は全空間に分布しており,原点から十分離れたところでは,$H$は$x^{-\frac{2}{\nu}}$で減少する.
但し解\ajMaru{10}は$\nu \ijou 2$のときは意味がない(物理的には,熱が一瞬で無限遠に到達することを意味する).

$\nu<2$のとき,規格化条件は
\[
    \int_{-\infty}^{\infty} \left[ \frac{\nu}{2(2-\nu)} ({\xi_0}^2 + \xi^2) \right]^{-\frac{1}{\nu}} \, d\xi = 1
\]
\[
    \int_{-\infty}^{\infty} \left[ \frac{\nu {\xi_0}^2}{2(2-\nu)} (1 +y^2) \right]^{-\frac{1}{\nu}} \xi_0\, dy = 1 .
\]
再び岩波数学公式\S~46(ii)によれば
\[
    \int_{0}^{\infty} \frac{dx}{x^\alpha (1+x^\lambda)^\beta} = \frac{1}{\lambda} B \left( \beta - \frac{1-\alpha}{\lambda}, \frac{1-\alpha}{\lambda} \right) \quad
    (\alpha<1, \beta,\lambda >0, \lambda\beta>1-\alpha)
\]
であるから,$\alpha=0, \beta=\dfrac{1}{\nu}, \lambda=2$として
\[
    \int_{0}^{\infty} (1+x^2)^{-\frac{1}{\nu}} \, dx = \frac{1}{2} B \left( \frac{1}{\nu}-\frac{1}{2}, \frac{1}{2} \right)
\]
\[
    \int_{-\infty}^{\infty} (1+x^2)^{-\frac{1}{\nu}} \, dx = B \left( \frac{1}{\nu}-\frac{1}{2}, \frac{1}{2} \right)
    = \frac{ \Gamma\left(\frac{1}{\nu}-\frac{1}{2}\right) \Gamma\left(\frac{1}{2}\right) }{\Gamma\left(\frac{1}{\nu}\right)}
    = \frac{ \sqrt{\pi} \Gamma\left(\frac{1}{\nu}-\frac{1}{2}\right) }{\Gamma\left(\frac{1}{\nu}\right)}
\]
となる.規格化の積分は
\[
    \left[ \frac{\nu {\xi_0}^2}{2(2-\nu)} \right]^{-\frac{1}{\nu}} \xi_0 \cdot \frac{ \sqrt{\pi} \Gamma\left(\frac{1}{\nu}-\frac{1}{2}\right) }{\Gamma\left(\frac{1}{\nu}\right)} = 1 .
\]
\[
    {\xi_0}^{\frac{2}{\nu}-1} = \frac{ \sqrt{\pi} \nu^{-\frac{1}{\nu}} }{ \{ 2(2-\nu)\}^{-\frac{1}{\nu}} } \cdot \frac{ \Gamma\left(\frac{1}{\nu}-\frac{1}{2}\right) }{\Gamma\left(\frac{1}{\nu}\right)}
\]
両辺$\nu$乗して
\[
    {\xi_0}^{2-\nu} = \frac{2(2-\nu) \pi^{\frac{\nu}{2}}}{\nu} \cdot \left( \frac{ \Gamma\left(\frac{1}{\nu}-\frac{1}{2}\right) }{\Gamma\left(\frac{1}{\nu}\right)} \right)^\nu
    \mytag{11}
\]
を得る.



\subsubsection*{$n\to 0$のとき}
$\xi \to \dfrac{x}{\sqrt{at}}$,\ajMaru{9}と
\[
    \lim_{n \to 0} \left( \frac{ \Gamma\left(\frac{1}{n}+\frac{1}{2}\right) }{ \Gamma\left(\frac{1}{n}\right) } \right)^n =1
\]
より${\xi_0}^2 \to \dfrac{2\cdot2}{n}$である.また\ajMaru{7}より
\[
    f(\xi) \to \lim_{n=0} \left[ \frac{n}{4} \left( \frac{4}{n} - \frac{x^2}{at} \right) \right]^{\frac{1}{n}}
    = \lim_{n=0} \left( 1 - \frac{nx^2}{4at} \right)^{\frac{1}{n}}
    = \exp\left( - \frac{x^2}{4at} \right).
\]
$H=H_0 e^{-x^2/4at}$として規格化条件は
\[
    Q = \int_{-\infty}^{\infty} H_0 e^{-x^2/4at} \, dx = H_0 \cdot \sqrt{\pi \cdot 4at}
    \qquad\yueni H_0 = \frac{Q}{2\sqrt{\pi at}}.
\]
したがって
\[
    H \to \frac{Q}{2\sqrt{\pi at}} \exp\left( - \frac{x^2}{4at} \right)
\]
となり\eqref{eq51.7:1D面熱源の場合の解}と一致する.


\end{kaitou}




\BackToTheToc