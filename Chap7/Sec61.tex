\section{Laplaceの式}

この章では,2つの連続媒質の境界面付近で起こる現象を調べることにする
(もちろん実際には,媒質は薄い遷移層に分かれているが,この層は薄いので面とみなすことができる).
境界が曲面の場合,表面付近で2つの媒質の圧力は異なっている.
この圧力差(\emph{表面圧})を求めるために,境界面の性質を考慮して,2つの媒質が熱力学的に平衡状態にある条件を書き下そう.



2つの媒質を1,2とし,境界面が無限小の変位$\delta\zeta$を行ったとする
($\delta\zeta$は,変化後の境界面から変化前の境界面におろした垂線の長さであり,
1から2に変位するとき$\delta\zeta>0$と定める).
境界の面積要素を$dS$とすると,変化前後の表面に挟まれた体積要素は$\delta\zeta \, dS$となる.
媒質1,2の圧力を$p_1, p_2$とすると,以上のような変位を行うのに必要な仕事は
\[
    \int (-p_1 + p_2) \delta\zeta \, dS
\]
である.

次に,表面積を変化させるのに必要な仕事は,表面積の変化$\delta S$に比例し$\alpha \, \delta S$と表せる
($\alpha$は\emph{表面張力係数}と呼ばれる).
よって,表面を変位させるのに必要な仕事$\delta W$は
\begin{equation}\label{eq61.1:表面を変位させるための仕事}
    \delta W = -\int (p_1 - p_2) \delta\zeta \, dS + \alpha \, \delta S
\end{equation}
となる.
熱力学的に平衡であるための条件は,もちろん$\delta W = 0$である.




次に,$\delta S$を$\delta\zeta$と関係付けるために,曲率半径を考えよう.
ある点の主曲率半径を$R_1, R_2$とする(媒質2側に曲がるとき曲率が正とする).
また,曲率円上の線素を$dl_1, dl_2$とする.
表面が$\delta\zeta$変位することで,
主曲率半径は$R_1'=R_1+\delta\zeta$,$R_2'=R_2+\delta\zeta$となり,
線素は$dl_1' = \dfrac{R_1+\delta\zeta}{R_1} dl_1$,$dl_2' = \dfrac{R_2+\delta\zeta}{R_2} dl_2$となる.
したがって微小面積$dS=dl_1dl_2$の変化は
\[
    dS' - dS = dl_1'dl_2' - dl_1dl_2
    = \left[ \left( 1 + \frac{\delta\zeta}{R_1} \right) \left( 1 + \frac{\delta\zeta}{R_2} \right) -1 \right] dl_1dl_2
    \simeq \left( \frac{\delta\zeta}{R_1} + \frac{\delta\zeta}{R_2} \right) dS ,
\]
つまり
\begin{equation}\label{eq61.2:面積の変分と曲率}
    \delta S = \int \left( \frac{1}{R_1} + \frac{1}{R_2} \right) \delta\zeta \, dS
\end{equation}
となる.これを\eqref{eq61.1:表面を変位させるための仕事}に代入し
\[
    \int \left\{ -(p_1 - p_2) + \alpha \left( \frac{1}{R_1} + \frac{1}{R_2} \right) \right\} \delta\zeta \, dS = 0
\]
\begin{equation}\label{eq61.3:Laplaceの式}
    \yueni p_1 - p_2 = \alpha \left( \frac{1}{R_1} + \frac{1}{R_2} \right) .
\end{equation}
これが\emph{Laplaceの式}であり,表面圧(\emph{Laplace圧})$p_1-p_2$を与える.
もし$R_1$と$R_2$が正なら$p_1>p_2$であり,圧力の大きい方の媒質が凸になる.
また境界が平面($R_1, R_2 \to \infty$)なら$p_1=p_2$であり,2つの媒質の圧力は等しい.



隣接する2つの媒質の力学的平衡を調べるために\eqref{eq61.3:Laplaceの式}を適用してみよう.
境界面にも媒質にも,外力は作用していないものとする.
よって圧力は2つの媒質で等しいから,平衡状態の方程式は
\begin{equation}\label{eq61.4:力学的平衡の式}
    \frac{1}{R_1} + \frac{1}{R_2} = \const
\end{equation}
である.
つまり,曲率の和は任意の自由表面上で一定でなければならない.
もし表面全体が自由表面なら,条件\eqref{eq61.4:力学的平衡の式}は表面が球状でなければならないことを意味する
(例:重力が無視できる場合の,小さい液滴の表面).
しかし,表面がある曲線によって支えられている場合(例:固体枠上の液体薄膜),表面の形はそれほど単純ではない.



条件\eqref{eq61.4:力学的平衡の式}が,固体枠に支えられた液体薄膜の平衡状態に適用される場合には,右辺の定数は0でなければならない.
和$1/R_1+1/R_2$は薄膜表面上のどこでも等しい値を取る.
一方,曲率は,反対側から見れば大きさが同じで符号が逆になるから,この定数は膜の反対側で逆符号となる.
よって
\begin{equation}
    \frac{1}{R_1} + \frac{1}{R_2} = 0 .
\end{equation}


次に,重力場中にある媒質表面の平衡条件を考えよう.
簡単のため,媒質2は大気であり,その圧力は表面で一定とみなせるとする.
また,媒質1は非圧縮性流体とする.
このとき$p_2=\const$であり,式(3.2)より$p_1 = \const - \rho gz$となる($z$軸は鉛直上向きにとる).
よって平衡条件は
\begin{equation}\label{eq61.6:重力場中の流体表面の平衡条件}
    \frac{1}{R_1} + \frac{1}{R_2} + \frac{\rho gz}{\alpha} = \const
\end{equation}
となる.


個々の場合に平衡状態での流体表面の形を求めるには,式\eqref{eq61.6:重力場中の流体表面の平衡条件}の形ではなく,
全自由エネルギーを最小にするという変分法の問題を解く方が便利である,ということを指摘しておこう.
この場合に解くべき式は
\begin{equation}\label{eq61.7:重力場中の流体表面の平衡条件(変分)}
    \alpha \int dS + \rho g \int z \, dV = \min,
\end{equation}
\begin{equation}
    \int dV = \const
\end{equation}
である.



定数$\rho, g, \alpha$は,条件\eqref{eq61.6:重力場中の流体表面の平衡条件}
\eqref{eq61.7:重力場中の流体表面の平衡条件(変分)}の中で$\alpha/\rho g$という形でしか出てこない.
この量の次元は長さの2乗である.
長さの次元を持つ量
\begin{equation}
    a = \sqrt{ \frac{2\alpha}{\rho g} }
\end{equation}
は,今考えている物質の\emph{毛細定数}と呼ばれる.
流体の表面の形はこの量のみで決まる.
毛細定数が,媒質の特徴的な長さのオーダーに比べて大きい場合には,表面の形を求める際に重力の影響を無視することができる.



条件\eqref{eq61.4:力学的平衡の式}\eqref{eq61.6:重力場中の流体表面の平衡条件}から表面の形を求めるためには,
表面の形が与えられた場合に,その曲率半径を求める式が必要である.
この公式は微分幾何により与えられるが,一般の場合にはやや複雑である.
しかし,表面が平面からわずかにずれている特殊な場合には,この式は簡単になる.
ここでは,微分幾何を用いずに,適当な式を導くことにしよう.

表面の形を$z = \zeta(x,y) $とし,$\zeta$は十分小さいとする.
よく知られているように,表面積は次で与えられる.
\begin{equation}
    S = \bigintsss \sqrt{ 1 + \left( \pdv{\zeta}{x} \right)^2 + \left( \pdv{\zeta}{y} \right)^2 } dxdy
    \simeq \bigintsss \left[ 1 + \frac{1}{2} \left( \pdv{\zeta}{x} \right)^2 + \frac{1}{2} \left( \pdv{\zeta}{y} \right)^2 \right] dxdy
\end{equation}
その変分は
\begin{align*}
    \delta S &= \int \left( \pdv{\zeta}{x} \pdv{\delta\zeta}{x} + \pdv{\zeta}{y} \pdv{\delta\zeta}{y} \right) dxdy \\
    & \gyoukan{←部分積分} \\
    &= - \int \left( \pdv[2]{\zeta}{x} + \pdv[2]{\zeta}{y} \right) \delta\zeta \, dxdy .
\end{align*}
これを\eqref{eq61.2:面積の変分と曲率}と比べ
\begin{equation}\label{eq61.11:曲率と微小変位の関係}
    \frac{1}{R_1} + \frac{1}{R_2} = - \left( \pdv[2]{\zeta}{x} + \pdv[2]{\zeta}{y} \right)
\end{equation}
を得る.


3つの媒質が隣接して平衡状態にある場合,表面の形は,共通の交線上で表面張力の和が0になるという条件から求められる.
この条件は,表面張力係数$\alpha$の値から決まるある角度(\emph{接触角})で境界面同士が交わることを意味する.



最後に,表面張力を考慮して,運動している2つの流体の境界面で満たされるべき条件を考えよう.
表面張力が無視できる場合は
\[
    ( \sigma_{2,ij} - \sigma_{1,ij} ) n_j = 0
\]
であったが,表面張力を考える場合には,右辺にLaplaceの式から決まる項を加えなければならない.
\begin{equation}
    ( \sigma_{2,ij} - \sigma_{1,ij} ) n_j = \alpha \left( \frac{1}{R_1} + \frac{1}{R_2} \right) n_i
\end{equation}
$\sigma_{ij} = -p \delta_{ij} + \sigma_{ij}'$を代入して粘性応力テンソルを用いて書けば
\begin{equation}\label{eq61.13:粘性流体の境界条件(表面張力あり)}
    (p_1-p_2) n_i = ( \sigma_{1,ij}' - \sigma_{2,ij}' ) n_j + \alpha \left( \frac{1}{R_1} + \frac{1}{R_2} \right) n_i .
\end{equation}
2つの流体がどちらも理想流体なら$\sigma_{ij}'=0$であり,\eqref{eq61.3:Laplaceの式}に戻る.



しかし\eqref{eq61.13:粘性流体の境界条件(表面張力あり)}は,まだ完全ではない.
というのは,表面張力係数$\alpha$が(例えば温度変化によって)非一様になるかもしれないからである.
この場合,法線応力のほかに接線応力を考える必要がある.
圧力が非一様な場合,単位体積に$-\Grad p$の力が加わるのと同様に,この場合,
表面の単位面積に$\Grad \alpha$の力が(接線方向に)かかる.
符号が正であるのは,表面張力は表面積を減らすように働くからである.
\eqref{eq61.13:粘性流体の境界条件(表面張力あり)}は
\begin{equation}\label{eq61.14:αが変わる場合の境界条件}
    \left[ (p_1-p_2) - \alpha \left( \frac{1}{R_1} + \frac{1}{R_2} \right) \right] n_i 
    = ( \sigma_{1,ij}' - \sigma_{2,ij}' ) n_j +  \pdv{\alpha}{x_i}
\end{equation}
となる($\vec{n}$は媒質1に向かうようにとる).
この条件は粘性流体でなければ満たされないことに注目せよ;
理想流体では$\sigma_{ij}'=0$であるから,\eqref{eq61.14:αが変わる場合の境界条件}は
(法線方向の力)=(接線方向の力)という式になる.
これは(両辺が0というつまらない場合を除けば)成立し得ない.







\BackToTheToc