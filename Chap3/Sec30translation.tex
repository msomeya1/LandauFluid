\begin{details}
\S~30--32は,第2版で新たに書き加えられた節で,M. I. Rabinovichとの共著である.
\end{details}

\section{\spade 準周期流と周波数ロッキング(日本語訳)}
以下の議論(\S~30--32)では,ある種の幾何学的表現を用いると便利である.
そのために,我々は流体の\emph{状態空間}という数学的概念を定義する;
空間中の各点は流体の特定の速度分布(速度場)に対応する.
そして,隣接する瞬間の状態は,隣接する点に対応する
\footnote{
数学の文献では,この無限次元の関数空間(場合によってはそれに代わる有限次元の空間もある;下記参照)を\emph{位相空間}と呼ぶこともある.
ここでは,物理学における通常の意味との混同を避けるため,この用語は使わないことにする.}.



状態空間において,定常流は点,周期流は閉曲線で表現され,それぞれ\emph{極限点}(または\emph{臨界点}),\emph{リミットサイクル}と呼ばれる.
流れが安定の場合,流れの発展を表す隣接する曲線は,$t\to\infty$で極限点またはリミットサイクルに向かう傾向がある.



リミットサイクル(または極限点)は,状態空間においてある\emph{吸引領域}を持ち,その領域から出発した経路は最終的にリミットサイクルに到達する.
このことから,リミットサイクルは\emph{アトラクター}と呼ばれる.
ここで強調したいのは,体積と境界条件(と$\R$の値)が与えられた流れに対して,複数のアトラクターが存在しうるということである.
状態空間が様々なアトラクターを含み,それぞれが独自の吸引領域を持つケースが起こり得る.
つまり,$\R>\Rcr$のとき,複数の安定な流れの領域が存在し,$\R$の値への到達の仕方によって異なる領域が発生することがある.
これらの安定領域は,非線形の運動方程式の解であることを強調しておきたい
\footnote{例えば,Couette流が安定でなくなったとき,新たに生じる流れのパターンは,実は円柱を特定の角速度で回転させる過程の履歴に依存する.}.


ここで,\S~\ref{sec:26}で述べた周期流が確立する臨界値よりもさらにReynolds数を大きくしたときに起こる現象について考えてみよう.
$\R$が大きくなると,やがてこの流れが不安定になる地点に到達する.
この不安定性は,原則として\S~\ref{sec:26}で述べた元の定常流の不安定性の判定と同様の手順で調べる必要がある.
乱れのない流れは周波数$\omega_1$の周期流$\v_0(\vec{r},t)$であり,運動方程式に$\v=\v_0+\v_2$($\v_2$は小さな補正項)を代入しなければならない.
すると,$\v_2$については再び線形方程式が得られるが,係数は座標と時間の関数となり,時間に対しては周期的になる(周期$T_1=2\pi/\omega_1$).
このような方程式の解は,次のような形で求められる.
\begin{equation}\label{eq30.1:周期的な摂動v2}
    \v_2 = \vec{\Pi}(\vec{r},t) e^{-i\omega t}
\end{equation} 
ここで,$\vec{\Pi}(\vec{r},t)$は時間の周期関数で,周期は同じく$T_1$である.
虚部$\gamma_2>0$の周波数$\omega=\omega_2+i\gamma_2$が存在すると再び不安定になる
(実部$\omega_2$は新たに現れる周波数となる).



周期$T_1$の間,摂動\eqref{eq30.1:周期的な摂動v2}は因子$\mu\equiv e^{-i\omega T_1}$だけ変化する.
この因子は周期流の\emph{乗数}と呼ばれ,流れにおける摂動の増幅または減衰を表す便利な量である.
連続媒質(流体)の周期流は,乗数が無限大になり,かつ独立した摂動も無限大になることに対応する.
この流れは,Reynolds数がある値$\R_\mathrm{cr,2}$になり,1つ以上の乗数の絶対値が1となる,すなわち$\mu$が複素平面上で単位円を横切る時,安定でなくなる.
方程式は実数であるから,乗数は複素共役のペアでこの円を横切るか,実数値$+1$または$-1$で単独に横切らなければならない.
周期流で安定性が失われるとき,状態空間内の不安定なリミットサイクル近傍において,経路パターンの定性的な変化を伴う;
この変化は局所的な\emph{分岐}と呼ばれる.
分岐の性質は,乗数が単位円を横切る点によってほぼ決定される
\footnote{摂動は有限時間(1周期$T_1$)では消えないので,乗数が0になることはありえない.}.



$\alpha$を無理数として,$\mu=\exp(\mp 2\pi\alpha i)$という形の複素共役な乗数の組が単位円を横切るときの分岐を考えてみよう.
これにより,新たな独立した周波数$\omega_2=\alpha\omega_1$を持つ二次的な流れが発生し,2つの非整合な(比が無理数である)周波数を持つ準周期流になる.
状態空間でこの流れに対応するのは,2次元トーラス上の開いた巻線の形をした経路であり
\footnote{数学用語では,\emph{トーラス}は囲まれた体積を持たない表面を表す.したがって,2次元トーラスは3次元の「ドーナツ」の2次元表面である.},
現在不安定なリミットサイクルはトーラスを生成する;
周波数$\omega_1$は「生成器」の周りの回転に,$\omega_2$はトーラスの周りの回転に対応している(図18).
最初の周期流が現れたとき,自由度は1つだったように,今は任意の2つの量(位相)があり,流れは2つの自由度を持つことになる.
周期的な運動の安定性が失われ,2次元のトーラスが形成されることは,流体力学では典型的な現象である.


Reynolds数がさらに増加したとき($\R>\R_\mathrm{cr,2}$),このような分岐によって生じる流れの複雑化を考えてみよう.
$\R$が増加するにつれて,新しい周期が次々と現れると考えるのが妥当であろう.
幾何学的な表現で言えば,これは2次元トーラスの安定性が失われ,その近くに3次元トーラスが形成され,さらに分岐して4次元トーラスに置き換わり…といったようなことを意味する.
次々と現れる新しい周波数に対応するReynolds数の間隔は急速に短くなり,流れのスケールはますます小さくなる.
このように,流れは急速に複雑で混乱した形になり,規則正しい\emph{層流}(流体が層ごとに異なる速度で動く)とは対照的に,\emph{乱流}と呼ばれるようになる.




このような乱流の発達の仕方やシナリオが実際に可能であると仮定して
\footnote{以下のシナリオは,L. D. Landau(1944)とE. Hopf(1948)が独立に提案した.},
関数$\v(\vec{r},t)$の一般形を書こう.
その時間依存性は,$N$個の異なる周波数$\omega_i$によって支配され,
周期$2\pi$でそれぞれ周期的な$N$個の位相$\phi_i=\omega_it+\beta_i$(と座標)の関数として見なすことができる.
このような関数は次のような和で表される.
\begin{equation}
    \v(\vec{r},t) = \sum \vec{A}_{p_1p_2\cdots p_N}(\vec{r}) \exp\bqty{ -i \sum_{i=1}^{n} p_i\phi_i }
\end{equation}
これは(26.13)の一般化であり,和はすべての整数$p_1, p_2, \ldots , p_N$についてとる.
この式で記述される流れは$N$個の任意の初期位相$\beta_i$を含み,$N$個の自由度を持つ
\footnote{位相$\phi_i$を$N$次元トーラス上の経路を表す座標とすると,対応する速度は定数$\dot{\phi}_i=\omega_i$となる.
このため,準周期流はトーラス上の等速運動として記述することができる.}.



位相が$2\pi$の整数倍しか違わない状態は,物理的に同一である.
したがって,各位相の本質的に異なる値は$0 \leq \phi_i \leq 2\pi$の範囲にある.
ここで一組の位相,$\phi_1 = \omega_1t+\beta_1$と$\phi_2 = \omega_2t+\beta_2$を考えてみよう.
ある瞬間に$\phi_1=\alpha$とすると,$\phi_1$はすべての時刻
\[
    t = \frac{\alpha-\beta_1}{\omega_1} + 2\pi s\frac{1}{\omega_1}
\]
において$\alpha$と「同じ」値を持つことになる($s$は任意の整数).これらの時刻において
\[
    \phi_2 = \beta_2 + \frac{\omega_2}{\omega_1} (\alpha-\beta_1+2\pi s)
\]
となる.
異なる周波数は非整合である(比が無理数である?)ため,$\omega_2/\omega_1$は無理数である.
$\phi_2$の各値を,$2\pi$の適当な整数倍を引いて0から$2\pi$までの範囲の値にすると,$s$が0から$\infty$まで変化するとき,$\phi_2$はその範囲内の任意の数に無限に近い値をとることがわかる.
つまり,十分長い時間の間に,$\phi_1,\phi_2$は任意の指定された組に無限に近い値を同時にとる.
これはどの位相でも同じである.
したがってこの乱流モデルでは,十分長い時間の間に,位相$\phi_i$の同時値の可能なセットで定義される,任意の指定された状態に無限に近い状態を流体は通過する.
しかし,その時間は$N$とともに急速に増加し,実際には周期性の痕跡が残らないほど大きくなる
\footnote{
このタイプの発達した乱流では,系(流体)が位相空間$\phi_1,\phi_2,\ldots,\phi_N$における選ばれた点の近くの与えられた微小体積に存在する確率は,
全体の体積$(2\pi)^N$に対するこの微小体積$(\delta\phi)^N$の比である.
したがって,十分長い時間の間に,系は与えられた点の近傍に,
全時間のうち$e^{-\kappa N}$(ここで$\kappa=\log(2\pi/\delta\phi)$)だけいる,と言うことができる.}.



ここで強調しておきたいのは,上述した乱流の発達の道筋は,基本的に線形化に基づくものだということである.
実際には,二次的な不安定性の発展により新たな周期解が出現しても,既に存在する周期解は消滅せず,逆にほとんど変化しないことが仮定されている.
このモデルでは,乱流はそのような多数の不変な解の重ね合わせに過ぎない.
しかし,一般にReynolds数が大きくなると解の性質が変化し,安定でなくなる.
摂動は相互作用し,流れを単純化することもあれば,複雑化することもある.
ここでは,最初の可能性について説明する.




簡単なケースとして,摂動が加わった解が2つの独立した振動数だけを含むと仮定しよう.
すでに述べたように,このような流れの幾何学的な表現は,2次元トーラス上の開いた巻線である.
$\R=\R_\mathrm{cr,1}$で生じる周波数$\omega_1$の摂動は,当然,$\R=\R_\mathrm{cr,2}$(周波数$\omega_2$の摂動が生じる)付近で強くなると仮定し,
したがって,その近傍で$\R$が比較的小さく変化しても変化しないと見なすことができる.
そこで,周波数$\omega_1$の周期流を背景とした周波数$\omega_2$の摂動の時間発展を記述するために,新たな変数 
\begin{equation}
    a_2(t) = \abs{a_2(t)}e^{-i\phi_2(t)}
\end{equation}
を用いる.
$\abs{a_2}$はトーラス生成器までの最短距離(周波数$\omega_1$の不安定なリミットサイクル),つまり二次的な周期流の相対振幅であり,$\phi_2$は後者の位相である.
ここで,周期$T_1=2\pi/\omega_1$の倍数である離散的な瞬間における$a_2(t)$の振る舞いを考えてみよう.
1周期の間,周波数$\omega_2$の摂動は乗数
\[
    \mu = \abs{\mu} \exp(-2\pi i\omega_2/\omega_1)
\]
で変化する.
そのような周期の整数倍($\tau$倍とする)経った後,$a_2$は$\mu^\tau$倍される.
$\R-\Rcr$は小さいと仮定しているので,摂動の成長因子も小さく,$\abs{\mu}-1$は正だが小さい.
よって,$a_2$は期間$T_1$の間にわずかに変化するだけで,位相$\phi_2$は単に$\tau$に比例して変化する.
したがって,離散変数$\tau$を連続的であるかのように扱い,$a_2(\tau)$の変化を$\tau$に関する微分方程式で表すことができる.


乗数という概念は,不安定性の発生から非常に短い時間間隔,
すなわち摂動がまだ線形方程式で記述可能な時期に関係している.
この範囲では,$a_2(\tau)$は上記の議論に従って$\mu^\tau$で変化し 
\[
    \dv{a_2}{\tau} = a_2(\tau) \log\mu
\]
となり,臨界Reynolds数の直上で
\begin{equation}
    \log\mu = \log\abs{\mu} - 2\pi i \omega_2/\omega_1
    \simeq \abs{\mu}-1 - 2\pi i \omega_2/\omega_1
\end{equation}
となる.
これは$\ddv{a_2}{\tau}$を$a_2$と${a_2}^*$の累で展開したときの最初の項であり,
$\abs{a_2}$が(まだ小さいながらも)増加すると次の項を考慮する必要がある.
同じ振動因子を含む項は,3次の$\propto a_2\abs{a_2}^2$である.したがって,次のようになる.
\begin{equation}
    \dv{a_2}{\tau} = a_2\log\mu -\beta_2 a_2\abs{a_2}^2
\end{equation}
ここで,$\beta_2$は$\mu$と同様に$\R$に依存する複素パラメータであり,$\Re\beta_2>0$である;
(26.7)に関する対応した議論と比較しよう.この式の実部はすぐに絶対値の定常値を与える.
\[
    \vqty{{a_2}^{(0)}}^2 = \frac{\abs{\mu}-1}{\Re\beta_2}
\]
虚部は位相$\phi_2(\tau)$についての方程式を与える.
上記の絶対値の定常値で
\begin{equation}
    \dv{\phi_2}{\tau} = \frac{2\pi\omega_2}{\omega_1} + \vqty{{a_2}^{(0)}}^2 \Im\beta_2 .
\end{equation}




これによると$\phi_2$は一定の速度で回転することになるが,この性質は,しかし,考えている近似においてのみ有効である:
$\R-\Rcr$が大きくなると,回転はもはや一様ではなく,トーラス上の回転の速度はそれ自体$\phi_2$の関数である.
これを考慮し,(30.6)の右辺に小さな摂動$\Phi(\phi_2)$を加える.
$\phi_2$の物理的に異なる値はすべて0から$2\pi$の範囲にあるので,$\Phi(\phi_2)$は周期$2\pi$で周期的である.
次に,無理数の比$\omega_2/\omega_1$を有理小数で近似する(これは任意の精度で行える):
$\omega_2/\omega_1=m_2/m_1 + \Delta/2\pi$
($m_1,m_2$は整数).
このとき,方程式は次のようになる.
\begin{equation}
    \dv{\phi_2}{\tau} = \frac{2\pi m_2}{m_1} + \Delta + \vqty{{a_2}^{(0)}}^2 \Im\beta_2 + \Phi(\phi_2)
\end{equation}
ここで,$m_1T_1$の倍数の時間,すなわち$\tau=m_1\tilde{\tau}$の値($\tilde{\tau}$は整数)についてのみ位相値を考えることにする.
(30.7)の右辺第1項は,時間$m_1T_1$において位相が$2\pi m_2$だけ,つまり$2\pi$の整数倍だけ変化するが,
この変化は無視することができる.
このとき,右辺全体は小さな量になるので,関数$\phi_2(\tilde{\tau})$の変化は連続変数$\tilde{\tau}$の微分方程式で記述することができる.
\begin{equation}\label{eq30.8:φ_2(τ)の微分方程式}
    \frac{1}{m_1} \dv{\phi_2}{\tilde{\tau}} = \Delta + \vqty{{a_2}^{(0)}}^2 \Im\beta_2 + \Phi(\phi_2)
\end{equation}
離散変数$\tilde{\tau}$の1ステップにおいて,$\phi_2/m_1$はわずかに変化するだけである.



一般に,\eqref{eq30.8:φ_2(τ)の微分方程式}は右辺が0となる定常解$\phi_2={\phi_2}^{(0)}$を持つ.
$\phi_2$が$m_1T_1$の倍数の時間で一定であることは,トーラス上にリミットサイクルが存在することを意味し,$m_1$ターン後に経路が閉じることを意味する.
$\Phi(\phi_2)$は周期的なので,このような解は$\Phi(\phi_2)$の上昇部分と下降部分に一組ずつ(最も簡単な場合は一組)発生する.
これらの2つのうち,後者だけが安定で,\eqref{eq30.8:φ_2(τ)の微分方程式}は$\phi_2={\phi_2}^{(0)}$付近で次のような形になる.
\[
    \dv{\phi_2}{\tilde{\tau}} = -\const \times (\phi_2 - {\phi_2}^{(0)})
\]
定数は正で,$\phi_2={\phi_2}^{(0)}$に近づく解が実際に存在する.
第二解は不安定で,定数は負である.




トーラス上の安定なリミットサイクルの形成は,準周期流の消滅と新たな周期流の確立という\emph{周波数ロッキング}と等価である.
この現象は,自由度の高い系では様々な形で起こりうるもので,非整合な周波数を多数持つ流れの重ね合わせのような流れの発生を防ぐことができる.
この意味で,Landau-Hopfシナリオが実際に発生する確率は非常に小さいと言える.
もちろん,特定のケースでロッキングが発生する前にいくつかの非整合周波数が現れないとは限らない.




\BackToTheToc