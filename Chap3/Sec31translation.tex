\section{\spade ストレンジアトラクター(日本語訳)}
\label{sec:31}
様々な流体力学的流れにおける乱流の起源について,完全な理論はまだ存在しない.
しかし,流れが乱れる過程については,主に微分方程式のモデル系の計算機による研究と,実験による裏付けに基づいて,様々なシナリオが提案されている.
\S~31と\S~32では,このような研究結果には触れず,これらの考え方の一端を説明するのが目的である.
実験結果は制限された体積内の流体力学的な流れに関するものであり,以下ではこのような流れを考えるということだけ言っておこう
\footnote{実際には,制限された体積内の熱対流と,有限な長さの同軸円筒間のCouette流を扱うことになる.
有限な大きさの物体まわりの流れにおける境界層および伴流の乱流形成のメカニズムに関する理論的な考えは,
かなりの量の実験結果が存在するにもかかわらず,これまであまり発展していない.}.



まず,一般論として以下のような重要な指摘がある;
周期流の安定性解析では,絶対値が1に近く,Rがわずかに変化した時単位円を横切ることができる乗数だけが注目される.
粘性流では,このような「危険な」乗数は,次の理由から常に有限個である.
運動方程式で許される摂動の様々なタイプ(モード)は,異なる空間スケール($\v_2$が大きく変化する距離)を持つ.
運動のスケールが小さくなると,その中の速度勾配が大きくなり,粘性によってより大きく遅らされることになる.
許容されるモードをスケールの小さい順に並べた場合,危険なものは最初の有限個だけであり,
十分に離れたところにあるものは強く減衰し,小さな絶対値を持つ乗数に対応することが確実である.
このことから,周期的粘性流の不安定性のタイプは,有限個の変数によって記述される散逸性の離散力学系と本質的に同じ方法で分析できると考えられる;
この変数とは,流体力学的に言えば,速度場の座標に関するフーリエ成分の振幅かもしれない.
それに応じて,状態空間は有限の次元を持つ.




数学的には,次のような方程式で表される系の時間変化を考えなければならない.
\begin{equation}\label{eq31.1:自律力学系の方程式}
    \dot{\vec{x}}(t) = \vec{F}(\vec{x})
\end{equation}
ここで,$\vec{x}(t)$は系を記述する$n$個の量$x^{(1)},x^{(2)}, \ldots , x^{(n)}$の空間におけるベクトルである.
関数$\vec{F}$はパラメータに依存し,その変化は流れの性質を変えうる
\footnote{数学の用語では,$\vec{F}$は系の\emph{ベクトル場}である.
\eqref{eq31.1:自律力学系の方程式}のように時間に陽に依存しない場合,系は\emph{自律的}であるという.}.
散逸系では,$\vec{x}$空間での$\dot{\vec{x}}$の発散は負であり,これは運動中にその空間での体積が収縮することを表している
\footnote{Hamilton力学系では,Liouvilleの定理により発散は0であり,その場合の$\vec{x}$の成分は系の一般化座標$q$と運動量$p$である.}.
\begin{equation}
    \Div\dot{\vec{x}} = \Div\vec{F} \equiv \pdv{F^{(i)}}{x^{(i)}} < 0.
\end{equation}


ここで,異なる周期流の間で起こりうる相互作用の結果に戻ろう.
周波数ロッキングは流れを単純化するが,相互作用によって準周期性が排除され,描像が著しく複雑になることがある.
これまで,周期流が不安定になると,さらに周期流が発生することが暗黙の了解となっていた.
しかし,これは論理的には必要ない.
速度揺らぎの振幅が制限されるということは,定常粘性流に対応する経路を含む状態空間が限定的に存在することを意味するだけで,
その空間の経路のパターンがどうなるかはあらかじめ分からない.
リミットサイクルやトーラス上の開いた巻線(周期流,準周期流に相当)になることもあれば,全く異なる挙動を示し,複雑で混乱した形になることもある.
この可能性は,乱流形成の数学的性質の理解とそのメカニズムの解明に極めて重要である.




限られた体積の中のすべての経路が不安定であると仮定すれば,体積内の複雑で混乱した経路の形を知ることができる.
その中には,不安定なサイクルだけでなく,限られた領域を離れることなく無限に曲がりくねっているオープンパスも含まれるかもしれない.
不安定であるということは,状態空間で隣接している2点が,それぞれの経路を進むにつれて離れていくことを意味する.
体積が限られていて,開いた経路は自分自身に限りなく近づくことができるので,最初は隣接していた点が同じ経路に属することもある.
このような複雑で不規則な経路の振る舞いは,乱流と関連している.


この描像にはさらに,初期条件のわずかな変化に対して流れが敏感に反応するという特徴がある.
流れが安定ならば,初期条件の設定にわずかな不確実性があっても,最終的な状態の決定には同程度の不確実性しか生じない.
流れが不安定な場合,初期の不確実性は時間とともに増大し,系の最終的な状態を予測することができなくなる(N. S. Krylov 1944;M. Born 1952).




散逸系の状態空間において,不安定な経路を吸引する集合が実際に存在することがあり(E. N. Lorenz 1963),
それは通常,確率論的アトラクターあるいは\emph{ストレンジアトラクター}と呼ばれている
\footnote{通常のアトラクター(安定なリミットサイクルや極限点など)とは異なり,
「ストレンジ」という言葉は,後述するその構造の複雑さを表している.
物理学の文献では,安定経路と不安定経路を含むが,物理実験でも数値実験でも検出できないほど誘引領域の小さい,
より複雑な誘引多様体も「ストレンジアトラクター」と呼ぶ.}.




一見すると,アトラクターに属するすべての経路が不安定であるという条件は,
隣接するすべての経路が$t\to\infty$でアトラクターに近づくという条件と矛盾しているように見える
(不安定であるということは経路が離れていくことを意味するため).
この矛盾は,経路が状態空間のある方向には不安定で,他の方向には安定(つまり吸引的)であることに注目すれば解消される.
$n$次元の状態空間において,ストレンジアトラクターに属する経路が$n-1$方向(経路に沿った方向が1つある)すべてで不安定になることはありえない.
なぜなら,これは状態空間における初期体積が連続的に増加することを意味し,散逸系ではありえないからである.
その結果,ある方向では隣接する経路がアトラクター経路に近づき,他の(不安定な)方向ではアトラクター経路から遠ざかる傾向がある(図19参照).
これを\emph{サドルパス}(鞍点経路)と呼び,サドルパスの集合がストレンジアトラクターを形成する.



ストレンジアトラクターは,わずか数回の分岐で新しい周期を形成して現れることがある;
無限小の非線形性でも,準周期領域(トーラス上の開いた巻線)を排除し,トーラス上にストレンジアトラクターを形成することがある(D. Ruelle and F. Takens 1971).
しかし,これは第二の分岐(定常領域の終わりから)では起こりえない.
この場合,2次元トーラス上に開いた巻線が形成される.
小さな非線形性を考慮すると,トーラスは存在し続けるので,ストレンジアトラクターはトーラス上に収容される可能性がある.% ?
しかし,2次元の表面には,不安定なパスの集合を引き寄せることはできない.
なぜなら,状態空間内の経路は互いに(あるいはそれ自身と)交差することができない
(古典系の振る舞いにおける因果律「ある瞬間の系の状態が,その後の瞬間の振る舞いを一意に決定する」に反する)からである.
2次元の表面では,交差が不可能であるため,経路は非常に整然としており,十分にランダムになることはない.



しかし,3回目の分岐でもストレンジアトラクターが形成されることはある(必ずというわけではない).
このアトラクターは,3周波数の準周期領域に代わって,3次元のトーラス上に存在する(S. Newhouse, D. Ruelle and F. Takens 1978).



ストレンジアトラクターの複雑で混乱した経路は,状態空間の中の限られた体積内に存在する.
実際の流体力学の問題で起こりうるストレンジアトラクターの分類はまだ知られておらず,そのような分類の基準となるものさえもない.
ストレンジアトラクターの構造に関する情報は,基本的に,
実際の流体力学の方程式とは全く異なる常微分方程式のモデル系をコンピュータで解いた事例から得られたものだけである.
しかし,経路の鞍型不安定性と系の散逸性から,ストレンジアトラクターの構造について一般的な結論を導き出すことが可能である.




ここではわかりやすくするために,3次元の状態空間を対象に,2次元トーラス内のアトラクターを考えることにする.
アトラクターに至る経路のうち,「定常」乱流の確立に至る過渡的な流れの領域を記述するものを考えてみよう.
横断面において,経路(というよりその跡)はある面積を占める.
この面積が経路に沿ってどのように大きさと形を変えるかを見てみよう.
鞍点経路付近の体積要素は一方向に膨張し,他方には収縮することがわかるが,系が散逸しているので後者の効果が強く,体積は減少せざるを得ない.
これらの方向は経路に沿って変化しなければならない.
さもなければ,後者が遠くなりすぎて,流体速度が大きく変化してしまうからである.
その結果,断面が小さくなり,平坦になり,湾曲する.
これは断面全体だけでなく,断面内の各面積要素にも適用されるはずである.
こうして,断面は空洞で区切られた入れ子状のゾーンに分離される.
時間の経過とともに(つまり経路に沿って),ゾーンの数は急速に増え,幅も狭くなる.
$t\to\infty$で形成されるアトラクターは,接触していない層の無数の多様体であり,その表面は鞍点経路(その吸引方向は「外側」)を担っている.
これらの層はその側面と両端で複雑に結合している;
アトラクターに属する各経路はすべての層を通り抜け,十分長い時間の間にアトラクターの任意の点に無限に近づくという\emph{エルゴード的}性質を持つ.
層の総体積と総断面積はゼロである.



このような一方向の多様体は,数学用語ではCantor集合と呼ぶ.
Cantor構造は,アトラクターの最も特徴的な性質であり,より一般的には$n\;(>3)$次元の状態空間の性質である.




状態空間におけるストレンジアトラクターの体積は常にゼロである.
しかし,より次元の小さば別の空間ではゼロでないこともある.
この値は以下のように求められる.
$n$次元空間全体を,辺$\varepsilon$,体積$\varepsilon^n$の小さな立方体に分割する.
アトラクターを完全に覆う立方体の最小個数を$N(\varepsilon)$とする.
アトラクターの次元$D$は極限
\begin{equation}\label{eq31.3:アトラクターの次元の定義}
    D = \lim_{\varepsilon\to0} \frac{\log N(\varepsilon)}{\log(1/\varepsilon)}
\end{equation}
で定義される
\footnote{これは数学では多様体の極限容量として知られている.
その定義はHausdorff次元やフラクタル次元の定義と似ている.}.
この極限の存在は,$D$次元空間におけるアトラクターの体積が有限であることを意味する;
$\varepsilon$が小さいとき,$N(\varepsilon)\simeq V \varepsilon^{-D}$であるから($V$は定数),
$N(\varepsilon)$は$D$次元空間の体積$V$を覆う$D$次元立方体の数と見なすことができる.
\eqref{eq31.3:アトラクターの次元の定義}に従って定義すると,その次元は明らかに状態空間の全次元$n$を超えることはできないが,
それより小さくてもよく,通常の次元と違って(Cantor集合のように)非整数であってもよい
\footnote{
集合を覆う$n$次元の立方体は「ほとんど空」であってもよく,このため$D<n$とすることができる.
通常の集合に対しては,\eqref{eq31.3:アトラクターの次元の定義}の定義は明白な結果を与える;
例えば,$N$個の孤立点の集合では$N(\varepsilon)=N, D=0$,
長さ$L$の線分では$N(\varepsilon)=L/\varepsilon, D=1$,
二次元の表面積$A$では$N(\varepsilon)=A/\varepsilon^2, D=2$,などとなる.}.



次の点が重要である.
乱流が既に確立している(ストレンジアトラクターに到達している)場合,
散逸系(粘性流体)の流れは,状態空間の次元が小さい非散逸系の確率的流れと原理的に同じである.
なぜなら,定常流では,エネルギーの粘性散逸は,
平均流(あるいは他の非平衡なソース)から来るエネルギーによって,長い時間の平均で埋め合わされるからである.
したがって,アトラクターに属する「体積」要素の時間発展を(アトラクターの次元で決まるある空間で)追跡すると,
ある方向への圧縮が,他の方向への隣接経路の発散による拡張によって補われ,平均的に保存されることになる.
この性質を利用して,アトラクターの次元を別の方法から推定することができる.


前述のようにストレンジアトラクター上の運動はエルゴード的であるため,
状態空間においてアトラクターに属する1つの不安定な経路に沿った運動を分析することで,その平均特性を確立することができる.
つまり,個々の経路に沿った運動が十分な時間続くと,アトラクターの性質を再現すると仮定する.

$\vec{x}=\vec{x}_0(t)$を,このような経路の方程式,\eqref{eq31.1:自律力学系の方程式}の解としよう.
そして,この経路に沿って動くときの「球状」体積要素の変形を考えてみよう.
この変形は,差分$\vec{\xi}=\vec{x}-\vec{x}_0(t)$(つまり,考えている経路と隣接する経路とのずれ)
に関して線形化された方程式\eqref{eq31.1:自律力学系の方程式}によって与えられる.
これらの方程式を成分で書けば
\begin{equation}\label{eq31.4:線形化した自律力学系の方程式}
    \dot{\xi}^{(i)} = A_{ik}(t) \xi^{(k)}, \quad
    A_{ik}(t) = \eval{\pdv{F^{(i)}}{x^{(k)}}}_{x=x_0(t)} .
\end{equation}
経路に沿った移動で,体積要素はある方向には圧縮され,ある方向には引き伸ばされ,球は楕円体となる.
半軸の方向と長さはともに変化する.
後者を$l_s(t)$とすると($s$は方向を表わす)\emph{Lyapunov指数}は
\begin{equation}\label{eq31.5:Lyapunov指数の定義}
    L_s = \lim_{t\to\infty} \frac{1}{t} \log \frac{l_s(t)}{l(0)}
\end{equation}
で定義される.
ここで,$l(0)$は任意に選んだ時刻$t=0$における元の球の半径である.
こうして決定された量は実数であり,空間の次元$n$に等しい数である.
そのうちの1つ(経路に沿った方向に対応する)は0である
\footnote{もちろん,$t=0$の初期条件を指定した\eqref{eq31.4:線形化した自律力学系の方程式}の解は,
すべての距離$l_s(t)$が小さいままであれば,実際には隣接した経路を表している.
しかし,このことは無限に長い時間を含む\eqref{eq31.5:Lyapunov指数の定義}の定義を無意味なものにするものではない;
任意の大きな$t$に対して,線形化された方程式が考えている時間を通して有効であるように$l(0)$を小さく選ぶことができる.}.



Lyapunov指数の和は,経路に沿った,状態空間における体積要素の平均変化を与える.
経路上の任意の点における体積の相対的な局所変化は,発散$\Div\vec{x}=\Div\vec{\xi}=A_{ii}(t)$によって与えられる.
経路に沿った平均発散は次のようになる
\footnote{V. I. Oseledets, \textit{Transactions of the Moscow Mathematical Society} \textbf{19}, 197, 1969.
参照.}.
\begin{equation}
    \lim_{t\to\infty} \frac{1}{t} \int_0^t \Div\vec{\xi} \, dt = \sum_{s=1}^n L_s
\end{equation}
散逸系では,この和は負である;$n$次元の状態空間での体積は圧縮される.
ストレンジアトラクターの次元は,「その」空間において平均的に体積が保存されるように定義される.
そのために,Lyapunov指数を
$L_1 \geq L_2 \geq \cdots \geq L_n$
の順に並べ,伸縮を補償するのに必要なだけの安定な方向を,圧縮という手段で考慮する.
このようにして定義されたアトラクターの次元$D_L$は$m$から$m+1$の間にある
($m$は,$L_{m+1}$を含むとき,その和がまだ正であるが負となるインデックスの数である)
\footnote{ゼロであるLyapunov指数を含むと,経路に沿った次元$D_L$が1増える.}.
$D_L=m+d \;(d<1)$の分数部は,以下の式で求められる(F. Ledrappier 1981).
\begin{equation}
    \sum_{s=1}^m L_s + L_{m+1}d = 0
\end{equation}
$d$の計算では,最も安定でない方向のみを考慮する
(数列の最後にある,絶対値が最も大きい負の$L_s$を省く)ため,次元の推定値$D_L$は一般に高すぎる.
この推定値は,原理的には,乱流の速度揺らぎの時間依存性の測定からアトラクターの次元を決定する方法を提供するものである.




\BackToTheToc