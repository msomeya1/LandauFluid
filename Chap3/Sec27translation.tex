\section{\spade 回転流の安定性(日本語訳)}
非常に大きなReynolds数の極限における,2つの回転円筒間の定常流(\S~18)の安定性を調べるには,
\S~4で重力場中の静止流体の力学的安定条件(Rayleigh 1916)を導いたのと同様の手法を用いることができる.
この方法の原理は,流体の任意の小さな要素を考え,この要素が当該流れの中でたどる経路から変位すると仮定することである.
この変位の結果,変位した要素に作用する力が現れる.元の流れが安定していれば,これらの力はその要素を元の位置に戻そうとするはずである.


乱れのない流れの中の各流体要素は,円筒の軸を中心とする円$r=\const$の中を移動する.
$\mu(r)=mr^2\dot{\phi}$を,質量$m$,角速度$\dot{\phi}$の要素の角運動量とする.
これに働く遠心力は$\mu^2/mr^3$であり,この力は回転する流体の半径方向の圧力勾配と釣り合う.
ここで,軸から距離$r_0$にある流体要素がその軌道からわずかにずれて,軸から距離$r>r_0$に移動したとしよう.
このとき,要素の角運動量は元の値$\mu_0 = \mu(r_0)$に等しいままである.
従って,その新しい位置で要素に作用する遠心力は${\mu_0}^2/mr^3$である.
要素がその初期位置に戻ろうとするためには,この力は距離$r$での圧力勾配と釣り合う平衡値$\mu^2/mr^3$より小さくなければならない.
したがって,安定のための必要条件は$\mu^2-{\mu_0}^2>0$である.
$\mu(r)$を$r-r_0$の冪で展開して
\begin{equation}
    \mu \dv{\mu}{r} > 0
\end{equation}


式(18.3)より,移動する流体粒子の角速度$\dot{\phi}$は
\[
    \dot{\phi} = \frac{ \Omega_2{R_2}^2 - \Omega_1 {R_1}^2 }{ {R_2}^2 - {R_1}^2 }
    + \frac{ (\Omega_2-\Omega_1){R_1}^2{R_2}^2  }{ {R_2}^2 - {R_1}^2 } \frac{1}{r^2}
\]
となる.
$\mu=mr^2\dot{\phi}$を計算し,確実に正となる因子を省くと,条件(27.1)は次のように書くことができる.
\begin{equation}
    (\Omega_2{R_2}^2 - \Omega_1 {R_1}^2) \dot{\phi} > 0
\end{equation}


角速度$\dot{\phi}$は内円筒の$\Omega_1$から外円筒の$\Omega_2$まで単調に変化する.
2つの円筒が反対方向に回転する場合,すなわち$\Omega_1$と$\Omega_2$が逆符号の場合,関数$\dot{\phi}$は円筒間で符号が変わり,
定数$\Omega_2{R_2}^2 - \Omega_1 {R_1}^2$との積はどこでも正になることはありえない.
したがってこの場合,(27.2)は流体中のすべての点で成立せず,流れは不安定になる.


ここで,2本の円柱を同じ方向に回転させ,この回転方向を正とすると,$\Omega_1>0, \Omega_2>0$となる.
このとき$\dot{\phi}$はどこでも正であり,条件(27.2)を満たすには,次のようにすればよい.
\begin{equation}
    \Omega_2{R_2}^2 > \Omega_1 {R_1}^2
\end{equation}
$\Omega_2{R_2}^2 < \Omega_1 {R_1}^2$ならば流れは不安定になる.
たとえば,外側の円柱が静止していて($\Omega/2=0$),内側の円柱が回転している場合,流れは不安定になる.
一方,内側の円柱が静止している場合($\Omega_1=0$),流れは安定である.


ここで強調したいのは,上記の議論では流体要素が変位するときの粘性力の影響が考慮されていないことである.
したがって,この方法は粘性が小さい場合,すなわち$\R$が大きい場合にのみ適用できる.


任意の$\R$に対する流れの安定性を調べるには,(26.4)式から始める一般的な方法が必要であり,
回転円筒間の流れについてはG. I. Taylor(1924)によって初めて行われた.
この場合,乱れのない速度分布$\v_0$は(円筒)径方向座標$r$にのみ依存し,角度$\phi$や軸方向座標$z$には依存しない.
したがって,式(26.4)の独立解の完全セットは,以下のような形で求めることができる.
\begin{equation}
    \v_1(r,\phi,z) = e^{i(n\phi+kz-\omega t)} \vec{f}(r)
\end{equation}
ベクトル$\vec{f}(r)$の方向は任意である.
波数$k$は連続的な値をとり,$z$方向の摂動の周期性を決定する.
数$n$は整数値$0,1,2, \ldots $のみをとる.
から導かれるように$n=0$は軸対称の摂動に対応する.
周波数の許容値は,必要な境界条件($r=R_1$と$r=R_2$では$\v_1=\vec{0}$)で方程式を解くことによって得られる.
このようにして定式化された問題は,与えられた$n$と$k$に対して,一般に固有振動数$\omega=\omega_n^{(j)}(k)$の不連続系列をもたらすが,
ここでは関数$\omega=\omega_n(k)$の分岐を表し,これらの振動数は一般に複素数となる.



この場合のReynolds数は,流れの種類を決定する比$R_1/R_2$と$\Omega_1/\Omega_2$が与えられた値に対して
$\Omega_1{R_1}^2/\nu$または$\Omega_2{R_2}^2/\nu$で表される.
Reynolds数の増加に伴う固有振動数$\omega=\omega_n^{(j)}(k)$の変化を追ってみよう.
$\R<\Rcr$では関数$\gamma(k)$は常に負であるが,$\R>\Rcr$では$k$のある範囲で$\gamma>0$となる.
$\R=\Rcr$のとき$\gamma(k)=0$となる$k$の値を$k_\mathrm{cr}$とする.
これに対応する関数(27.4)は,元の流れが安定でなくなった瞬間に流体中に生じる(元の流れに重なる)流れの性質を与えており,
それは円柱の軸に沿って周期$2\pi/k_\mathrm{cr}$を持つ.
実際の安定性の限界は,もちろん摂動の形,すなわち$\Rcr$が最小となる関数$\omega_n^{(j)}(k)$によって決定され,
ここで注目すべきはこれらの「最も危険な」摂動である.
原則として(下記参照),これらは軸対称である.
計算が非常に複雑になるため,円柱と円柱の間の空間が狭い場合のみ,かなり完全な研究がなされてきた.
$h \equiv R_2-R_1 \ll R = (R_1+R_2)/2 $のように,円筒の間隔が狭い場合のみ,かなり完全な研究がなされている.
その結果は次のようなものである
\footnote{
詳しい説明は
\begin{itemize}
    \item N. E. Kochin, I. A. Kibel' and N. V. Roze, \textit{Theoretical Hydromechanics}, Part 2, Moscow 1963
    \item S. Chandrasekhar, \textit{Hydrodynamic and Hydromagnetic Stability}, Oxford 1961
    \item P. G. Drazin and W. H. Reid, \textit{Hydrodynamic Stability}, Cambridge 1981
\end{itemize}
にある.
}.




純虚数関数$\omega(k)$が最小の$\Rcr$を与える解に対応することがわかった.
したがって,$k=k_\mathrm{cr}$のとき,$\Im(\omega)$だけでなく$\omega$自体もゼロになる.
これは,定常回転流の最初の不安定性が,同じく定常である別の流の出現につながることを意味する
\footnote{このような場合,\emph{安定性の交換}と呼ばれる.
いくつかの特殊なケースについての実験と数値計算の結果は,この性質が考慮された流れに対して一般的なものであり,
$h$が小さいことに依存しないことを示唆している.}.
円筒に沿って規則正しく並んだトロイダルな\emph{Taylor渦}からなる.
2本の円柱が同じ方向に回転する場合,図14はこれらの渦の流線を円柱の子午面断面上に投影したものであり,速度$\v_1$は実際には方位角成分をも持っている.
各周期の長さ$2\pi/k_\mathrm{cr}$には,回転方向が反対の2つの渦が存在する.



$\Rcr$より少し大きい$\R$では,$k$の値は1つではなく,$\Im(\omega)>0$となるような範囲が存在する.
ただし,得られる流れは,さまざまな周期性を持つ流れの重ね合わせになるとは考えない方がよい.
実際には,それぞれの$\R$に対して,ある一定の周期性を持った流れが発生し,全体の流れが安定化する.
しかし,この周期性は線形化された式(26.4)からは求めることができない.



図15は,$R_1/R_2$が与えられたときの不安定領域(斜線)と安定領域を分ける曲線の概形である.
曲線の右側の枝は,2つの円柱が同じ方向に回転した場合に対応し,直線$\Omega_2{R_2}^2=\Omega_1{R_1}^2$に漸近する.
この性質は,実際には$h$の小ささに依存しない一般的なものである.
Reynolds数が増加すると,与えられたタイプの流れに対して,与えられた$\Omega_1/\Omega_2$の値に対応する原点を通る直線に沿って上方へと移動していく.
図の右側では,$\Omega_2{R_2}^2/\Omega_1{R_1}^2>1$となるような線は,不安定領域の境界である曲線に合致せず,不安定領域の境界となる.
一方,$\Omega_2{R_2}^2/\Omega_1{R_1}^2<1$ならば,(27.3)の条件に従って,十分大きなReynolds数で不安定領域に入っていくことになる.
図の左側($\Omega_1$と$\Omega_2$が逆符号)では,原点を通る任意の線が斜線領域の境界を通る.
つまり,Reynolds数が十分に大きいときには,どのような比率$\abs{\Omega_2/\Omega_1}$でも結局は定常流が不安定になることがわかり,やはりこれまでの結果と一致している.
$\Omega_2=0$の場合(内筒だけが回転する場合),$\R=h\Omega_1R_1/\nu$で定義されるReynolds数が次のようになると不安定となる.
\begin{equation}
    \Rcr = 41.2 \sqrt{R/h}
\end{equation}



考察している流れでは,粘性は安定化効果を持ち,
$\nu=0$で安定な流れは粘性を考慮すると安定なままであり,不安定なものが粘性流体に対して安定になることもある.


回転円筒間の流れにおける軸対称性のない摂動については,これまで体系的な研究がなされていない.
特定のケースについて計算した結果,図15の右側では,軸対称の摂動が常に最も危険であることに変わりはないことが示唆された.
しかし,左側では,$\abs{\Omega_2/\Omega_1}$が十分に大きい場合,境界曲線の形は
が十分に大きい場合,軸対称性のない摂動を考慮すると境界曲線の形が多少変化することがある.
その場合,摂動周波数の実部がゼロにならないので,結果として流れが定常でなくなり,不安定性の性質がかなり変わってくる.

回転する円筒間の流れの極限的な場合($h\to 0$)は,相対運動する2つの平行な平面の間の流れである(\S~17参照).
この流れは,$\R=uh/\nu$($u$は平面の相対速度)のいかなる値に対しても,無限に小さな摂動に対して安定である.




\BackToTheToc