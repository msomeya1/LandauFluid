\section{\spade 周期倍分岐による乱流への遷移(日本語訳)}

ここで,乗数が$-1$または$+1$を通過するときの,周期流の安定性の喪失について考えてみよう.



$n$次元の状態空間において,$n-1$個の乗数は,考えている周期的経路に近い$n-1$個の異なる方向
(その経路の各点での接線の方向と同じではない)の経路の振る舞いを決定する.
$\pm 1$付近の乗数が$l$番目の方向だとしよう.
他の$n-2$個の乗数は絶対値が小さいので,対応する$n-2$方向のすべての経路は,時間の経過とともに$l$番目の方向と接線の方向を含む二次元表面$\Sigma$に近づいていくことになる.
リミットサイクルの近くでは,状態空間は$t\to\infty$でほぼ2次元になると言える
(経路は$\Sigma$のどちらの側にもあり,一方の側から他方の側に行くこともあるので,厳密には2次元にはなり得ない).
$\Sigma$付近の経路のフラックスを面$\sigma$で切るとする.
各経路が$\sigma$を繰り返し通過するとき,最初の交点$\vec{x}_j$に従って,次の交点$\vec{x}_{j+1}$を決定する.
関係式$\vec{x}_{j+1}=f(\vec{x}_j ;\R)$
は\emph{Poincar\'{e}写像}または\emph{シーケンス写像}と呼ばれ,$\R$,この場合はReynolds数
\footnote{熱対流の場合はRayleigh数である(\S~56).}に依存する.
その値は,周期流が安定でなくなる分岐点に近いかどうかを決める.
すべての経路は$\Sigma$に近いので,$\sigma$に出会う点の集合はほぼ一次元で直線で近似できる
するとPoincar\'{e}写像は一次元変換
\begin{equation}\label{eq32.1:1次元Poincare写像}
    x_{j+1}=f(x_j ;\R)
\end{equation}
となり,$x$は単に直線に沿った座標となる\footnote{この節ではもちろん$x$は物理空間での座標とは関係ない.}.
離散変数$j$は,周期を単位として測った時間の役割を担う.




写像\eqref{eq32.1:1次元Poincare写像}は,分岐点付近の流れの性質を決定する別の方法を提供する.
周期流そのものは,写像\eqref{eq32.1:1次元Poincare写像}の\emph{固定点},つまり写像によって変化しない値$x_j=x_*$,すなわち,$x_{j+1}=x_j$に対応するものである.
乗数は$x_j=x_*$における微分$\mu=\ddv{x_{j+1}}{x_j}$で,$x_*$付近の点$x_j=x_*+\xi$は$x_{j+1}\simeq x_* + \mu\xi$に写像される.
$\abs{\mu}<1$の場合,固定点は安定であり,写像のアトラクターである;
$x_*$付近のいくつかの点から始めて写像を反復することにより,$\abs{\mu}^r$の形で漸近的に後者に近づく($r$は反復の数).
しかし,$\abs{\mu}>1$の場合,固定点は不安定になる.




ここで,乗数が$-1$を通過するときの周期流の安定性の喪失について考えてみよう.
式$\mu=-1$は,初期の摂動が時間$T_0$後に同じ大きさのまま符号を変え,さらに時間$T_0$の後に元の値に戻ることを意味する.
したがって,周期$T_0$のリミットサイクルの近くで$\mu$が$-1$を通過すると,周期$2T_0$の新しいリミットサイクルが発生する(\emph{周期倍分岐})
\footnote{本節では,基本周期(最初の周期流の周期)を$T_1$ではなく$T_0$と表す.
連続する周期倍分岐に対応する臨界Reynolds数は,ここでは$\R_1, \R_2, \ldots$と表し,添字crは付けない($\R_1$は以前の$\R_\mathrm{cr,2}$を置き換える).}.
図20は,このような2つの連続した分岐を従来の方法で表したもので,
図a,bの連続した曲線は安定リミットサイクル$2T_0$と$4T_0$を,破線の曲線は不安定になったリミットサイクルを示している.



Poincar\'{e}写像の固定点を任意に$x=0$とすると,その近傍で周期倍分岐を記述する写像は次の展開で表される.
\begin{equation}\label{eq32.2:周期倍分岐を記述する写像}
    x_{j+1} = -[1+(\R-\R_1)]x_j + {x_j}^2 + \beta {x_j}^3
\end{equation}
ここで$\beta>0$である
\footnote{$\R-\R_1$の係数は$\R$を適切に再定義することで1に,${x_j}^2$の係数は$x_j$を再定義することで+1にできるので,\eqref{eq32.2:周期倍分岐を記述する写像}ではそれが行われたものとする.}.
$\R<\R_1$では固定点$x_*=0$は安定で,$\R>\R_1$では不安定である.
周期倍分岐がどのように起こるかを見るためには,写像\eqref{eq32.2:周期倍分岐を記述する写像}を2回反復しなければならない.
つまり,2ステップ(2単位時間)後を考え,再形成された写像の固定点を決定する必要がある;
もしこれらが存在し安定なら,それらは周期倍分岐サイクルに対応する.



写像\eqref{eq32.2:周期倍分岐を記述する写像}を2回反復すると,(小さな量$x_j$と$\R-\R_1$に関して必要な精度で)写像
\begin{equation}\label{eq32.3:周期倍分岐の写像の反復}
    x_{j+2} = x_j +2(\R-\R_1)x_j -2(1+\beta) {x_j}^3
\end{equation}
が得られる.
これは常に固定点$x_*=0$を持つ.
$\R<\R_1$のとき,この点以外に固定点はなく,安定で,乗数は$\abs{\ddv{x_{j+2}}{x_j}}<1$である;
周期1($T_0$を単位とする)の流れに対して,時間間隔2も1周期である.
$\R=\R_1$のとき乗数は$+1$であり,$\R>\R_1$のとき点$x_*=0$は不安定になる.
その段階で一組の安定固定点が形成される;
\begin{equation}
    {x_*}^{(1),(2)} = \pm \sqrt{ \frac{\R-\R_1}{1+\beta} }
\end{equation}
これは2倍周期の安定リミットサイクルに相当する
\footnote{単に2サイクルと呼ぶ.関連する固定点は\emph{サイクル要素}と呼ぶ.}.
写像\eqref{eq32.3:周期倍分岐の写像の反復}は各点をそのまま残すが,\eqref{eq32.2:周期倍分岐を記述する写像}は一方を他方に変える.
ここで強調しておかなければならないのは,1周期サイクルはこの分岐で消滅するのではなく,運動方程式の(不安定な)解として残るということである.




分岐点付近では,運動はまだ周期1の「ほぼ周期的」なもので,経路が戻る点${x_*}^{(1)}$と${x_*}^{(2)}$は接近している.
その間の間隔${x_*}^{(1)}-{x_*}^{(2)}$は,周期2の振動の振幅の指標であり,
定常流が不安定になる地点で周期流が始まった後の振幅の増加(26.10)と同様に,$\sqrt{\R-\R_1}$で増加する.



周期倍分岐の繰り返しは,乱流の形成に至る一つのルートである.
このシナリオでは分岐の数は無限であり,分岐は($\R$が大きくなるにつれて)どんどん間隔を狭めて続いていく.
臨界値の列$\R_1, \R_2, \ldots$は有限の極限値に近づき,それを超えると周期性が完全に消失し,
複雑な非周期的アトラクターが空間内に形成され,このシナリオでの乱流形成に関係している.
このシナリオは,普遍性とスケール不変性という注目すべき性質を持っていることがわかるだろう(M. J. Feigenbaum 1978)
\footnote{周期倍分岐の列(以下,$1,2,\ldots$と番号をつける)は,周期流の最初の分岐から始まる必要はない.
原理的には,最初の数回の分岐の後,\S~30で述べたメカニズムによって非整合周波数がロックされたときに始まるかもしれない.}.



以下に示す定量的な理論は,分岐が($\R$の増加とともに)非常に速く互いに続くため,
分岐と分岐の間でも,状態空間の経路の集合が占める領域はほぼ2次元のままであり,
一連の分岐は単一のパラメータに依存する1次元Poincar\'{e}写像で記述できるという仮説から出発するものである.




以下で用いる写像の選択は次のように正当化される.
$x$の変動範囲のかなりの部分で,写像は$\abs{\ddv{f(x,\lambda)}{x}}>1$の伸長写像でなければならず,これによって不安定性が発生する.
また,写像は,ある範囲から出た経路をある範囲に戻すものでなければならず,そうでなければ速度変動が際限なく増大することになり,これは不可能である.
この2つの要求を同時に満たすことができるのは,非単調関数$f(x;\lambda)$,つまり一対一ではない写像\eqref{eq32.1:1次元Poincare写像}のみであり,
$x_{j+1}$の値は前の$x_j$によって一意に決まるがその逆はありえない.
このような関数の最も単純なものは,一つの最大値を持ち,その近くで
\begin{equation}\label{eq32.5:1次元写像1-λx^2}
    x_{j+1} = f(x_j; \lambda) = 1-\lambda {x_j}^2
\end{equation}
と書ける.
ここで$\lambda$は正のパラメータで,(流体力学的に)$\R$の増加関数とみなされる
\footnote{一対一でない写像の許容性は,一次元の処理の近似性に依存する.
もし,すべての経路が一つの曲面$\Sigma$上に正確に存在し,Poincar\'{e}写像が厳密に一次元なら,
異なる$x_j$を持つ二つの経路が$x_{j+1}$で交差することになり,この非一意性は不可能になる.
同じ意味で,写像の固定点が写像関数の極限にある場合,乗数が0になる可能性は近似性に起因する.
そのような点は「超安定」と表現され,上記の関係によるよりも速く接近することができる.}.
ここでは,$x$の変動範囲を任意に$[-1,+1]$にとることにする.
$\lambda$が0と2の間にあるとき,写像\eqref{eq32.5:1次元写像1-λx^2}のすべての反復は$x$をその範囲に残す.



写像\eqref{eq32.5:1次元写像1-λx^2}の固定点は$x_*=1-\lambda {x_*}^2$の解である.
これは$\lambda>\Lambda_1$のときに不安定になる.
ここで,$\Lambda_1$は乗数が$\mu=-2\lambda x_*=-1$となる$\lambda$の値で,2つの方程式から$\Lambda_1=3/4$を得る.
これが$\lambda$の最初の臨界値であり,最初の周期倍分岐の位置と2サイクルの出現を決定する.
ここでは,プロセスのいくつかの定性的な特徴を決定する近似的な手法によって,その後の分岐の出現を追跡してみよう.
この方法では特性定数の正確な値は得られないが,正確な記述はその後に行う.




写像\eqref{eq32.5:1次元写像1-λx^2}を繰り返すと次のようになる.
\begin{equation}\label{eq32.6:写像1-λx^2の2回反復}
    x_{j+2} = 1 -\lambda +2\lambda^2{x_j}^2 -\lambda^3{x_j}^4
\end{equation}
ここでは,${x_j}^4$の項を無視することにする.
残りの式は,スケール変換
\[
    x_j \to \frac{x_j}{\alpha_0}, \quad \alpha_0 = \frac{1}{1-\lambda}
\]
によって
\footnote{$\lambda=1$のときは不可能である(写像\eqref{eq32.6:写像1-λx^2の2回反復}の固定点が中心極値と一致する:$x_*=0$).
しかし,$\lambda=1$は,ここで必要とされる次の臨界値$\lambda_2$ではないことは確かである.}
\[
    x_{j+2} = 1-\lambda_1 {x_j}^2
\]
という形になる.
\eqref{eq32.5:1次元写像1-λx^2}と異なる点は,$\lambda$が
\begin{equation}\label{eq32.7:λ1の定義}
    \lambda_1 = \phi(\lambda) \equiv 2\lambda^2(\lambda-1)
\end{equation}
のように置き換わっただけである.
この操作をスケールファクター$\alpha_1=\dfrac{1}{1-\lambda_1}$等を用いて繰り返すと,同じ形の写像の列が得られる.
\begin{equation}\label{eq32.8:写像1-λx^2のm回反復}
    x_{j+2^m} = 1-\lambda_m {x_j}^2, \quad \lambda_m = \phi(\lambda_{m-1})
\end{equation}
写像\eqref{eq32.8:写像1-λx^2のm回反復}の固定点は$2^m$サイクルに対応する
\footnote{誤解を避けるために強調しておくと,スケール変換後の写像\eqref{eq32.8:写像1-λx^2のm回反復}は
$\abs{x} \leq \abs{ \alpha_0 \alpha_1 \cdots \alpha_{m-1} }$の拡張範囲に定義されなければならず,
\eqref{eq32.5:1次元写像1-λx^2},\eqref{eq32.6:写像1-λx^2の2回反復}のように$\abs{x}\leq 1$ではない.
しかし,無視された項を考慮すると,\eqref{eq32.8:写像1-λx^2のm回反復}は実際には写像関数の中心極値付近の範囲の記述しか与えることができない.}.
これらはすべて\eqref{eq32.5:1次元写像1-λx^2}と同じ形なので,$\lambda_m=\Lambda_1=3/4$のとき,$2^m$サイクル($m=1,2,3, \ldots$)は不安定になることがすぐに推測できる.
初期パラメータ$\lambda$の臨界値$\Lambda_m$は,以下の連立方程式を解くことで得られる.
\[
    \Lambda_1 = \phi(\Lambda_2), \;
    \Lambda_2 = \phi(\Lambda_3), \; \ldots ,
    \Lambda_{m-1} = \phi(\Lambda_m)
\]
これらは,図21のような構成でグラフィカルに得られたものである.
明らかに,$m\to\infty$で数列は$\Lambda_\infty=\phi(\Lambda_\infty)$の解である有限の極限$\Lambda_\infty$に収束し,
値は$\Lambda_\infty=\dfrac{1+\sqrt{3}}{2} = 1.37$である.
スケールファクターも有限の極限に近づく:$\alpha_m \to \alpha=\dfrac{1}{1-\Lambda_\infty} = -2.8$.




$m$が大きいとき,$\Lambda_m$が$\Lambda_\infty$にどのように近づくかは簡単にわかる.
$\Lambda_\infty - \Lambda_m$が小さいとき,式$\Lambda_m = \phi(\Lambda_{m+1})$から
\begin{equation}\label{eq32.9:Λ∞への幾何級数的収束}
    \Lambda_\infty - \Lambda_{m+1} = \frac{\Lambda_\infty - \Lambda_m}{\delta} .
\end{equation}
ここで$\delta = \phi'(\Lambda_\infty) = 4+\sqrt{3} = 5.73$である.
したがって,$\Lambda_\infty - \Lambda_m \propto \delta^m$,すなわち,$\Lambda_m$は幾何級数的に極限値に近づく.
連続する臨界数の間隔についても同じ関係が成り立つ.
式\eqref{eq32.9:Λ∞への幾何級数的収束}は等価な形式で書くことができる.
\begin{equation}\label{eq32.10:Λmの幾何級数的収束}
    \Lambda_{m+2} - \Lambda_{m+1} = \frac{\Lambda_{m+1} - \Lambda_m}{\delta}
\end{equation}



流体力学に関しては,$\lambda$がReynolds数の関数とみなされることは既に述べたとおりであり,
したがって,$\R$には連続した周期倍分岐に対応する臨界値があり,有限の極限値$\R_\infty$に至る傾向がある.
これらの値に対しては,$\Lambda_m$の場合と同じ定数$\delta$の極限関係\eqref{eq32.9:Λ∞への幾何級数的収束},\eqref{eq32.10:Λmの幾何級数的収束}が成り立つことは明らかである.


%%%


以上の議論から,この過程の基本的な特徴,すなわち,分岐の無限性,その出現時刻が\eqref{eq32.9:Λ∞への幾何級数的収束},\eqref{eq32.10:Λmの幾何級数的収束}に従って極限$\Lambda_\infty$に収束すること,およびスケールファクター$\alpha$の存在が説明できる.
しかし,このようにして求めた特性定数の値は厳密なものではない.
収束係数(\emph{Feigenbaum数})$\delta$とスケールファクター$\alpha$の正確な値(写像\eqref{eq32.5:1次元写像1-λx^2}をコンピュータで反復して求めたもの)は
\begin{equation}
    \delta = 4.6692\cdots, \quad
    \alpha = -2.5029\cdots 
\end{equation}
であり,極限値は$\Lambda_\infty=1.401$となる
\footnote{$\Lambda_\infty$の値は,初期写像でのパラメータの使い方,つまり関数$f(x;\lambda)$に依存するので多少任意だが,$\delta,\alpha$の値はこれに依存しない.}.
$\delta$の値は比較的大きく,収束が早いため,わずかな周期の倍増で極限関係がほぼ満たされる結果となる.



上記の導出の欠点は,${x_j}^2$の1乗以上を無視した場合,写像\eqref{eq32.8:写像1-λx^2のm回反復}は次の分岐が起こるという事実だけをもたらし,
この写像が記述する$2^m$サイクルのすべての要素を決定することができない点である
\footnote{
すなわち,写像\eqref{eq32.5:1次元写像1-λx^2}を反復したとき,$2^m$個の点${x_*}^{(1)},{x_*}^{(2)}, \ldots$は連続して互いに変化し(周期的で),
$2^m$倍の反復写像に対して固定(安定化)されている.
疑惑を避けるために,導関数$\ddv{x_{j+2^m}}{x_j}$はすべての点${x_*}^{(1)},{x_*}^{(2)}, \ldots$で必ず同じであることに注意する必要がある
(したがって次の分岐では同時に$-1$を通る).
この性質の証明はここではしないが,明らかに必要である.
}.
実際には,反復写像\eqref{eq32.5:1次元写像1-λx^2}は${x_j}^2$の多項式で,その次数は反復ごとに2倍となる.
これは,$x_j=0$(これも常に極値である)を中心に対称に存在する極値の数が急激に増加する$x_j$の複雑な関数である.



注目すべきは,$\delta,\alpha$の値だけでなく,無限に反復される写像の極限形も,ある意味で,初期写像$x_{j+1}=f(x_j;\lambda)$の形に依存しないことである;
1つのパラメータを持つ関数$f(x;\lambda)$が,滑らかで,2次の最大値(その位置を$x=0$とする)を持つというだけで十分である.
最大値から大きく離れたところで,最大値に対して対称である必要はない.
この\emph{普遍性}は,説明した理論の一般性をかなり高めてくれる.
この性質を厳密に定式化すると次のようになる.





ここで,$f(x)$で指定された写像を考えよう.
すなわち,$\lambda$を特別に選ぶことにより(後述),$f(x;\lambda)$を$f(0)=1$という条件で正規化したものを考えよう.
この関数を2回適用すると,関数$f[f(x)]$を得る.
この関数と$x$のスケールをファクター$\alpha_0 = 1/f(1)$だけ変えて,新しい関数
\[
    f_1(x) = \alpha_0 f[f(x/\alpha_0)]  
\]
が得られ,やはり$f_1(0)=1$となる.
この操作を繰り返すと,次の漸化式で結ばれた関数列が得られる
\footnote{この手順は,前に\eqref{eq32.8:写像1-λx^2のm回反復}の導出に用いた方法と明らかに類似している.}.
\begin{equation}\label{eq32.12:fm(x)の反復写像}
    f_{m+1}(x) = \alpha_m f_m[f_m(x/\alpha_m)] \equiv \hat{T} f_m, \quad
    \alpha_m = 1/f_m(1)
\end{equation}
この数列が$m\to\infty$で一定の極限関数$f_\infty(x)\equiv g(x)$に近づくならば,
$g$は\eqref{eq32.12:fm(x)の反復写像}で定義される演算子$\hat{T}$の「固定関数」,すなわち関数関係
\begin{equation}\label{eq32.13:g(x)の反復写像}
    g(x) = \hat{T} g \equiv \alpha g[g(x/\alpha)], \quad
    \alpha = 1/g(1), \quad g(0) = 1
\end{equation}
を満足しなければならない.
許容関数$f(x)$の性質から,$g(x)$は滑らかで,$x=0$で二次の極値を持つ必要がある.
$f(x)$の形式は式\eqref{eq32.13:g(x)の反復写像}やその解に課される条件に影響しない.
導出で用いたスケール変換($\abs{\alpha_m}>1$)の後,方程式の解は変数$x$のすべての値,
($-1 \leq x \leq 1$ではなく)$-\infty$から$+\infty$の間で決定されることを強調しなければならない.
また,関数$g(x)$は必然的に偶関数である.
なぜなら,許容関数$f(x)$には偶関数が含まれ,偶関数写像は何度反復しても確実に偶関数のままだからである.




式\eqref{eq32.13:g(x)の反復写像}のそのような解は,解析的な形では導けないが,実際に存在し,一意である.
それは極値を無限に持ち,大きさが無限の関数で,定数$\alpha$は$g(x)$と共に決定される.
実用上は,この関数は$[-1,1]$の範囲で導けば十分であり,その後は$\hat{T}$の演算を繰り返すことによって範囲外でも続けることができる.
\eqref{eq32.12:fm(x)の反復写像}の$\hat{T}$の反復の各段階で,
範囲$[-1,1]$における$f_{m+1}(x)$の値は,この範囲の一部を係数$\abs{\alpha_m}\simeq \abs{\alpha}$で短縮した$f_m(x)$の値で決定されていることに注意されたい.
これは,多くの反復の極限において,範囲$[-1,1]$(したがって$x$軸全体)における$g(x)$の決定は,初期関数のその最大値の近くのより小さい部分によって支配されており,ここに普遍性の究極の原因がある
\footnote{
方程式\eqref{eq32.13:g(x)の反復写像}の一意解が存在するという主張は,コンピュータシミュレーションによって成り立っている.
解は$[-1,1]$の範囲では$x^2$の高次の多項式として求められ,シミュレーションの精度は$\hat{T}$の反復によって関数を継続したい$x$値の範囲(その外)の幅とともに高くなるはずである.
$[-1,1]$の範囲では$g(x)$には極値があり,その付近では$g(x)=1-1.528x^2$となる.
もしこれが最大なら,これは$g$の符号を変えても式\eqref{eq32.13:g(x)の反復写像}が不変なことから任意に選択できる.
%(ただし,式\eqref{eq32.13:g(x)の反復写像}の符号を変えると極限は変化する)
}.


%%%%%%%

関数$g(x)$は無限回の周期倍分岐によって形成される非周期的なアトラクターの構造を決定する.
無限回の分岐は,関数$f(x;\lambda)$に対して非常に明確なパラメータ値$\lambda=\Lambda_\infty$で発生する.
したがって,写像\eqref{eq32.12:fm(x)の反復写像}の反復によって$f(x;\lambda)$から形成される関数は,実際にはこの孤立した$\lambda$の値に対してのみ$g(x)$に収束することは明らかである.
このことから,演算子$\hat{T}$の固定関数は,$\lambda$の$\Lambda_\infty$からの小さなずれに対応する小さな変化に関して不安定であることがわかる.
この不安定性を研究することによって,やはり$f(x)$の特定の形式に依存しない普遍定数$\delta$を決定することができる
\footnote{
原論文
M. J. Feigenbaum, \textit{Journal of Statistical Physics} \textbf{19}, 25, 1978; \textbf{21}, 669, 1979
参照.}.



スケールファクター$\alpha$は,周期が2倍になる際のアトラクターの(状態空間における)幾何学的特性の変化(減少)を決めるもので,
この特性は$x$軸上のリミットサイクル要素間の距離である.
ただし,倍増のたびにサイクル要素の数はさらに増えるので,この記述はより具体的かつ正確なものにしなければならない.
すべての点間の距離について,スケールが同じように変化することができないことは,先験的に明らかである
\footnote{これらは伸縮しない範囲$[-1,1]$の距離であり,最初からすべてのサイクル要素を含む$x$の範囲として任意に設定されたものである.
$\alpha$は負であるから,分岐は$x=0$に対する要素の位置の反転を伴う.}.
例えば,隣接する2点が写像関数のほぼ直線的な部分によって変換される場合,2点間の距離は係数$\abs{\alpha}$だけ減少するが,
写像関数の極値付近の部分によって変換される場合,その距離は係数$\alpha^2$だけ減少する.



分岐点($\lambda=\Lambda_m$)で$2^m$周期の各要素(点)は隣接する2点に分裂し,
その距離は徐々に長くなるが,次の分岐点までの$\lambda$の変動範囲では各点は接近したままである.
時間の経過とともに,つまり連続した写像$x_{j+1}=f(x_j;\lambda)$において,
サイクルの要素が互いに変換されるのを追うと,ペアの各成分は$2^m$単位時間後に他の成分に変化する.
このことは,ペアになった点間の距離が,新しく形成される2倍周期の振動振幅の指標となることを意味し,この意味で特に物理的な興味がある.



$2^{m+1}$サイクルのすべての要素を,時間の経過とともに通過する順に並べ,$x_{m+1}(t)$で表すと,
基本周期$T_0$を単位とする時間$t$は,$t/T_0 = 1,2,\ldots,2^{m+1}$の整数値をとる.
これらの要素は$2^m$周期の要素がペアに分かれて形成される.
各ペアの点間の間隔は
\begin{equation}\label{eq32.14:2^m+1サイクルのペア間の間隔}
    \xi_{m+1}(t) = x_{m+1}(t) - x_{m+1}(t+T_m)
\end{equation}
であり,ここで$T_m=2^mT_0=\dfrac{1}{2}T_{m+1}$は$2^m$サイクルの周期,
つまり$2^{m+1}$サイクルの周期の半分である.
\eqref{eq32.14:2^m+1サイクルのペア間の間隔}の区間が1サイクルから次のサイクルへ移るときの変化を決めるスケールファクターとして関数$\sigma_m(t)$を用いよう
\footnote{
2サイクルは,$\lambda$の値の異なる範囲$(\Lambda_{m-1},\Lambda_m)$と$(\Lambda_m,\Lambda_{m+1})$に存在し,
これらの範囲で量\eqref{eq32.14:2^m+1サイクルのペア間の間隔}はかなり変化するので,
定義\eqref{eq32.15:スケールファクターσm(t)の定義}におけるその意味をより正確にする必要がある.
ここでは,サイクルが超安定となる$\lambda$の値(\eqref{eq32.5:1次元写像1-λx^2}の脚注参照)をとることとし,
そのような値は,各サイクルが存在する範囲に一つ存在する.}:
\begin{equation}\label{eq32.15:スケールファクターσm(t)の定義}
    \frac{\xi_{m+1}(t)}{\xi_m(t)} = \sigma_m(t)
\end{equation}
明らかに
\begin{equation}
    \xi_{m+1}(t+T_m) = -\xi_{m+1}(t)
\end{equation}
となり,したがって
\begin{equation}
    \sigma_m(t+T_m) = -\sigma_m(t)
\end{equation}
である.



関数$\sigma_m(t)$は複雑な性質を持つが,$m$が大きいときの極限形は次の簡単な式で非常によく近似される.
\begin{equation}
    \sigma_m(t) = 
    \begin{cases}
        1/\alpha \quad \pqty{ 0<t<\frac{1}{2}T_m } \\[7pt]
        1/\alpha^2 \quad \pqty{ \frac{1}{2}T_m<t<T_m }
    \end{cases}
\end{equation}
ただし$t$の原点は適切に選ぶものとする
\footnote{
$\sigma_m(t)$の性質の研究については,原理的には簡単だが手間がかかるので,ここでは触れない.
M. J. Feigenbaum, \textit{Los Alamos Science} \textbf{1}, 4, 1980.
参照.}.



これらの式から,周期倍分岐が起こったときの流れの周波数スペクトルの変化について,いくつかの結論が得られる.
流体力学的には,$x_m(t)$は流体速度の特性としてとらえることができる.
周期$T$の流れの場合,連続時間$t$の関数$x_m(t)$のスペクトルは,周波数$k\omega_m$($k=1,2,3,\ldots$),
すなわち基本周波数$\omega_m=2\pi/T_m$とその高調波を含んでいる.
周期倍分岐のあとは,流れは周期$T_{m+1}=2T_m$の関数$x_{m+1}(t)$によって記述される.
そのスペクトルには,同じ周波数$k\omega_m$だけでなく,
$\omega_m$の低調波である周波数$\dfrac{1}{2}l\omega_m$($l=1,3,5,\ldots$)が含まれる.



次のように書くことにしよう.
\[
    x_{m+1}(t) = \frac{1}{2} [ \xi_{m+1}(t) + \eta_{m+1}(t) ]
\]
ここで,$\xi_{m+1}$は差分\eqref{eq32.14:2^m+1サイクルのペア間の間隔}であり,
\[
    \eta_{m+1}(t) = x_{m+1}(t) + x_{m+1}(t+T_m)
\]
である.
$\eta_{m+1}(t)$のスペクトルは周波数$k\omega_m$のみを含み,低調波に対するFourier成分
\[
    \frac{1}{T_{m+1}} \int_0^{T_{m+1}} \eta_{m+1}(t) e^{i\pi lt/T_m} \, dt
    = \frac{1}{2T_m} \int_0^{T_m} [ \eta_{m+1}(t)-\eta_{m+1}(t+T_m) ] e^{i\pi lt/T_m} \, dt
\]
は0である($\eta_{m+1}(t+T_m)=\eta_{m+1}(t)$を用いた).
一方,第一近似では,分岐において量$\eta_m(t)$は変化しない:
$\eta_{m+1}(t) \simeq \eta_m(t).$
これは,周波数$k\omega_m$の振動の強さも変化しないことを意味している.



一方,$\xi_{m+1}(t)$のスペクトルは,低調波$\dfrac{1}{2}l \omega_m$のみを含み,倍増段階$m+1$で現れる新しい周波数となる.
これらの成分の合計強度は積分値
\begin{equation}
    I_{m+1} = \frac{1}{T_{m+1}} \int_0^{T_{m+1}} {\xi_{m+1}}^2(t) \, dt
\end{equation}
で与えられる.
$\xi_{m+1}(t)$を$\xi_m(t)$で表すと,次のようになる.
\[
    I_{m+1} = \frac{1}{2T_m} \cdot 2\int_0^{T_m} {\sigma_m}^2(t) {\xi_m}^2(t) \, dt
\]
(32.16)〜(32.18)より
\[
    I_{m+1} = \frac{1}{2} \pqty{ \frac{1}{\alpha^2} + \frac{1}{\alpha^4} } \frac{1}{T_m} \int_0^{T_m} {\xi_m}^2(t) \, dt
    = \frac{1}{2} \pqty{ \frac{1}{\alpha^2} + \frac{1}{\alpha^4} } I_m
\]
であり,最終的に
\begin{equation}\label{eq32.20:ImとIm+1の比}
    \frac{I_m}{I_{m+1}} = 10.8
\end{equation}
となる.
このように,周期倍分岐の後に現れる新しい成分の強さは,分岐数に依存しない明確な因子によって,次の分岐のためのものを上回る(M. J. Feigenbaum 1979)
\footnote{
これは低調波の強さの合計だけでなく,それぞれの低調波にも当てはまる.
分岐$m$の後に現れる各低調波に対して,分岐$m+1$の後には2つ(右に1つ,左に1つ)の低調波が存在する.
したがって,連続する2つの分岐の後に現れる個々のピークの強さの比は\eqref{eq32.20:ImとIm+1の比}の2倍である.
この量のより正確な値は$10.48$である.
これは,普遍関数$g(x)$によって点$\lambda=\Lambda_\infty$自体の状態を分析することによって得られる.
この点では,すべての周波数がすでに存在しており,1つ前の脚注で提起した問題に対応する問題は生じないが,1つの問題がある.
M. Nauenberg and J. Rudnick, \textit{Physical Review} B \textbf{24}, 493, 1981
参照.
}.




ここで,$\lambda$が$\Lambda_\infty$よりもさらに大きくなったとき(Reynolds数$\R>\R_\infty$),乱流域での流れの特性の発達を考えてみよう.
非周期的なアトラクターは,その形成の瞬間($\lambda=\Lambda_\infty$)には1次元のPoincar\'{e}写像で記述されるので,
$\lambda$が$\Lambda_\infty$より少し大きくても,その性質をこの写像で扱うことが許されると考えることができる.




周期倍分岐の無限列が形成するアトラクターは,その外観は\S~31で定義したストレンジアトラクターではない;
$m\to\infty$のときに安定な$2^m$サイクルの極限として生じる$2^\infty$サイクルも安定である.
このアトラクターの点は,区間$[-1,1]$において,無数のCantor集合を形成している.
この間隔の測定値,すなわち要素全体の「長さ」はゼロである;
その次元は0から1の間であり,$0.54$であることがわかっている
\footnote{P. Grassberger, \textit{Journal of Statistical Physics} \textbf{26}, 173, 1981
参照.}.




$\lambda>\Lambda_\infty$のとき,アトラクターはストレンジアトラクター,すなわち不安定な経路の集合を吸引するものとなる.
区間$[-1,1]$において,これに属する点は全長が0でない範囲を占める.
これらの範囲は,連続した2次元の帯の断面平面$\sigma$上の軌跡であり,その帯は多くのターンを持ち,閉じている.
この場合,一次元的な扱いは近似的なものであることを忘れてはならない.
実際には,帯は小さいがゼロではない厚みを持つ.
したがって,その断面を形成する線分は,実際には幅がゼロでない短冊状である.
この幅の向こう側には,\S~\ref{sec:31}で述べたような層状のCantor構造がある
\footnote{この方向のアトラクターの次元は1よりずっと小さいが,これは普遍的な性質ではなく,特定の写像に依存する.}.
この構造には興味がないので,1次元のPoincar\'{e}写像の議論に戻ろう.




$\lambda$が$\Lambda_\infty$を超えて大きくなったときの,ストレンジアトラクターの発達は,一般に次のようになる.
ある$\lambda>\Lambda_\infty$に対して,アトラクターは区間$[-1,1]$内のいくつかの範囲を占める.
これらの範囲の間の空間は吸引領域であり,$2^m$を超えない周期を持つ不安定サイクルの要素を含む.
$\lambda$が増加すると,ストレンジアトラクター上の経路の発散率が増加し,周期$2^m, 2^{m+1}, \ldots$のサイクルを次々と吸収し,「拡大」する.
アトラクターが占める範囲の数は減少し,その長さは増加する.
したがって,上記のバンドのターンは半分になり,その幅は広がる.
このように,アトラクターは逆カスケード的に単純化されていく.
不安定な$2^m$サイクルがアトラクターに吸収されることを,\emph{逆倍分岐}と呼ぶ.
図22は,この過程を2つの連続した逆分岐について示したものである.
図22aでは,バンドは4ターンしているが,逆分岐によって2ターンのバンドになり(図22b),最後の分岐では,1ターンだけしてねじれた後閉じたバンドになる(図22c).





連続する逆倍分岐に対応する$\lambda$の値を$\overline{\Lambda}_{m+1}$と書こう
(ただし$\overline{\Lambda}_m > \overline{\Lambda}_{m+1}$の順に並んでいるとする).
これらは,順方向の分岐と同じ普遍因子$\delta$で幾何級数的に収束していくことがわかる.




最後の($\lambda$が増加する)逆分岐の前に,アトラクターは写像\eqref{eq32.5:1次元写像1-λx^2}の固定点$x_*$を含むギャップによって分けられた2つの範囲を占め,これは周期1の不安定サイクルに対応する:
\[
    x_* = \frac{\sqrt{1+4\lambda}-1}{2\lambda}
\]
この点が膨張アトラクターの境界に達したとき,値$\lambda=\overline{\Lambda}_1$で分岐が起こる.
図22bより,アトラクターバンドの外側の境界は1ループ後に内側の境界となり,別のループ後にはターン間のギャップの境界となることがわかる.
このことから,$\lambda=\overline{\Lambda}_1$は$x_{j+2}=x_*$という条件によって与えられることがわかる.
ここで
\[
    x_{j+2} = 1-\lambda(1-\lambda)^2
\]
は,アトラクターの境界である点$x_j=1$上で写像を2回繰り返した結果であり,$\overline{\Lambda}_1=1.543$である.
先の逆分岐$\overline{\Lambda}_2, \overline{\Lambda}_3, \ldots$は,
$\overline{\Lambda}_{m+1}$と$\overline{\Lambda}_m$の間の漸化式により,次々と近似的に求めることができる.
この近似的な関係は,前述の順倍分岐と同じ方法で導かれ,\eqref{eq32.7:λ1の定義}の関数$\phi(\Lambda)$と同じ
$\overline{\Lambda}_m = \phi(\overline{\Lambda}_{m+1})$
の形となる.
対応するグラフの構成は図21の上段に示す通りである.
$\phi(\Lambda)$は順分枝列と逆分枝列で同じなので,数$\Lambda_m$と$\overline{\Lambda}_m$の列(それぞれ下からと上から)の共通極限
$\Lambda_\infty \equiv \overline{\Lambda}_\infty$への収束を支配する式も同じである.
\begin{equation}
    \overline{\Lambda}_{m+1} - \Lambda_\infty = \frac{1}{\delta} (\overline{\Lambda}_m-\Lambda_\infty)
\end{equation}




$\lambda>\Lambda_\infty$の場合のストレンジアトラクター特性の発達は,それに対応した周波数スペクトルの変化を伴っている.
流れのランダムさは,アトラクターの幅とともに強くなる「ノイズ」成分の存在によって表現される.
この背景には,不安定サイクルの基本周波数とその高調波,低調波に対応する離散的なピークが存在する;
連続する逆分岐では,関連する低調波が順次,順方向の分岐で出現したのとは逆の順序で消失する.
これらの周波数を作り出すサイクルが不安定であることは,ピークの幅が広くなることで示される.



\subsection*{交互流による乱流への遷移}

最後に,乗数が$\mu=+1$を通過するときに,周期流がなくなることを考えよう.



このタイプの分岐は(1次元のPoincar\'{e}写像で)関数$x_{j+1}=f(x_j;\R)$で記述され,
Reynolds数のある値$\R=\Rcr$で直線$x_{j+1}=x_j$に接する.
接点を$x_j=0$とすると,その付近での写像関数は
\begin{equation}\label{eq32.22:交互流での1次元Poincare写像の展開}
    x_{j+1} = (\R-\Rcr) + x_j + {x_j}^2
\end{equation}
と展開される\footnote{$\R-\Rcr$の係数と${x_j}^2$の正の係数は,
$\R$と$x_j$の適切な定義によって1とすることができ,\eqref{eq32.22:交互流での1次元Poincare写像の展開}でも仮定されている.}.
$\R<\Rcr$のとき(図23参照),2つの固定点
\[
    {x_*}^{(1),(2)} = \mp \sqrt{\Rcr-\R}
\]
が存在し,${x_*}^{(1)}$が安定,${x_*}^{(2)}$が不安定な周期流に相当する.
$\R=\Rcr$のとき,両点での乗数は$+1$であり,2つの周期流は合体する.
$\R>\Rcr$のとき,固定点は複素領域に入り込み消滅する.



$\R-\Rcr$が小さいと,曲線\eqref{eq32.22:交互流での1次元Poincare写像の展開}と直線$x_{j+1}=x_j$が接近する($x_j=0$付近).
したがって,この$x$の範囲では,写像\eqref{eq32.22:交互流での1次元Poincare写像の展開}の各反復は経路のトレースをわずかに動かすだけであり,
全範囲をカバーするには多くのステップが必要である.
つまり,比較的長い時間間隔では,経路は状態空間において規則的で,ほぼ周期的である.
このような経路は,物理空間における規則正しい層流に相当する.
このことは,乱流の発生について理論的に可能なもう一つのシナリオをもたらす(P. Manneville and Y. Pomeau 1980).




写像関数の特定の領域は,経路をランダム化する領域と隣接しており,状態空間において局所的に不安定な経路の集合に対応することが想像される.
しかし,この集合自体はアトラクターではなく,時間の経過とともに系を表す点は集合から外れていく.
$\R<\Rcr$のとき,経路は安定なサイクルに達し,物理空間には周期的な層流が確立される.
$\R>\Rcr$のとき,安定なサイクルはなく,乱流と層流が交互に現れる運動が生じ,このシナリオは交互流による乱流への遷移と呼ばれている.



乱流周期の長さについては,一般的な結論は得られない.
しかし,層流の持続時間の$\R-\Rcr$に対する依存性は簡単に知ることができる.
そのためには,差分方程式\eqref{eq32.22:交互流での1次元Poincare写像の展開}を微分方程式として書き下せばよい.
$x_j$は1写像ステップでわずかに変化するだけであるから,$x_{j+1}-x_j$を連続変数$t$に関する微分$\ddv{x}{t}$に置き換える.
\begin{equation}
    \dv{x}{t} = \R-\Rcr + x^2
\end{equation}
$x=0$の両側にある点$x_1$と$x_2$の間の区間を,$\R-\Rcr$よりずっと大きな距離で,
しかしまだ展開\eqref{eq32.22:交互流での1次元Poincare写像の展開}が有効な範囲内で,横断するのに必要な時間$\tau$を求めよう.このとき
\[
    \tau = \frac{1}{\sqrt{\R-\Rcr}} \bqty{ \tan^{-1} \pqty{\frac{x}{\sqrt{\R-\Rcr}}} }_{x_1}^{x_2}
\]
であるから
\begin{equation}
    \tau \propto \frac{1}{\sqrt{\R-\Rcr}}
\end{equation}
となり,必要な依存性が得られる.
したがって,$\R-\Rcr$が大きくなるにつれて層流の持続時間は短くなる.



このシナリオでは,始点への近づき方と発生する乱流の性質が未解決のままである.



\BackToTheToc