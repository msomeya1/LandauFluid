\section{\spade 定常流の安定性(日本語訳)}\label{sec:26}
与えられた定常条件下での粘性流のいかなる問題に対しても,原理的には流体力学の方程式の厳密な定常解が存在しなければならない.
これらの解は,形式的にはすべてのReynolds数に対して存在する.
しかし,運動方程式の解は,たとえ厳密なものであっても,自然界で実際に起こりうるものばかりではない.
そのような解は,流体力学の方程式に従うだけでなく,安定でなければならない.
どんな小さな摂動が起きても,時間の経過とともに減少していかなければならない.
逆に,流れに必然的に発生する小さな摂動が時間とともに大きくなるようであれば,その流れは不安定であり,実際には存在し得ない
\footnote{
第1版では,無限小の摂動に対する不安定性を\emph{絶対不安定}と呼んだ.
この形容詞は現在の文脈では使われないが,(より慣習的な用語にしたがって)\emph{対流}(\S~28)との対比としては機能する.
}.


ある流れの無限小摂動に対する安定性を数学的に調べるには,次のようにする.
定常解(その速度分布を$\v_0(\vec{r})$とする)に,非定常の小さな摂動$\v_1(\vec{r},t)$を重ね,
そうしてできた速度$\v=\v_0+\v_1$が運動方程式を満足しなければならない.
$\v_1$に関する方程式は,
\begin{equation}\label{eq26.1:NS方程式}
    \pdv{\v}{t} + (\v\cdot\Grad)\v = -\frac{\Grad p}{\rho} + \nu\Laplacian\v, \quad \Div\v=0
\end{equation}
に速度および圧力
\begin{equation}\label{eq26.2:平均場と摂動場の分離}
    \v = \v_0+\v_1, \quad p = p_0+p_1
\end{equation}
を代入して得られる.ここで,既知の関数$\v_0$と$p_0$は,摂動のない方程式
\begin{equation}
    (\v_0\cdot\Grad)\v_0 = -\frac{\Grad p_0}{\rho} + \nu\Laplacian\v_0, \quad \Div\v_0=0
\end{equation}
を満たしている.
$\v_1$の1次以上の項を省略すると次のようになる.
\begin{equation}\label{eq26.4:摂動場の満たす方程式}
    \pdv{\v_1}{t} + (\v_0\cdot\Grad)\v_1 + (\v_1\cdot\Grad)\v_0
    = -\frac{\Grad p_1}{\rho} + \nu\Laplacian\v_1 , \quad \Div\v_1=0
\end{equation}
境界条件は,$\v_1$が固定された固体表面で0となることである.


このように,$\v_1$は斉次線形微分方程式系を満たし,係数は座標の関数だけで時間に依存しない.
このような方程式の一般的な解は,$\v_1$が時間に対して$e^{-i\omega t}$のように依存する特解の和として表すことができる.
摂動の振動数$\omega$は任意ではなく,方程式\eqref{eq26.4:摂動場の満たす方程式}を適切な境界条件とともに解くことによって決定される.
振動数は一般に複素数である.
虚部が正である$\omega$が存在すれば,$e^{-i\omega t}$は時間とともに無限に増加する.
つまり,このような摂動は一度生じると増大し,流れはこのような摂動に対して不安定である.
流れが安定であるためには,あらゆる可能な振動数$\omega$の虚部が負であることが必要である.
そうすれば,発生する摂動は時間とともに指数関数的に減少する.


しかし,このような安定性の数学的検討は非常に複雑である.
有限の大きさを持つ物体のまわりの定常流の安定性に関する理論的な問題は,まだ解決されていない.
Reynolds数が十分に小さい場合には,定常流が安定であることは確かである.
実験データによると,$\R$が大きくなってある値$\Rcr$(\emph{臨界Reynolds数})に達し,それを超えると無限小摂動に対して流れが不安定になるようである.
したがって,Reynolds数が十分に大きい場合($\R>\Rcr$),固体まわりの定常流は不可能である.
臨界Reynolds数はもちろん普遍的な定数ではなく,流れの種類によって異なる値をとり,10--100程度のオーダーのようである.
例えば円柱のまわりの流れでは,$\R = ud/\nu \simeq 30$($d$は円柱の直径)で減衰しない非定常流が観察されている.

ここで,大きなReynolds数で定常流が不安定になることによって生じる非定常流の性質について考えてみよう(L. D. Landau 1944).
まず,$\Rcr$よりわずかに大きいReynolds数におけるこの流れの性質を調べることから始める.
$\R<\Rcr$の場合,すべての可能な小さな摂動に対する複素振動数$\omega=\omega_1+i\gamma_1$の虚部は負($\gamma_1<0$)である.
$\R=\Rcr$の場合,虚部が0となる振動数が1つ存在する.
$\R>\Rcr$の場合,この振動数の虚部は正だが,$\R$が$\Rcr$に近いと$\gamma_1$は実部$\omega_1$に比べて小さくなる
\footnote{
ある流れに対して起こりうる摂動振動数の集合(\emph{スペクトル})には,飛び飛びの値(\emph{離散スペクトル})と様々な振動数の領域全体(\emph{連続スペクトル})の両方が含まれる.
有限の大きさの物体まわりの流れでは,$\gamma_1>0$の振動数は離散スペクトルの中にしか存在しないように思われる.
その理由は連続スペクトルの振動数に対応する摂動は一般に無限遠で0にならないからであるが,そこでの摂動のない流れは確かに安定で均質な平面平行流である.
}.
この振動数に対応する関数$\v_1$は次のような形をとる.
\begin{equation}\label{eq26.5:摂動場の時間・空間依存性}
    \v_1 = A(t) \vec{f}(x,y,z)
\end{equation}
ここで$\vec{f}$は座標のある複素関数であり,複素振幅$A(t)$は
\footnote{例によって,振幅と言ったときには\eqref{eq26.6:摂動場の複素振幅A}の実部と理解する.}
\begin{equation}\label{eq26.6:摂動場の複素振幅A}
    A(t) = \const \times e^{\gamma_1t}e^{-i\omega_1t}
\end{equation}
で与えられる.
しかし,この$A(t)$の式が成り立つのは,実は定常流が途絶えた後の短い時間だけで,因子$e^{\gamma_1t}$は時間とともに急激に増加し,
\eqref{eq26.5:摂動場の時間・空間依存性},\eqref{eq26.6:摂動場の複素振幅A}のような式を導く上記の$\v_1$の決定方法は$\abs{\v_1}$が小さいときだけ適用される.
もちろん現実には,非定常流の振幅の絶対値$\abs{A}$は際限なく増加するわけではなく,有限の値になる傾向がある.
$\Rcr$に近い$\R$の場合(もちろん常に$\R>\Rcr$を意味する),この有限の値は小さく,次のように決定することができる.


二乗振幅$\abs{A}^2$の時間微分を求めよう.
$t$が非常に小さく,\eqref{eq26.6:摂動場の複素振幅A}がまだ有効なとき,$\ddv{\abs{A}^2}{t}=2\gamma_1\abs{A}^2$が得られる.
この式は,実際には$A$と$A^*$の冪展開の第1項に過ぎない.
絶対値$\abs{A}$が(小さいながらも)大きくなると,展開の次の項を考慮する必要がある.
次の項は$A$の3次の項である.
しかし,微分$\ddv{\abs{A}^2}{t}$の正確な値に興味があるのではなく,因子$e^{-i\omega_1t}$の周期$2\pi/\omega_1$に比べて十分長い時間での時間平均に興味がある.
ところで$\omega_1 \gg \gamma_1$であるから,この周期は$\abs{A}$が大きく変わる時間$1/\gamma_1$に比べて短い.
しかし,3次の項には周期的な項が含まれるため,平均化すると消滅してしまう
\footnote{厳密には,3次の項は平均化すると0ではなく4次の項を与えるが,これは展開の4次の項の中に含まれていると考える.}.
4次の項には,$A^2{A^*}^2 = \abs{A}^4$に比例し,平均化しても消滅しない項が含まれている.
したがって,4次の項まで含めれば
\begin{equation}\label{eq26.7:Landau方程式}
    \overline{ \dv{\abs{A}^2}{t} } = 2\gamma_1 \abs{A}^2 - \alpha \abs{A}^4 .
\end{equation}
ここで,$\alpha$(\emph{Landau定数})は,正または負である.


我々は,(元の流れに重なった)無限小の摂動が$\R>\Rcr$で最初に不安定になるケースに興味がある.
これは$\alpha>0$に対応する.
平均操作は$1/\gamma_1$に比べて短い時間間隔でしか行われないので,\eqref{eq26.7:Landau方程式}の$\abs{A}^2$と$\abs{A}^4$の上には平均を表すバーを付けなかった.
同じ理由で,微分の上のバーも省略されたものとして方程式を解く.
式\eqref{eq26.7:Landau方程式}の解は
\[
    \frac{1}{\abs{A}^2} = \frac{\alpha}{2\gamma_1} + \const\times e^{-2\gamma_1t}
\]
である.
従って,$\abs{A}^2$が漸近的に有限の極限値に近づくことは明らかである:
\begin{equation}\label{eq26.8:摂動場の複素振幅Aの漸近値}
    {\abs{A}_\textrm{max}}^2 = \frac{2\gamma_1}{\alpha}
\end{equation}


量$\gamma_1$はReynolds数の関数である.
$\Rcr$の近傍では$\R-\Rcr$の冪で展開することができる.
しかし,臨界Reynolds数の定義から$\gamma_1(\Rcr)=0$であるから,1次までとると
\begin{equation}\label{eq26.9:Landau方程式のγのRによる展開}
    \gamma_1 = \const\times(\R-\R_\textrm{cr})
\end{equation}
となる.
これを\eqref{eq26.8:摂動場の複素振幅Aの漸近値}に代入すると,振幅の絶対値$A$は$\R-\Rcr$の平方根に比例することがわかる.
\begin{equation}\label{eq26.10:摂動場の複素振幅AのRによる展開}
    \abs{A}_\textrm{max} \propto \sqrt{\R-\R_\textrm{cr}}
\end{equation}


ここで,\eqref{eq26.7:Landau方程式}の$\alpha<0$の場合について簡単に議論しよう.
この展開の2項は摂動の振幅の極限値を決定するには不十分であり,高次の負の項を入れる必要がある;
これを$-\beta\abs{A}^6$(ただし$\beta>0$)とすると次のようになる.
\begin{equation}
    \abs{A}_\textrm{max}^2 = \frac{\abs{\alpha}}{2\beta} \pm \sqrt{ \frac{\alpha^2}{4\beta^2} + \frac{2\abs{\alpha}}{\beta} \gamma_1 }
\end{equation}
$\gamma_1$は\eqref{eq26.9:Landau方程式のγのRによる展開}と同じものである.
図13bにその依存性を示す;図13aは$\alpha>0$,\eqref{eq26.10:摂動場の複素振幅AのRによる展開}に対応する.
$\R>\Rcr$のときは定常流が存在せず,
$\R=\Rcr$のときは摂動が不連続にゼロでない振幅になるが,それでも$\abs{A}^2$の冪展開が成り立つほど小さいと仮定している
\footnote{このような系は\emph{ハード}な自励振動と呼ばれ,\emph{ソフト}な自励振動が無限小の摂動に対して不安定であるのと対照的である.}.
$\Rcr'<\R<\Rcr$の範囲では,摂動のない流れは\emph{準安定}で,無限小の摂動に対しては安定だが,有限振幅の摂動に対しては不安定である
(連続な曲線;破線は不安定枝を示す).



$\R>\Rcr$のときの,小さな摂動に対する不安定の結果起こる非定常流に話を戻そう.
$\Rcr$に近い$\R$の場合,後者の流れは,定常流$\v_0(\vec{r})$に,\eqref{eq26.10:摂動場の複素振幅AのRによる展開}のように$\R$とともに増加する,小さいが有限の振幅を持つ周期流$\v_1(\vec{r},t)$を重ねることによって表現できる.
この流れの速度分布は次のような形になる.
\begin{equation}
    \v_1 = \vec{f}(\vec{r}) e^{-i(\omega_1t+\beta_1)}
\end{equation}
ここで,$\vec{f}$は座標の複素関数,$\beta_1$は何らかの初期位相である.
$\R-\Rcr$が大きいと,速度を$\v_0$と$\v_1$に分離することはもはや意味がない.
この場合,振動数$\omega_1$の周期的な流れが発生することになる.
時間の代わりに位相$\phi_1 \equiv \omega_1t+\beta_1$を独立変数として用いれば,関数$\v_1(\vec{r},\phi_1)$は周期$2\pi$の$\phi_1$に関する周期関数であると言える.
しかし,この関数は,もはや単純な三角関数ではない.Fourier級数で展開すると
\begin{equation}
    \v = \sum_p \vec{A}_p(\vec{r}) e^{-i\phi_1p}
\end{equation}
(ここで,和は正負のすべての整数$p$に対してとる)であり,
基本振動数$\omega_1$を持つ項だけでなく,振動数が$\omega_1$の整数倍の項も含む.


式\eqref{eq26.7:Landau方程式}は,時間因子$A(t)$の絶対値のみを決定し,その位相$\phi_1$は決定せず本質的に不定のままで,流れが始まる瞬間にたまたま起こる特定の初期条件に依存する.
初期位相$\beta_1$は,これらの条件によって,どのような値にもなりうる.
このように周期的な流れは,その流れが起こる定常的な外部条件によって一義的に決まるものではない.
速度の初期位相という1つの量が任意に残っているのである.
流れが1つの自由度を持つのに対して,外部条件によって完全に決定される定常流は,自由度を持たないと言ってよいかもしれない.

%%%%%%%%%% 問題1 %%%%%%%%%%

\begin{mondai}{}{}
摂動のない流れと摂動の間のエネルギーバランスの方程式を,摂動が弱いことを仮定せずに導け.
\end{mondai}
\begin{kaitou}
\eqref{eq26.2:平均場と摂動場の分離}を\eqref{eq26.1:NS方程式}に代入し,$\v_1$の2次の項を残すと
\[
    \pdv{\v_1}{t} + (\v_0\cdot\Grad)\v_1 + (\v_1\cdot\Grad)\v_0 + (\v_1\cdot\Grad)\v_1
    = -\Grad p_1 + \frac{1}{\R} \Laplacian\v_1
    \mytag{1}
\]
となる.ただし,\S~19で述べたような無次元の形で考えるものとする.
この式と$\v_1$の内積をとり,$\Div\v_0=0,\,\Div\v_1=0$を用いると,次のようになる.
\[
    \pdv{t} \pqty{\frac{1}{2}{v_1}^2} = -v_{1i}v_{1k} \pdv{v_{0i}}{x_k} - \frac{1}{\R}\pdv{v_{1i}}{x_k}\pdv{v_{1i}}{x_k}
    + \pdv{x_k} \bqty{ -\frac{1}{2}{v_1}^2(v_{0k}+v_{1k}) -p_1v_{1k} + \frac{1}{\R}v_{1i}\pdv{v_{1i}}{x_k} }
\]
右辺の最後の項は,流れの全領域で積分すると,その領域の境界面や無限遠では$\v_0=\v_1=0$となるので,0になる.
このことから,必要な関係式として
\[
    \dot{E_1} = T - \frac{D}{\R}
    \mytag{2}
\]
\[
    E_1 = \int \frac{1}{2}{v_1}^2 \dV, \quad
    T = -\int v_{1i}v_{1k} \pdv{v_{0i}}{x_k} \dV, \quad
    D = \int \pqty{\pdv{v_{1i}}{x_k}}^2 \dV
    \mytag{3}
\]
を得る.
関数$T$は,平均流と摂動流の間のエネルギー交換を表し,正負どちらの符号にもなり得る.
関数$D$は散逸エネルギー損失であり,常に$D>0$である.
なお,\ajMaru{1}の$\v_1$に非線形の項は\ajMaru{2}の関係式には寄与していないことに注意.

式\ajMaru{2}は$\Rcr$の下限を与える(O. Reynolds 1894; W. M'F. Orr 1907):
微分$\ddv{E_1}{t}$は$\R<\R_E$ならば負,つまり摂動は時間とともに減少しなければならない.
ここで
\[
    \R_E = \min\pqty{\frac{D}{T}}
    \mytag{4}
\]
は境界条件と方程式$\Div\v_1=0$を満たす関数$\v_1(\vec{r})$に対して取られる関数で,
有限な最小値の存在は,$T$と$D$がともに2次の同次関数であるという事実から数学的に言える.
このことは,準安定性の下限$\R$の存在を証明し,それ以下では摂動がない流れはいかなる摂動に対しても安定であることを示す.
しかし,\ajMaru{4}で与えられる「エネルギーの推定値」は,ほとんどの場合において,低すぎる.

\end{kaitou}



\BackToTheToc