\section{\spade パイプ内の流れの安定性(日本語訳)} 
\S~17で議論したパイプ内の定常流は,異常な形で安定性を失う.
流れは$x$方向(パイプに沿った方向)に一様なので,乱れのない速度分布$\v_0$は$x$に依存しない.
したがって,\S~27の手順と同様に,式(26.4)の解を求めることができる.
\begin{equation}
    \v_1 = e^{i(kx-\omega t)} \vec{f}(y,z)
\end{equation}
ここでも,ある$k$の値で$\gamma=\Im(\omega)$が最初にゼロになる値$\R=\Rcr$が存在する.
しかし,関数$\omega(k)$の実部がゼロにならないことが重要である.


$\Rcr$をわずかに超える$\R$の値では,$\gamma(k)>0$となる$k$の値の範囲は小さく,
$\omega(k)$が最大となる点,すなわち$\ddv{\gamma}{k}=0$の近くにある(図16からわかるように).
流れの一部にわずかな摂動が生じたとすると,それは(28.1)の形の成分を重ね合わせた波束となる.
時間の経過とともに,$\gamma(k)>0$となる成分は増幅され,それ以外の成分は減衰する.
このようにして増幅された波束は,波束の群速度$\ddv{\omega}{k}$に等しい速度で下流に運ばれる(\S~67)が,
ここでは波数が$\ddv{\gamma}{k}=0$となる点付近の小さな範囲にある波を考えているので,量
\begin{equation}
    \dv{\omega}{k} \simeq \dv{\Re(\omega)}{k}
\end{equation}
は実数であり,したがってパケットの実際の伝搬速度である.


この摂動の下流への変位は非常に重要で,\S~27で説明したものとは全く異なる安定性の喪失を引き起こす.



$\Im(\omega)$の正値は今,下流に向かう摂動の増幅のみを意味するので,2つの可能性がある.
一つは,波束の移動にもかかわらず,空間のどの点でも時間の経過とともに摂動が際限なく大きくなるケースで,
このような無限小の摂動に対する不安定性は\emph{絶対不安定}と呼ばれる.
一方,パケットは非常に速く流され,空間のどの点でも$t\to\infty$で摂動がゼロになる.
これは\emph{対流不安定}と呼ばれる
\footnote{不安定性の種類を決定する一般的な方法は,『物理的運動学』\S~62で説明されている.}.
Poiseuille流の場合は,2番目の種類が発生するようである;次の脚注を参照.



ある座標系で対流した不安定はパケットとともに運動する別の座標系では絶対的な不安定となり,
絶対的な不安定はパケットから十分な速度で遠ざかる座標系では対流するようになる,という意味で,
両者の違いは,不安定を考慮する参照座標系の選択によって決まる相対的なものである.
しかし,今回のケースでは,不安定性を見るべき好ましい参照座標系,
すなわち管壁が静止している参照座標系が存在することによって,その違いに物理的な意味が与えられているのである.
また,実際のパイプは長さが有限であるため,どこかで発生した摂動は,原理的には実際に層流を乱す前に配管の外に持ち出される可能性がある.



擾乱は,ある点での時間ではなく,座標$x$(下流側)に対して増加するので,この種の不安定性を調べるには,次のようにするのが合理的である.
ある地点で,与えられた周波数$\omega$の連続的に働く摂動が流れに加えられたと仮定して,この摂動が下流に運ばれるときにどうなるかを調べてみる.
関数$\omega(k)$を反転させると,与えられた(実)周波数$\omega$に対応する波数$k$がわかる.
$\Im(k)<0$ならば,係数$e^{ikx}$は$x$とともに増加し,つまり摂動は下流に増幅される.
方程式$\Im k(\omega,\R)=0$で与えられる$\omega\R$平面上の曲線は,\emph{中立安定曲線}または\emph{中立曲線}と呼ばれ,
安定領域を定義し,各$\R$について,下流で増幅される摂動と減衰される摂動の周波数を分離するものである.



実際の計算は非常に複雑である.
完全な解析的検討は,平面Poiseuille流(2つの平行な平面の間;C. C. Lin 1945)に対してのみ行われている.
ここではその結果を示すことにする
\footnote{C. C. Lin, \textit{The Theory of Hydrodynamic Stability}, Cambridge 1955 参照.
これらとその後の研究については,前の脚注で触れたDrazinとReidの本に記載がある.}.




平面間の(乱れのない)流れは,流れの方向($x$軸に沿って)だけでなく,$xz$平面全体($y$軸は平面に垂直)において一様である.
したがって,(26.4)式の解は次のような形で求めることができる.
\begin{equation}
    \v_1 = e^{i(k_xx+k/_zz-\omega t)} \vec{f}(y)
\end{equation}
という形の解を求めることができる.
波動ベクトル$\vec{k}$は$xz$平面上の任意の方向である.
しかし,$\R$が大きくなるにつれて最初に現れる成長する摂動にのみ興味がある.
なぜなら,これが安定性の限界を支配しているからである.
ある波数において,減衰しない最初の摂動は$x$方向に$\vec{k}$を持ち,$f_x=0$であることが示される.
したがって,$xy$平面上の摂動のみを考えればよく,$z$に依存しない2次元の摂動(無摂動流と同じ)を考えればよい
\footnote{
このことの証明(H. B. Squire 1933)として,
(28.3)のような摂動を持つ方程式(26.4)は,2次元摂動の方程式と$\R$を$\R\cos\phi$に置き換えるだけで異なる形になり,
$xz$平面での$\vec{k}$と$\v_0$間の角度を$\phi$とすることができる,ということがある.
したがって,与えられた$k$を持つ3次元摂動の臨界数$\tilde{\R}_{\mathrm{cr}}$は,
$\tilde{\R}_{\mathrm{cr}} = \Rcr \sec\phi > \Rcr$であり,ここで$\Rcr$は2次元摂動に対して計算されたものである.}.




図17に面間の流れの中立曲線を模式的に示す.
曲線内の斜線部分が不安定領域である
\footnote{
$k\R$平面での中立曲線も同様の形をしている.
中立曲線上では$\omega$も$k$も実数なので,2つの平面の曲線は同じ依存性を異なる変数で表現したものである.}.
減衰しない摂動が可能な$\R$の最小値は,S. A. Orszag(1971)による後のより正確な計算によれば,$\Rcr=5772$であることが判明した;
Reynolds数はここで次のように定義される.
\begin{equation}
    \R = U_\mathrm{max} h/2\nu,
\end{equation}
ここで$U_\mathrm{max}$は最大流速,
$\dfrac{1}{2}h$は平面間の距離の半分,すなわち流速がゼロから最大値まで増加する距離である
\footnote{2次元のPoiseuille流に対する$\R$の定義も文献にある.
$\R=\overline{U}h/\nu$,ここで$\overline{U}$は断面上で平均化された流体速度である.
$\overline{U}=\dfrac{2}{3}U_\mathrm{max}$なので,(28.4)に従って$\R$を定義すると$\overline{U}h/\nu = \dfrac{4}{3}\R$となる.}.
$\R=\Rcr$は擾乱波数$k_\mathrm{cr}=2.04/h$に相当する.
$\R\to\infty$として,中立曲線の2つの枝は$\R$軸に漸近し,上下の枝はそれぞれ$\omega h/U_\mathrm{max} \simeq \R^{-3/11}$と$\R^{-3/7}$で,
各枝において$\omega$と$k$は$\omega h/U \simeq (kh)^3$により関係づけられる.




したがって,ある最大値($\sim U/h$)を超えないゼロでない周波数$\omega$に対して,摂動が増幅される有限の$\R$の範囲が存在することになる
\footnote{
2次元Poiseuille流の不安定性が対流であることの証明は,
S. V. Iordanski\v{i} and A. G. Kulikovski\v{i}, \textit{Soviet Physics JETP} \textbf{22}, 915, 1966
にある.
しかし,この証明は,中性曲線の2つの枝が横軸に近い,つまり,それぞれの枝で$kh \ll 1$である,非常に大きな$\R$の範囲にのみ関連している.
この問題は,中性曲線上で$kh \sim 1$となるような$\R$の値では未解決のままである.}.
この場合,流体の粘性が小さくても有限であることは,ある意味で,厳密に理想的な流体の状況に比べて不安定にする効果があることは注目される
\footnote{この性質はHeisenberg(1924)によって発見された.}.
$\R\to\infty$のとき,任意の有限な周波数の摂動は減衰するが,有限の粘性を導入するとやがて不安定領域に達し,
さらに粘性を増加させる($\R$を減少させる)とようやくこの領域から抜け出ることができるからである.





円形断面を持つ管内流れの安定性に関する完全な理論的研究はまだなされていないが,
利用可能な結果から,流れは任意のReynolds数における無限小の摂動に対して安定性(絶対および対流)を持つと仮定する十分な根拠を与えてくれる.
擾乱のない流れが軸対称であるとき,擾乱は次のような形で求めることができる.
\begin{equation}
    \v_1 = e^{i(n\phi+kz-\omega t)} \vec{f}(r)
\end{equation}
(27.4).
軸対称の摂動($n=0$)は常に減衰することが証明されたと考えてよい.
これまで研究されてきた軸対称でない摂動(特定の$n$の値,特定のReynolds数の範囲)の中にも,減衰しない摂動は見つかっていない.
パイプ内の流れの安定性は,パイプの入り口での擾乱を注意深く防げば,非常に大きな$\R$の値,実際には$\R \simeq 10^5$まで層流を維持できるという事実によっても示唆されている.
この場合 
\begin{equation}
    \R = U_\mathrm{max} d/2\nu = \overline{U}d/\nu
\end{equation}
$d$はパイプの直径,$U_\mathrm{max}$はパイプ軸上の流体速度である.



平面間の流れと円管内の流れは,半径$R_1$と$R_2$($R_2>R_1$)をもつ2つの同軸円筒面間の環状管内の流れの限定例とみなすことができる.
$R_1=0$では円管となり,極限$R_1 \to R_2$が平面間の流れに対応することになる.
$R_1/R_2 < 1$のすべての非ゼロ値に対して,$R_1/R_2 \to 0, \Rcr \to \infty$のとき,臨界値$\Rcr$が存在するようである.




これらのPoiseuille流にはそれぞれ,有限振幅の摂動に対する安定性の限界を決める臨界数$\Rcr'$も存在する.
$\R<\Rcr'$のとき,配管内の非減衰非定常流は不可能である.
もし乱流が発生した場合,$\R<\Rcr'$では乱流領域は下流に運ばれて小さくなり,完全に消滅するが,
$\R>\Rcr'$では乱流領域は時間の経過とともに大きくなり,より多くの流れを含むようになる.
流れの擾乱がパイプの入り口で絶えず起こる場合,$\R<\Rcr'$であれば,最初はどんなに強くても,パイプの下方のある距離で減衰することになる.
一方,$\R>\Rcr'$であれば,流れはパイプ全体で乱流となり,$\R$が大きければ,より弱い摂動で実現できる.
$\Rcr'$と$\Rcr$の間の範囲では,層流は準安定的である.
円形断面の管では,$\R \simeq 1800$で減衰しない乱流が観測され,$\R \simeq 1000$以上では平行な平面間の流れが観測されている.




パイプ内の層流の崩壊は「ハード」であるため,抗力の不連続な変化を伴う.
$\R>\Rcr'$の管内流れの場合,抗力の$\R$に対する依存性は,層流の場合と乱流の場合の本質的に2つある(\S~43参照).
抗力は,どのような$\R$の値であっても,不連続性を持っており,一方から他方への変化が発生する.



この節を終えるにあたって,もう一つ述べておくことがある.
無限に長いパイプの流れに対して得られた安定限界(中立曲線)には,もう一つ意味がある.
長さは幅に比べて非常に大きいが,有限であるパイプの中の流れを考えてみよう.
両端に速度分布の指定による境界条件を与える(たとえば,パイプの両端を多孔質シールで閉じて一様な分布にする)と,
パイプの両端付近以外の場所では,乱れのない速度分布は$x$に依存しないPoiseuille形式をとると考えることができる.
このように定義された有限系に対して,無限小の摂動に対する安定性の問題を提案することができる.
このような大域的安定性の条件を確立するための一般的な手順は,『物理的運動学』\S~65で述べられている.
無限パイプに対する前述の中立曲線は,その両端の特定の境界条件がどうであれ,有限パイプの大域的安定性の限界でもあることが示せる
\footnote{参照.}.



\BackToTheToc
